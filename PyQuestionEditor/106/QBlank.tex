% !TEX encoding = UTF-8 Unicode
% !TEX TS-program = xelatex 
\begin{QUESTIONS}
    \begin{QUESTION}
        \begin{ExamInfo}{106}{學測}{選填}{A}
        \end{ExamInfo}
        \begin{QBODY}
            遞迴數列$\langle {{a}_{n}}\rangle $滿足${{a}_{n}}={{a}_{n-1}}+f(n-2)$,其中$n\ge 2$且$f(x)$為二次多項式。若${{a}_{1}}=1,{{a}_{2}}=2,{{a}_{3}}=5,{{a}_{4}}=12$,則${{a}_{5}}= \TCNBOX{ \TCN\TCN }$     。
        \end{QBODY}
        \begin{QFROMS}
        \end{QFROMS}
        \begin{QTAGS}
        \end{QTAGS}
        \begin{QANS}
            $25$
        \end{QANS}
        \begin{QSOL}
        \end{QSOL}
        \begin{QEMPTYSPACE}
        \end{QEMPTYSPACE}
    \end{QUESTION}
    \begin{QUESTION}
        \begin{ExamInfo}{106}{學測}{選填}{B}
        \end{ExamInfo}
        \begin{QBODY}
            坐標平面上,$ABC$內有一點$P$滿足$\lvec{AP}=(\frac{4}{3},\frac{5}{6})$及$\lvec{AP}=\frac{1}{2}\lvec{AB}+\frac{1}{5}\lvec{AC}$。若$A,P$連線交$\overline{BC}$於$M$,則$\lvec{AM}= \TCNBOX{\left(\FR{\TCN\TCN}{\TCN\TCN} ,\FR{\TCN\TCN}{\TCN\TCN} \right)}$。(化成最簡分數)
        \end{QBODY}
        \begin{QFROMS}
        \end{QFROMS}
        \begin{QTAGS}
        \end{QTAGS}
        \begin{QANS}
            $(\frac{40}{21},\frac{25}{21})$
        \end{QANS}
        \begin{QSOL}
        \end{QSOL}
        \begin{QEMPTYSPACE}
        \end{QEMPTYSPACE}
    \end{QUESTION}
    \begin{QUESTION}
        \begin{ExamInfo}{106}{學測}{選填}{C}
        \end{ExamInfo}
        \begin{QBODY}
            若$a$為正整數且方程式$5{{x}^{3}}+(a+4){{x}^{2}}+ax+1=0$的根都是有理根,則$a= \TCNBOX{\TCN}$     。
        \end{QBODY}
        \begin{QFROMS}
        \end{QFROMS}
        \begin{QTAGS}
        \end{QTAGS}
        \begin{QANS}
            $7$
        \end{QANS}
        \begin{QSOL}
        \end{QSOL}
        \begin{QEMPTYSPACE}
        \end{QEMPTYSPACE}
    \end{QUESTION}
    \begin{QUESTION}
        \begin{ExamInfo}{106}{學測}{選填}{D}
        \end{ExamInfo}
        \begin{QBODY}
            設${{a}_{1}},{{a}_{2}},\cdots ,{{a}_{9}}$為等差數列且$k$為實數。若方程組
        $\left\{ \begin{aligned}
        & {{a}_{1}}x-{{a}_{2}}y+2{{a}_{3}}z=k+1 \\ 
        & {{a}_{4}}x-{{a}_{5}}y+2{{a}_{6}}z=-k-5 \\ 
        & {{a}_{7}}x-{{a}_{8}}y+2{{a}_{9}}z=k+9 \\ 
        \end{aligned} \right.$有解,
        則$k=\TCNBOX{\TCN\TCN}$      。
        \end{QBODY}
        \begin{QFROMS}
        \end{QFROMS}
        \begin{QTAGS}
        \end{QTAGS}
        \begin{QANS}
            $-5$
        \end{QANS}
        \begin{QSOL}
        \end{QSOL}
        \begin{QEMPTYSPACE}
        \end{QEMPTYSPACE}
    \end{QUESTION}
    \begin{QUESTION}
        \begin{ExamInfo}{106}{學測}{選填}{E}
        \end{ExamInfo}
        \begin{QBODY}
            設$a,b,x$皆為正整數且滿足$a\le x\le b$及$b-a=3$。若用內插法從$\log a,\log b$求得$\log x$的
        近似值為
        $\log x\approx \frac{1}{3}\log a+\frac{2}{3}\log b=\frac{1}{3}(1+2\log 3-\log 2)+\frac{2}{3}(4\log 2+\log 3)$,
        則$x$的值為 $\TCNBOX{\TCN\TCN}$        。
        \end{QBODY}
        \begin{QFROMS}
        \end{QFROMS}
        \begin{QTAGS}
        \end{QTAGS}
        \begin{QANS}
            $47$
        \end{QANS}
        \begin{QSOL}
        \end{QSOL}
        \begin{QEMPTYSPACE}
        \end{QEMPTYSPACE}
    \end{QUESTION}
    \begin{QUESTION}
        \begin{ExamInfo}{106}{學測}{選填}{F}
        \end{ExamInfo}
        \begin{QBODY}
            一隻青蛙位於坐標平面的原點,每步隨機朝上、下、左、右跳一單位長,總共跳了四步。青蛙跳了四步後恰回到原點的機率為$\TCNBOX{\FR{\TCN}{\TCN\TCN}}$。(化成最簡分數)
        \end{QBODY}
        \begin{QFROMS}
        \end{QFROMS}
        \begin{QTAGS}
        \end{QTAGS}
        \begin{QANS}
            $\frac{9}{64}$
        \end{QANS}
        \begin{QSOL}
        \end{QSOL}
        \begin{QEMPTYSPACE}
        \end{QEMPTYSPACE}
    \end{QUESTION}
    \begin{QUESTION}
        \begin{ExamInfo}{106}{學測}{選填}{G}
        \end{ExamInfo}
        \begin{QBODY}
            地面上甲、乙兩人從同一地點同時開始移動。甲以每秒4公尺向東等速移動,乙以每秒3公尺向北等速移動。在移動不久之後,他們互望的視線被一圓柱體 建築物阻擋了6秒後才又相見。此圓柱體建築物底圓的直徑為$\TCNBOX{\TCN\TCN.\TCN}$公尺。
        \end{QBODY}
        \begin{QFROMS}
        \end{QFROMS}
        \begin{QTAGS}
        \end{QTAGS}
        \begin{QANS}
            $14.4$
        \end{QANS}
        \begin{QSOL}
        \end{QSOL}
        \begin{QEMPTYSPACE}
        \end{QEMPTYSPACE}
    \end{QUESTION}

\end{QUESTIONS}