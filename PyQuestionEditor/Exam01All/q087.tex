% !TEX encoding = UTF-8 Unicode
% !TEX TS-program = xelatex 
\begin{QUESTIONS}
    \begin{QUESTION}
        \begin{ExamInfo}{87}{學測}{單選}{1}
        \end{ExamInfo}
        \begin{ExamAnsRateInfo}{}{}{}{}
        \end{ExamAnsRateInfo}
        \begin{QBODY}
            當$x$介於0與$2\pi $之間,直線$y=1-x$與函數$y=\tan x$的圖形,共有幾個交點?
            \begin{QOPS}
                \QOP $0$	
                \QOP $1$
                \QOP $2$	
                \QOP $3$
                \QOP $4$
            \end{QOPS}            
        \end{QBODY}
        \begin{QFROMS}
        \end{QFROMS}
        \begin{QTAGS}\end{QTAGS}
        \begin{QANS}
            (4)
        \end{QANS}
        \begin{QSOLLIST}
        \end{QSOLLIST}
        \begin{QEMPTYSPACE}
        \end{QEMPTYSPACE}
    \end{QUESTION}
    \begin{QUESTION}
        \begin{ExamInfo}{87}{學測}{單選}{2}
        \end{ExamInfo}
        \begin{ExamAnsRateInfo}{}{}{}{}
        \end{ExamAnsRateInfo}
        \begin{QBODY}
            設$1-i$為${{x}^{\text{2}}}+ax+3-i=\text{ }0$的一根,則$a$的值為何?
            \begin{QOPS}
                \QOP $-3$	
                \QOP $-2$
                \QOP $-1-i$	
                \QOP $2$
                \QOP $3$
            \end{QOPS}            
        \end{QBODY}
        \begin{QFROMS}
        \end{QFROMS}
        \begin{QTAGS}\end{QTAGS}
        \begin{QANS}
            (1)
        \end{QANS}
        \begin{QSOLLIST}
        \end{QSOLLIST}
        \begin{QEMPTYSPACE}
        \end{QEMPTYSPACE}
    \end{QUESTION}
    \begin{QUESTION}
        \begin{ExamInfo}{87}{學測}{單選}{3}
        \end{ExamInfo}
        \begin{ExamAnsRateInfo}{}{}{}{}
        \end{ExamAnsRateInfo}
        \begin{QBODY}
            設事件A發生的機率為$\frac{1}{2}$,事件B發生的機率為$\frac{1}{3}$。若以$P$表事件A或事件B發生的機率,則
            \begin{QOPS}
                \QOP $p\le \frac{1}{6}$	
                \QOP $\frac{1}{6}<p\le \frac{1}{3}$
                \QOP $\frac{1}{3}<p<\frac{1}{2}$	
                \QOP $\frac{1}{2}\le p\le \frac{5}{6}$
                \QOP $p>\frac{5}{6}$
            \end{QOPS}
            
        \end{QBODY}
        \begin{QFROMS}
        \end{QFROMS}
        \begin{QTAGS}\end{QTAGS}
        \begin{QANS}
            (4)
        \end{QANS}
        \begin{QSOLLIST}
        \end{QSOLLIST}
        \begin{QEMPTYSPACE}
        \end{QEMPTYSPACE}
    \end{QUESTION}
    \begin{QUESTION}
        \begin{ExamInfo}{87}{學測}{單選}{4}
        \end{ExamInfo}
        \begin{ExamAnsRateInfo}{}{}{}{}
        \end{ExamAnsRateInfo}
        \begin{QBODY}
            如圖,$ABCDEF$為一正六邊形。那麼下列向量內積中,何者最大?
            %TODO: 補圖
            \begin{QOPS}
                \QOP $\lvec{AB}\cdot\lvec{AB}$
                \QOP $\lvec{AB}\cdot\lvec{AC}$
                \QOP $\lvec{AB}\cdot\lvec{AD}$
                \QOP $\lvec{AB}\cdot\lvec{AE}$
                \QOP $\lvec{AB}\cdot\lvec{AF}$
            \end{QOPS}
        \end{QBODY}
        \begin{QFROMS}
        \end{QFROMS}
        \begin{QTAGS}\end{QTAGS}
        \begin{QANS}
            (2)
        \end{QANS}
        \begin{QSOLLIST}
        \end{QSOLLIST}
        \begin{QEMPTYSPACE}
        \end{QEMPTYSPACE}
    \end{QUESTION}
\end{QUESTIONS}\begin{QUESTIONS}
    \begin{QUESTION}
        \begin{ExamInfo}{87}{學測}{多選}{5}
        \end{ExamInfo}
        \begin{ExamAnsRateInfo}{}{}{}{}
        \end{ExamAnsRateInfo}
        \begin{QBODY}
            已知「偶數的平方是4的倍數;奇數的平方除以4餘數為1」。考慮五個數:513 , 226 , 216 , 154 , 145,試問下列何者可以和上述五數中的某一數相加成為完全平方數?
            \begin{QOPS}
                \QOP 513	
                \QOP 226
                \QOP 216	
                \QOP 154
                \QOP 145
            \end{QOPS}
            
        \end{QBODY}
        \begin{QFROMS}
        \end{QFROMS}
        \begin{QTAGS}\end{QTAGS}
        \begin{QANS}
            (1)(3)(5)
        \end{QANS}
        \begin{QSOLLIST}
        \end{QSOLLIST}
        \begin{QEMPTYSPACE}
        \end{QEMPTYSPACE}
    \end{QUESTION}
    \begin{QUESTION}
        \begin{ExamInfo}{87}{學測}{多選}{6}
        \end{ExamInfo}
        \begin{ExamAnsRateInfo}{}{}{}{}
        \end{ExamAnsRateInfo}
        \begin{QBODY}
            設4不共點的三直線之方程式分別為\\
            $\begin{aligned}
              & \text{ax - }4\text{y }=\text{ }1, \\ 
             & (\text{a }+1)\text{x }+\text{ }3\text{y }=\text{ }2, \\ 
             & \text{x - }2\text{y }=\text{ }3, \\ 
            \end{aligned}$\\
            其中$a$為實數。試問$a$為何值時,上述三直線會圍出一個直角三角形?
            \begin{QOPS}
                \QOP $–8 $
                \QOP $–4$
                \QOP $1	$
                \QOP $3$
                \QOP $5$
            \end{QOPS}            
        \end{QBODY}
        \begin{QFROMS}
        \end{QFROMS}
        \begin{QTAGS}\end{QTAGS}
        \begin{QANS}
            (1)(2)(4)(5)
        \end{QANS}
        \begin{QSOLLIST}
        \end{QSOLLIST}
        \begin{QEMPTYSPACE}
        \end{QEMPTYSPACE}
    \end{QUESTION}
    \begin{QUESTION}
        \begin{ExamInfo}{87}{學測}{多選}{7}
        \end{ExamInfo}
        \begin{ExamAnsRateInfo}{}{}{}{}
        \end{ExamAnsRateInfo}
        \begin{QBODY}
            下列敘述何者為真?
            \begin{QOPS}
                \QOP $\sin {{50}^{\circ }}<\cos {{50}^{\circ }}$	
                \QOP $\tan {{50}^{\circ }}<\cot {{50}^{\circ }}$
                \QOP $\tan {{50}^{\circ }}<\sec {{50}^{\circ }}$	
                \QOP $\sin {{230}^{\circ }}<\cos {{230}^{\circ }}$	
                \QOP $\tan {{230}^{\circ }}<\cot {{230}^{\circ }}$
            \end{QOPS}
        \end{QBODY}
        \begin{QFROMS}
        \end{QFROMS}
        \begin{QTAGS}\end{QTAGS}
        \begin{QANS}
            (3)(4)
        \end{QANS}
        \begin{QSOLLIST}
        \end{QSOLLIST}
        \begin{QEMPTYSPACE}
        \end{QEMPTYSPACE}
    \end{QUESTION}
    \begin{QUESTION}
        \begin{ExamInfo}{87}{學測}{多選}{8}
        \end{ExamInfo}
        \begin{ExamAnsRateInfo}{}{}{}{}
        \end{ExamAnsRateInfo}
        \begin{QBODY}
            在空間中,下列那些點可與$A\left( \text{1,2,3} \right)$,$B\left( \text{2,5,3} \right)$,$\text{C}\left( \text{2,6,4} \right)$三點構成一平行四邊形?
            \begin{QOPS}
                \QOP $\left(  -1,-5,-2 \right)$	
                \QOP $\left( 1,1,2 \right)$
                \QOP $\left( 1,3,4 \right)$	
                \QOP $\left( 3,7,6 \right)$
                \QOP $\left( 3,9,4 \right)$
            \end{QOPS}
        \end{QBODY}
        \begin{QFROMS}
        \end{QFROMS}
        \begin{QTAGS}\end{QTAGS}
        \begin{QANS}
            (2)(3)(5)
        \end{QANS}
        \begin{QSOLLIST}
        \end{QSOLLIST}
        \begin{QEMPTYSPACE}
        \end{QEMPTYSPACE}
    \end{QUESTION}
    \begin{QUESTION}
        \begin{ExamInfo}{87}{學測}{多選}{9}
        \end{ExamInfo}
        \begin{ExamAnsRateInfo}{}{}{}{}
        \end{ExamAnsRateInfo}
        \begin{QBODY}
            設$a$與$b$均為實數,且二次函數 $f\left( x \right)=a{{\left( x-1 \right)}^{2}}+b$滿足 f(4)> 0, f(5)< 0 。試問下列何者為真?
            \begin{QOPS}
                \QOP $ f(0) >0$	
                \QOP $ f(-1)>0$
                \QOP $ f(-2)>0$	
                \QOP $ f(-3)>0$
                \QOP $ f(-4)>0$
            \end{QOPS}            
        \end{QBODY}
        \begin{QFROMS}
        \end{QFROMS}
        \begin{QTAGS}\end{QTAGS}
        \begin{QANS}
            (1)(2)(3)
        \end{QANS}
        \begin{QSOLLIST}
        \end{QSOLLIST}
        \begin{QEMPTYSPACE}
        \end{QEMPTYSPACE}
    \end{QUESTION}
    \begin{QUESTION}
        \begin{ExamInfo}{87}{學測}{多選}{10}
        \end{ExamInfo}
        \begin{ExamAnsRateInfo}{}{}{}{}
        \end{ExamAnsRateInfo}
        \begin{QBODY}
            %TODO: 補圖
            圖為某池塘中布袋蓮蔓延的面積與時間的關係圖。假設其關係為指數函數,試問下列敘述何者為真?
            \begin{QOPS}
                \QOP 此指數函數的底數為$2$。
                \QOP 在第5個月時,布袋蓮的面積就會超過$30m^2$。
                \QOP 布袋蓮從$4m^2$ 蔓延到$12m^2$,只需$1.5$個月。
                \QOP 設布袋蓮蔓延到$2m^2$、$3m^2$、$6m^2$所需的時間分別為$t_1$、$t_2$、$t_3$,則 $t_1+t_2=t_3$。
                \QOP 布袋蓮在第1到第3個月之間的蔓延平均速度等於在第2到第4個月之間的蔓延平均速度。
            \end{QOPS}
        \end{QBODY}
        \begin{QFROMS}
        \end{QFROMS}
        \begin{QTAGS}\end{QTAGS}
        \begin{QANS}
            (1)(2)(4)
        \end{QANS}
        \begin{QSOLLIST}
        \end{QSOLLIST}
        \begin{QEMPTYSPACE}
        \end{QEMPTYSPACE}
    \end{QUESTION}
\end{QUESTIONS}
\begin{QUESTIONS}
    \begin{QUESTION}
        \begin{ExamInfo}{87}{學測}{選填}{11}
        \end{ExamInfo}
        \begin{ExamAnsRateInfo}{}{}{}{}
        \end{ExamAnsRateInfo}
        \begin{QBODY}
            在三位數中,百位數與個位數之差的絕對值為$2$的數,共有$\TCNBOX{\TCN\TCN\TCN}$
            個。
    
        \end{QBODY}
        \begin{QFROMS}
        \end{QFROMS}
        \begin{QTAGS}\end{QTAGS}
        \begin{QANS}
            $150$
        \end{QANS}
        \begin{QSOLLIST}
        \end{QSOLLIST}
        \begin{QEMPTYSPACE}
        \end{QEMPTYSPACE}
    \end{QUESTION}
    \begin{QUESTION}
        \begin{ExamInfo}{87}{學測}{選填}{12}
        \end{ExamInfo}
        \begin{ExamAnsRateInfo}{}{}{}{}
        \end{ExamAnsRateInfo}
        \begin{QBODY}
            設$a$與 b均為實數。若
            $\frac{\text{a}}{{{\text{2}}^{\text{1}}}}+\frac{\text{b}}{{{\text{2}}^{\text{2}}}}+\frac{\text{a}}{{{\text{2}}^{\text{3}}}}+\frac{\text{b}}{{{\text{2}}^{\text{4}}}}+\ \ \cdots +\frac{\text{a}}{{{\text{2}}^{\text{2}\ \text{n}-\text{1}}}}+\frac{\text{b}}{{{\text{2}}^{\text{2}\ \text{n}}}}+\ \cdots =\text{3,}$
            則$2a+b$=    $\TCNBOX{\TCN}$     。
            
        \end{QBODY}
        \begin{QFROMS}
        \end{QFROMS}
        \begin{QTAGS}\end{QTAGS}
        \begin{QANS}
            $9$
        \end{QANS}
        \begin{QSOLLIST}
        \end{QSOLLIST}
        \begin{QEMPTYSPACE}
        \end{QEMPTYSPACE}
    \end{QUESTION}
    \begin{QUESTION}
        \begin{ExamInfo}{87}{學測}{選填}{13}
        \end{ExamInfo}
        \begin{ExamAnsRateInfo}{}{}{}{}
        \end{ExamAnsRateInfo}
        \begin{QBODY}
            某公司有甲、乙、丙三條生產線,現欲生產三萬個產品,如果甲、乙、丙三條生產線同時開動,則需10小時;如果只開動乙、丙兩條生產線,則需15小時;如果只開動甲生產線15小時,則需再開動丙生產線30小時,才能完成所有產品。問如果只開動乙生產線,則需
            $\TCNBOX{\TCN\TCN}$小時才能生產三萬個產品。
        \end{QBODY}
        \begin{QFROMS}
        \end{QFROMS}
        \begin{QTAGS}\end{QTAGS}
        \begin{QANS}
            $20$
        \end{QANS}
        \begin{QSOLLIST}
        \end{QSOLLIST}
        \begin{QEMPTYSPACE}
        \end{QEMPTYSPACE}
    \end{QUESTION}
    \begin{QUESTION}
        \begin{ExamInfo}{87}{學測}{選填}{14}
        \end{ExamInfo}
        \begin{ExamAnsRateInfo}{}{}{}{}
        \end{ExamAnsRateInfo}
        \begin{QBODY}
            長方體中,互為歪斜線的稜線共有$\TCNBOX{\TCN\TCN}$對。

        \end{QBODY}
        \begin{QFROMS}
        \end{QFROMS}
        \begin{QTAGS}\end{QTAGS}
        \begin{QANS}
            $24$
        \end{QANS}
        \begin{QSOLLIST}
        \end{QSOLLIST}
        \begin{QEMPTYSPACE}
        \end{QEMPTYSPACE}
    \end{QUESTION}
    \begin{QUESTION}
        \begin{ExamInfo}{87}{學測}{選填}{15}
        \end{ExamInfo}
        \begin{ExamAnsRateInfo}{}{}{}{}
        \end{ExamAnsRateInfo}
        \begin{QBODY}
            %TODO:補圖
            在圖中,ABC是邊長為8的正三角形撞球檯,線段BP =$\sqrt{2}$。今由P點將一粒球以平行BA方向射出,最後又回到P點。球所走的路徑,如圖箭號所示。則此路徑的長度
            為$\TCNBOX{\TCN\TCN}$             。
            
        \end{QBODY}
        \begin{QFROMS}
        \end{QFROMS}
        \begin{QTAGS}\end{QTAGS}
        \begin{QANS}
            $24$
        \end{QANS}
        \begin{QSOLLIST}
        \end{QSOLLIST}
        \begin{QEMPTYSPACE}
        \end{QEMPTYSPACE}
    \end{QUESTION}
    \begin{QUESTION}
        \begin{ExamInfo}{87}{學測}{選填}{16}
        \end{ExamInfo}
        \begin{ExamAnsRateInfo}{}{}{}{}
        \end{ExamAnsRateInfo}
        \begin{QBODY}
            在等比數列$\left\langle {{a}_{n}} \right\rangle $中,
            $\begin{aligned}
              & {{a}_{1}}=1, \\ 
             & {{a}_{4}}=2-\sqrt{5}\ , \\ 
             & {{a}_{n+2}}={{a}_{n+1}}+{{a}_{n}}\ \,,\quad n\ge 1 \\ 
            \end{aligned}$
            則$\left\langle {{a}_{n}} \right\rangle $的公比=    $\TCNBOX{\TCN\TCN\sqrt{5}}$
        \end{QBODY}
        \begin{QFROMS}
        \end{QFROMS}
        \begin{QTAGS}\end{QTAGS}
        \begin{QANS}
            $\FR{1-\sqrt{5}}{2}$
        \end{QANS}
        \begin{QSOLLIST}
        \end{QSOLLIST}
        \begin{QEMPTYSPACE}
        \end{QEMPTYSPACE}
    \end{QUESTION}
    \begin{QUESTION}
        \begin{ExamInfo}{87}{學測}{選填}{17}
        \end{ExamInfo}
        \begin{ExamAnsRateInfo}{}{}{}{}
        \end{ExamAnsRateInfo}
        \begin{QBODY}
            如圖,A、B分別位於一河口的兩岸邊。某人在通往A點的筆直公路上,距離A點50公尺的C點與距離A點200公尺的D點,分別測得$\angle ACB=60^\circ$,$\angle ADB=30^\circ$,則A與B的
            距離為 $\TCNBOX{\TCN\TCN\sqrt{\TCN}}$ 公尺。
            %TODO: 補圖            
        \end{QBODY}
        \begin{QFROMS}
        \end{QFROMS}
        \begin{QTAGS}\end{QTAGS}
        \begin{QANS}
            $50\sqrt{7}$
        \end{QANS}
        \begin{QSOLLIST}
        \end{QSOLLIST}
        \begin{QEMPTYSPACE}
        \end{QEMPTYSPACE}
    \end{QUESTION}
    \begin{QUESTION}
        \begin{ExamInfo}{87}{學測}{選填}{18}
        \end{ExamInfo}
        \begin{ExamAnsRateInfo}{}{}{}{}
        \end{ExamAnsRateInfo}
        \begin{QBODY}
            設$f\left( x \right)$為一多項式。若$\left( x+1 \right)f\left( x \right)$除以${{x}^{2}}+x+1$的餘式為$5x+3$,則$f\left( x \right)$除以${{x}^{2}}+x+1$的餘式為$\TCNBOX{\TCN x + \TCN}$。
        \end{QBODY}
        \begin{QFROMS}
        \end{QFROMS}
        \begin{QTAGS}\end{QTAGS}
        \begin{QANS}
            $2x+5$
        \end{QANS}
        \begin{QSOLLIST}
        \end{QSOLLIST}
        \begin{QEMPTYSPACE}
        \end{QEMPTYSPACE}
    \end{QUESTION}
    \begin{QUESTION}
        \begin{ExamInfo}{87}{學測}{選填}{19}
        \end{ExamInfo}
        \begin{ExamAnsRateInfo}{}{}{}{}
        \end{ExamAnsRateInfo}
        \begin{QBODY}
            %TODO:補圖
            在圖(五)中,圓$O$的半徑為6,$F$的坐標為(4,0),$Q$在圓$O$上,$P$ 為 $\overline{FQ}$ 的中垂線與$\overline{OQ}$的交點。當$Q$在圓$O$上移動時,動點$P$的軌跡方程式為
            $\TCNBOX{\FR{(x-\TCN)^2}{\TCN} +\FR{(y-\TCN)^2}{\TCN} = 1 }$
        \end{QBODY}
        \begin{QFROMS}
        \end{QFROMS}
        \begin{QTAGS}\end{QTAGS}
        \begin{QANS}
            $\FR{(x-2)^2}{9} + \FR{(y-0)^2}{5} =1 $
        \end{QANS}
        \begin{QSOLLIST}
        \end{QSOLLIST}
        \begin{QEMPTYSPACE}
        \end{QEMPTYSPACE}
    \end{QUESTION}
    \begin{QUESTION}
        \begin{ExamInfo}{87}{學測}{選填}{20}
        \end{ExamInfo}
        \begin{ExamAnsRateInfo}{}{}{}{}
        \end{ExamAnsRateInfo}
        \begin{QBODY}
            下表所列為各項主要食品的平均消費價格,以及民國70年維持一家四口所需各項食品的平均需要量。若以拉氏指數來衡量,那麼民國76年
            主要食品的費用比民國70年高出的百分率為 $\TCNBOX{\TCN\TCN\TCN}$\%。(小數點以下四捨五入)\\
        \begin{tabular}{|c|c|c|c|}
        	\hline
        	項 目  & 70年價格 & 76年價格 & 70年平均用量 \\ \hline
        	蓬萊米  &  7.6  &  16   &   45    \\
        	豬 肉  &  49   &  97   &    5    \\ 
        	虱目魚  &  36   &  74   &   0.5   \\ 
        	包心白菜 &  5.6  &  15   &    4    \\
        	香 蕉  &  4.7  &  13   &    3    \\ 
        	花生油  &  25   &  54   &   0.8   \\ \hline
        \end{tabular} 
        \end{QBODY}
        \begin{QFROMS}
        \end{QFROMS}
        \begin{QTAGS}\end{QTAGS}
        \begin{QANS}
            $109$
        \end{QANS}
        \begin{QSOLLIST}
        \end{QSOLLIST}
        \begin{QEMPTYSPACE}
        \end{QEMPTYSPACE}
    \end{QUESTION}
\end{QUESTIONS}