% !TEX encoding = UTF-8 Unicode
% !TEX TS-program = xelatex
\begin{QUESTIONS}
    \begin{QUESTION}
        \begin{ExamInfo}{96}{學測}{單選}{1}
        \end{ExamInfo}
        \begin{ExamAnsRateInfo}{71}{95}{81}{37}
        \end{ExamAnsRateInfo}
        \begin{QBODY}
            設 $f(x)=ax^6 -bx^4 +3x- 2$,其中 $a,b$ 為非零實數,則 $f(5)- f(-5)$ 之值為 
		\begin{QOPS} 
			\QOP $-30$        
			\QOP $0$        
			\QOP $2\sqrt{2}$ 
			\QOP $30$ 
			\QOP 無法確定(與 $a,b$有關)。
		\end{QOPS}
        \end{QBODY}
        \begin{QFROMS}
        \end{QFROMS}
        \begin{QTAGS}\QTAG{奇偶函數}\QTAG{B1C2多項式函數}\QTAG{B1C2-2多項式的運算與應用}\end{QTAGS}
        \begin{QANS}
            (4)
        \end{QANS}
        \begin{QSOLLIST}
        \end{QSOLLIST}
        \begin{QEMPTYSPACE}
        \end{QEMPTYSPACE}
    \end{QUESTION}
    \begin{QUESTION}
        \begin{ExamInfo}{96}{學測}{單選}{2}
        \end{ExamInfo}
        \begin{ExamAnsRateInfo}{35}{58}{28}{19}
        \end{ExamAnsRateInfo}
        \begin{QBODY}
            試問共有多少個正整數 $n$ 使得坐標平面上通過點 $A(-n, 0)$ 與點 $B(0, 2)$ 的直線亦通過點 $P(7, k)$ , 其中 $k$ 為某一正整數?
			\begin{QOPS} 
				\QOP 2 個 
				\QOP 4 個 
				\QOP 6 個 
				\QOP 8 個 
				\QOP 無窮多個
			\end{QOPS}
        \end{QBODY}
        \begin{QFROMS}
        \end{QFROMS}
        \begin{QTAGS}\QTAG{B3C2-1直線方程式及其圖形}\QTAG{B3C2直線與圓}\end{QTAGS}
        \begin{QANS}
            (2)
        \end{QANS}
        \begin{QSOLLIST}
        \end{QSOLLIST}
        \begin{QEMPTYSPACE}
        \end{QEMPTYSPACE}
    \end{QUESTION}
    \begin{QUESTION}
        \begin{ExamInfo}{96}{學測}{單選}{3}
        \end{ExamInfo}
        \begin{ExamAnsRateInfo}{58}{88}{62}{24}
        \end{ExamAnsRateInfo}
        \begin{QBODY}
            設某沙漠地區某一段時間的溫度函數為 $f(t) = -t^2 +10t +11$,其中 $1 \leq  t \leq 10$ ,則這段時間內該地區的最大溫差為
			\begin{QOPS} 
				\QOP 9 
				\QOP 16 
				\QOP 20 
				\QOP 25	
				\QOP 36
			\end{QOPS}
        \end{QBODY}
        \begin{QFROMS}
        \end{QFROMS}
        \begin{QTAGS}\QTAG{B1C2多項式函數}\QTAG{B1C2-1簡單的多項式及圖形}\QTAG{最值}\end{QTAGS}
        \begin{QANS}
            (4)
        \end{QANS}
        \begin{QSOLLIST}
        \end{QSOLLIST}
        \begin{QEMPTYSPACE}
        \end{QEMPTYSPACE}
    \end{QUESTION}
    \begin{QUESTION}
        \begin{ExamInfo}{96}{學測}{單選}{4}
        \end{ExamInfo}
        \begin{ExamAnsRateInfo}{47}{78}{46}{17}
        \end{ExamAnsRateInfo}
        \begin{QBODY}
            坐標平面上方程式 $\frac{x^2}{9} + \frac{y^2}{4} =1$ 的圖形與 $\frac{(x+1)^2}{16} - \frac{y^2}{9}=1$的圖形共有幾個交點? 
			\begin{QOPS} 
				\QOP 1 個 
				\QOP 2 個 
				\QOP 3 個

				\QOP 4 個 
				\QOP 0 個
			\end{QOPS}
        \end{QBODY}
        \begin{QFROMS}
        \end{QFROMS}
        \begin{QTAGS}\QTAG{B4C4-3雙曲線}\QTAG{B4C4二次曲線}\QTAG{B4C4-2橢圓}\QTAG{圖形}\end{QTAGS}
        \begin{QANS}
            (1)
        \end{QANS}
        \begin{QSOLLIST}
        \end{QSOLLIST}
        \begin{QEMPTYSPACE}
        \end{QEMPTYSPACE}
    \end{QUESTION}
    \begin{QUESTION}
        \begin{ExamInfo}{96}{學測}{單選}{5}
        \end{ExamInfo}
        \begin{ExamAnsRateInfo}{47}{67}{46}{28}
        \end{ExamAnsRateInfo}
        \begin{QBODY}
            關於坐標平面上函數 $y = \sin x$ 的圖形和 $y = \frac{x}{10\pi}$ 的圖形之交點個數,下列哪一個選項是正確的? 
			\begin{QOPS}
				\QOP 交點的個數是無窮多 
				\QOP 交點的個數是奇數且大於 20 
				\QOP 交點的個數是奇數且小於 20 
				\QOP 交點的個數是偶數且大於或等於 20 
				\QOP 交點的個數是偶數且小於 20
			\end{QOPS}
        \end{QBODY}
        \begin{QFROMS}
        \end{QFROMS}
        \begin{QTAGS}\QTAG{B5C2三角函數II}\end{QTAGS}
        \begin{QANS}
            (3)
        \end{QANS}
        \begin{QSOLLIST}
        \end{QSOLLIST}
        \begin{QEMPTYSPACE}
        \end{QEMPTYSPACE}
    \end{QUESTION}
\end{QUESTIONS}
\begin{QUESTIONS}
    \begin{QUESTION}
        \begin{ExamInfo}{96}{學測}{多選}{6}
        \end{ExamInfo}
        \begin{ExamAnsRateInfo}{19}{35}{12}{10}
        \end{ExamAnsRateInfo}
        \begin{QBODY}
            若 $\Gamma = \{ z|z$ 為複數且 $|z - 1|=1\}$, 則下列哪些點會落在圖形 $\Omega =\{w|w=iz,z \in \Gamma \}$ 上?  
			\begin{QOPS} 
				\QOP $2i$    \QOP $-2i$ 
				\QOP $1+i$ 
				\QOP $1-i$ 
				\QOP $-1+i$
			\end{QOPS}
        \end{QBODY}
        \begin{QFROMS}
        \end{QFROMS}
        \begin{QTAGS}\QTAG{B5C2三角函數II}\end{QTAGS}
        \begin{QANS}
            (1)(3)(5)
        \end{QANS}
        \begin{QSOLLIST}
        \end{QSOLLIST}
        \begin{QEMPTYSPACE}
        \end{QEMPTYSPACE}
    \end{QUESTION}
    \begin{QUESTION}
        \begin{ExamInfo}{96}{學測}{多選}{7}
        \end{ExamInfo}
        \begin{ExamAnsRateInfo}{25}{42}{20}{13}
        \end{ExamAnsRateInfo}
        \begin{QBODY}
            坐標平面上有相異兩點 $P$, $Q$,其中 $P$ 點坐標為 $(s,t)$。
			已知線段 $PQ$ 的中垂線 $L$ 的方程式為 $3x - 4 y = 0$,試問下列哪些選項是正確的? 
    		\begin{QOPS} 
    			\QOP 向量$\lvec{PQ}$ 與向量 $(3,-4)$  平行 
    			\QOP 線段 $\overline{PQ}$ 的長度等於 $\frac{|6s-8t|}{5}$ 
    			\QOP $Q$ 點坐標為 $(t,s)$ 
    			\QOP 過 $Q$ 點與直線 $L$ 平行之直線必過點 $(-s,-t)$ 
    			\QOP 以 $O$ 表示原點,則向量 $\lvec{OP}+\lvec{OQ}$ 與向量 $\lvec{PQ}$ 的內積必為 0 
    		\end{QOPS}
        \end{QBODY}
        \begin{QFROMS}
        \end{QFROMS}
        \begin{QTAGS}\QTAG{B3C3-2平面向量的內積}\QTAG{B3C3-3面積與二階行列式}\QTAG{B3C3平面向量}\QTAG{距離}\end{QTAGS}
        \begin{QANS}
            (1)(2)(4)(5)
        \end{QANS}
        \begin{QSOLLIST}
        \end{QSOLLIST}
        \begin{QEMPTYSPACE}
        \end{QEMPTYSPACE}
    \end{QUESTION}
    \begin{QUESTION}
        \begin{ExamInfo}{96}{學測}{多選}{8}
        \end{ExamInfo}
        \begin{ExamAnsRateInfo}{30}{36}{32}{22}
        \end{ExamAnsRateInfo}
        \begin{QBODY}
            下列哪些選項中的矩陣經過一系列的列運算後可以化成
			$\left( \begin{array}{cccc} 1 & 2 & 3 & 7 \\0 & 1 & 1 & 2 \\0 & 0 & 1 & 1 \\\end{array}\right)$

			\begin{QOPS} 
				\QOP $\left( \begin{array}{cccc} 1 & 2 & 3 & 7 \\0 & 1 & 1 & 2 \\0 & 2 & 3 & 5 \\\end{array}\right)$

				\QOP $\left( \begin{array}{cccc} -1 & 3 & -1 & 0 \\-1 & 1 & 1 & 0 \\3 & 1 & -7 & 0 \\\end{array}\right)$

				\QOP $\left( \begin{array}{cccc} 1 & 1 & 2 & 5 \\1 & -1 & 1 & 2 \\1 & 1 & 2 & 5 \\\end{array}\right)$

				\QOP $\left( \begin{array}{cccc} 2 & 1 & 3 & 6 \\-1 & 1 & 1 & 0 \\0 & 1 & 1 & 2 \\\end{array}\right)$

				\QOP $\left( \begin{array}{cccc} 1 & 3 & 2 & 7 \\0 & 1 & 1 & 2 \\0 & 1 & 0 & 1 \\\end{array}\right)$
			\end{QOPS}
        \end{QBODY}
        \begin{QFROMS}
        \end{QFROMS}
        \begin{QTAGS}\QTAG{B4C3矩陣}\QTAG{B4C3-1線性方程組與矩陣}\QTAG{方程組}\end{QTAGS}
        \begin{QANS}
            (1)(5)
        \end{QANS}
        \begin{QSOLLIST}
        \end{QSOLLIST}
        \begin{QEMPTYSPACE}
        \end{QEMPTYSPACE}
    \end{QUESTION}
    \begin{QUESTION}
        \begin{ExamInfo}{96}{學測}{多選}{9}
        \end{ExamInfo}
        \begin{ExamAnsRateInfo}{42}{70}{39}{17}
        \end{ExamAnsRateInfo}
        \begin{QBODY}
            坐標空間中,在 $xy$ 平面上置有三個半徑為 1 的球兩兩相切,設其球心分別為 $A,B,C$。
			今將第四個半徑為 1 的球置於這三個球的上方,且與這三個球都相切,並保持穩定。
			設第四個球的球心為 $P$,試問下列哪些選項是正確的?
			\begin{QOPS}
			 \QOP 點 $A,B,C$ 所在的平面和 $xy$  平面平行
			 \QOP 三角形 $ABC$ 是一個正三角形 
			 \QOP 三角形 $PAB$ 有一邊長為 $\sqrt{2}$ 
			 \QOP 點$P$到直線 $AB$的距離為 $\sqrt{3}$ 
			 \QOP 點 $P$ 到 $xy$ 平面的距離為$1+ \sqrt{3}$
			\end{QOPS}
        \end{QBODY}
        \begin{QFROMS}
        \end{QFROMS}
        \begin{QTAGS}\QTAG{B4C1-2空間向量的坐標表示法}\QTAG{B4C1空間向量}\end{QTAGS}
        \begin{QANS}
            (1)(2)(4)
        \end{QANS}
        \begin{QSOLLIST}
        \end{QSOLLIST}
        \begin{QEMPTYSPACE}
        \end{QEMPTYSPACE}
    \end{QUESTION}
    \begin{QUESTION}
        \begin{ExamInfo}{96}{學測}{多選}{10}
        \end{ExamInfo}
        \begin{ExamAnsRateInfo}{28}{52}{21}{11}
        \end{ExamAnsRateInfo}
        \begin{QBODY}
            設 $a$ 為大於 1 的實數,考慮函數 $f (x) = a^x$ 與 $g(x) = \log_a x$ ,試問下列哪些選項是正確的? 
			\begin{QOPS} 
				\QOP 若 $f(3)=6$,則 $g(36)=6$ 
				\QOP $\frac{f (238)}{f(219)} = \frac{f (38)}{ f (19)}$ 
				\QOP $g(238)-g(219)= g(38)-g(19)$ 
				\QOP  若 $P, Q$ 為 $y = g(x)$ 的圖形上兩相異點,則直線 $PQ$ 之斜率必為正數 
				\QOP 若直線 $y=5x$ 與 $y= f(x)$ 的圖形有兩個交點,則直線 $y = \frac{1}{5}x$ 與 $y=g(x)$ 的圖形也有兩個交點。
			\end{QOPS}
        \end{QBODY}
        \begin{QFROMS}
        \end{QFROMS}
        \begin{QTAGS}\QTAG{圖形}\QTAG{B1C3-4對數函數}\QTAG{指數律}\QTAG{跨章節試題}\QTAG{B1C3-2指數函數}\QTAG{B1C3指對數函數}\QTAG{對數律}\end{QTAGS}
        \begin{QANS}
            (1)(2)(4)(5)
        \end{QANS}
        \begin{QSOLLIST}
        \end{QSOLLIST}
        \begin{QEMPTYSPACE}
        \end{QEMPTYSPACE}
    \end{QUESTION}
    \begin{QUESTION}
        \begin{ExamInfo}{96}{學測}{多選}{11}
        \end{ExamInfo}
        \begin{ExamAnsRateInfo}{22}{35}{16}{15}
        \end{ExamAnsRateInfo}
        \begin{QBODY}
            設 $f(x)$ 為一實係數三次多項式且其最高次項係數為1, 已知 $f(1)=1$, $f(2)=2$, $f(5)=5$,則 $f(x) = 0$ 在下列哪些區間必定有實根? 
			\begin{QOPS} 
				\QOP $(-\infty , 0)$ 
				\QOP  $(0, 1)$ 
				\QOP (1, 2) 
				\QOP (2, 5) 
				\QOP $(5, \infty)$ 
			\end{QOPS}
        \end{QBODY}
        \begin{QFROMS}
        \end{QFROMS}
        \begin{QTAGS}\QTAG{插值法}\QTAG{B1C2-3多項式方程式}\QTAG{勘根定理}\QTAG{B1C2多項式函數}\QTAG{因式定理}\end{QTAGS}
        \begin{QANS}
            (2)(4)
        \end{QANS}
        \begin{QSOLLIST}
        \end{QSOLLIST}
        \begin{QEMPTYSPACE}
        \end{QEMPTYSPACE}
    \end{QUESTION}
\end{QUESTIONS}
\begin{QUESTIONS}
    \begin{QUESTION}
        \begin{ExamInfo}{96}{學測}{填充}{A}
        \end{ExamInfo}
        \begin{ExamAnsRateInfo}{41}{77}{36}{10}
        \end{ExamAnsRateInfo}
        \begin{QBODY}
            設實數 $x$ 滿足 $0<x<1$,且 $\log_x 4 - \log_2 x=1$,則 $x=\TCNBOX{\FR{\TCN}{\TCN}}$。
        \end{QBODY}
        \begin{QFROMS}
        \end{QFROMS}
        \begin{QTAGS}\QTAG{B1C3-4對數函數}\QTAG{方程式}\QTAG{B1C3指對數函數}\end{QTAGS}
        \begin{QANS}
            $\dfrac{1}{4}$
        \end{QANS}
        \begin{QSOLLIST}
        \end{QSOLLIST}
        \begin{QEMPTYSPACE}
        \end{QEMPTYSPACE}
    \end{QUESTION}
    \begin{QUESTION}
        \begin{ExamInfo}{96}{學測}{填充}{B}
        \end{ExamInfo}
        \begin{ExamAnsRateInfo}{27}{64}{16}{1}
        \end{ExamAnsRateInfo}
        \begin{QBODY}
            在坐標平面上的 $\triangle ABC$ 中,$P$ 為 $\overline{BC}$ 邊之中點,$Q$ 在 $\overline{AC}$ 邊上且 $\overline{AQ} = 2\overline{QC}$。 已知 $\lvec{PA} =(4, 3)$, $\lvec{PQ} = ( 1 , 5 )$ , 則 $\lvec{BC} = \TCNBOX{(\TCN\TCN,\TCN\TCN)}$ 。
        \end{QBODY}
        \begin{QFROMS}
        \end{QFROMS}
        \begin{QTAGS}\QTAG{線性組合}\QTAG{B3C3-1平面向量的表示法}\QTAG{B3C3平面向量}\end{QTAGS}
        \begin{QANS}
            $(-1,12)$
        \end{QANS}
        \begin{QSOLLIST}
        \end{QSOLLIST}
        \begin{QEMPTYSPACE}
        \end{QEMPTYSPACE}
    \end{QUESTION}
    \begin{QUESTION}
        \begin{ExamInfo}{96}{學測}{填充}{C}
        \end{ExamInfo}
        \begin{ExamAnsRateInfo}{66}{93}{76}{29}
        \end{ExamAnsRateInfo}
        \begin{QBODY}
            在某項才藝競賽中,為了避免評審個人主觀影響參賽者成績太大,主辦單位規定:先將 15 位評審給同一位參賽者的成績求得算術平均數,再將與平均數相差超過 15 分的評審成績剔除後重新計算平均值做為此參賽者的比賽成績。現在有一位參賽者所獲 15 位評審的平均成績為 76 分,其中有三位評審給的成績 92、45、55 應剔除,則這個參賽者的比賽成績為$\TCNBOX{\TCN\TCN}$ 分。
        \end{QBODY}
        \begin{QFROMS}
        \end{QFROMS}
        \begin{QTAGS}\QTAG{平均數}\QTAG{B2C4-1一維數據分析}\QTAG{B2C4數據分析}\end{QTAGS}
        \begin{QANS}
            $79$
        \end{QANS}
        \begin{QSOLLIST}
        \end{QSOLLIST}
        \begin{QEMPTYSPACE}
        \end{QEMPTYSPACE}
    \end{QUESTION}
    \begin{QUESTION}
        \begin{ExamInfo}{96}{學測}{填充}{D}
        \end{ExamInfo}
        \begin{ExamAnsRateInfo}{75}{92}{82}{51}
        \end{ExamAnsRateInfo}
        \begin{QBODY}
            某巨蛋球場 E 區共有 25 排座位,此區每一排都比其前一排多 2 個座位。小明坐在正中間那一排 (即第 13 排),發現此排共有 64 個座位,則此球場 E 區共有$\TCNBOX{\TCN\TCN\TCN\TCN}$ 個座位。
        \end{QBODY}
        \begin{QFROMS}
        \end{QFROMS}
        \begin{QTAGS}\QTAG{B2C1數列級數}\QTAG{B2C1-2級數}\QTAG{等差級數}\end{QTAGS}
        \begin{QANS}
            $1600$
        \end{QANS}
        \begin{QSOLLIST}
        \end{QSOLLIST}
        \begin{QEMPTYSPACE}
        \end{QEMPTYSPACE}
    \end{QUESTION}
    \begin{QUESTION}
        \begin{ExamInfo}{96}{學測}{填充}{E}
        \end{ExamInfo}
        \begin{ExamAnsRateInfo}{29}{66}{18}{3}
        \end{ExamAnsRateInfo}
        \begin{QBODY}
            設 $P, A, B$ 為坐標平面上以原點為圓心的單位圓上三點,其中 $P$ 點坐標為 $(1, 0)$,$A$ 點坐標為 $(\frac{-12}{13},\frac{5}{13})$,且 $\angle APB$為直角,則 $B$ 點坐標為 $\TCNBOX{(\FR{\TCN\TCN}{\TCN\TCN},\FR{\TCN\TCN}{\TCN\TCN})}$ 。
        \end{QBODY}
        \begin{QFROMS}
        \end{QFROMS}
        \begin{QTAGS}\QTAG{B3C2直線與圓}\end{QTAGS}
        \begin{QANS}
            $(\FR{12}{13},\FR{-5}{13})$
        \end{QANS}
        \begin{QSOLLIST}
        \end{QSOLLIST}
        \begin{QEMPTYSPACE}
        \end{QEMPTYSPACE}
    \end{QUESTION}
    \begin{QUESTION}
        \begin{ExamInfo}{96}{學測}{填充}{F}
        \end{ExamInfo}
        \begin{ExamAnsRateInfo}{26}{38}{28}{12}
        \end{ExamAnsRateInfo}
        \begin{QBODY}
            某公司生產多種款式的「阿民」公仔,各種款式只是球帽、球衣或球鞋顏色不同。其中球帽共有黑、灰、紅、藍四種顏色,球衣有白、綠、藍三種顏色,而球鞋有黑、白、灰三種顏色。公司決定紅色的球帽不搭配灰色的鞋子,而白色的球衣則必須搭配藍色的帽子,至於其他顏色間的搭配就沒有限制。在這些配色的要求之下,最多可有 $\TCNBOX{\TCN\TCN}$ 種不同款式的「阿民」公仔。
        \end{QBODY}
        \begin{QFROMS}
        \end{QFROMS}
        \begin{QTAGS}\QTAG{乘法原理加法原理}\QTAG{B2C2-1簡單的邏輯與集合}\QTAG{B2C2排列組合}\end{QTAGS}
        \begin{QANS}
            $25$
        \end{QANS}
        \begin{QSOLLIST}
        \end{QSOLLIST}
        \begin{QEMPTYSPACE}
        \end{QEMPTYSPACE}
    \end{QUESTION}
    \begin{QUESTION}
        \begin{ExamInfo}{96}{學測}{填充}{G}
        \end{ExamInfo}
        \begin{ExamAnsRateInfo}{19}{39}{13}{5}
        \end{ExamAnsRateInfo}
        \begin{QBODY}
            摸彩箱裝有若干編號為 $1, 2, \cdots ,10$ 的彩球,其中各種編號的彩球數目可能不同。今從中隨機摸取一球,依據所取球的號數給予若干報酬。現有甲、乙兩案:甲案為當摸得彩球的號數為 $k$ 時, 其所獲報酬同為 $k$;乙案為當摸得彩球的號數為 $k$ 時,其所獲報酬為 $11- k$ ( $k = 1, 2, \dots ,10 )$。已知
依甲案每摸取一球的期望值為 $\frac{67}{14}$,則依乙案每摸取一球的期望值為 
$\TCNBOX{\frac{\TCN\TCN}{\TCN\TCN}}$。
        \end{QBODY}
        \begin{QFROMS}
        \end{QFROMS}
        \begin{QTAGS}\QTAG{B5C1機率與統計}\end{QTAGS}
        \begin{QANS}
            $\frac{87}{14}$
        \end{QANS}
        \begin{QSOLLIST}
        \end{QSOLLIST}
        \begin{QEMPTYSPACE}
        \end{QEMPTYSPACE}
    \end{QUESTION}
    \begin{QUESTION}
        \begin{ExamInfo}{96}{學測}{填充}{H}
        \end{ExamInfo}
        \begin{ExamAnsRateInfo}{32}{63}{26}{7}
        \end{ExamAnsRateInfo}
        \begin{QBODY}
            坐標平面上有一以點 $V(0, 3)$ 為頂點、 $F(0, 6)$ 為焦點的拋物線。設 $P(a, b)$ 為此拋物線上一點, $Q(a,0)$ 為 $P$ 在 $x$ 軸上的投影,滿足 $\angle FPQ=60^\circ$ ,則 $b= \TCNBOX{\TCN\TCN}$。
        \end{QBODY}
        \begin{QFROMS}
        \end{QFROMS}
        \begin{QTAGS}\QTAG{B4C4二次曲線}\QTAG{B4C4-1拋物線}\QTAG{圖形}\end{QTAGS}
        \begin{QANS}
            $12$
        \end{QANS}
        \begin{QSOLLIST}
        \end{QSOLLIST}
        \begin{QEMPTYSPACE}
        \end{QEMPTYSPACE}
    \end{QUESTION}
    \begin{QUESTION}
        \begin{ExamInfo}{96}{學測}{填充}{I}
        \end{ExamInfo}
        \begin{ExamAnsRateInfo}{15}{37}{6}{2}
        \end{ExamAnsRateInfo}
        \begin{QBODY}
            在$\triangle ABC$中,$M$為 $\overline{BC}$ 邊之中點,若 $\overline{AB}=3$, $\overline{AC}=5$,且 $\angle BAC=120^\circ$,則 $\tan \angle BAM$ = 
	$\TCNBOX{\TCN\sqrt{\TCN}}$ 。
        \end{QBODY}
        \begin{QFROMS}
        \end{QFROMS}
        \begin{QTAGS}\QTAG{廣義角}\QTAG{B3C1三角}\QTAG{B3C1-2廣義角與極坐標}\end{QTAGS}
        \begin{QANS}
            $5 \sqrt{3}$
        \end{QANS}
        \begin{QSOLLIST}
        \end{QSOLLIST}
        \begin{QEMPTYSPACE}
        \end{QEMPTYSPACE}
    \end{QUESTION}
\end{QUESTIONS}
