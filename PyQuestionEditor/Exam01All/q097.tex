% !TEX encoding = UTF-8 Unicode
% !TEX TS-program = xelatex 
\begin{QUESTIONS}
    \begin{QUESTION}
        \begin{ExamInfo}{097}{學測}{單選}{1}
        \end{ExamInfo}
        \begin{ExamAnsRateInfo}{60}{92}{62}{26}
        \end{ExamAnsRateInfo}
        \begin{QBODY}
			對任意實數 $x$ 而言,$27^{(x^2 + \frac{2}{3})}$ 的最小值為 
			\begin{QOPS} 
				\QOP $3$        
				\QOP $3\sqrt{3}$
				\QOP $9$ 
				\QOP $27$
				\QOP $81\sqrt{3}$
			\end{QOPS}
        \end{QBODY}
        \begin{QFROMS}
        \end{QFROMS}
        \begin{QTAGS}\QTAG{B1C3指對數函數}\end{QTAGS}
        \begin{QANS}
            (3)
        \end{QANS}
        \begin{QSOLLIST}
        \end{QSOLLIST}
        \begin{QEMPTYSPACE}
        \end{QEMPTYSPACE}
    \end{QUESTION}
    \begin{QUESTION}
        \begin{ExamInfo}{097}{學測}{單選}{2}
        \end{ExamInfo}
        \begin{ExamAnsRateInfo}{75}{96}{85}{44}
        \end{ExamAnsRateInfo}
        \begin{QBODY}
			在職棒比賽中 ERA 值是了解一個投手表現的重要統計數值。其計算方式如下:若此投手共主投 $n$ 局,其總責任失分為 $E$,則其 ERA 值為 $\frac{E}{n}\times 9$ 。有一位投手在之前的比賽中共主投了 90局,且這 90 局中他的 ERA 值為 3.2。在最新的一場比賽中此投手主投 6 局無責任失分,則打完這一場比賽後,此投手的 ERA 值成為 
			\begin{QOPS} 
				\QOP 2.9 
				\QOP 3.0  
				\QOP 3.1 
				\QOP 3.2  
				\QOP 3.3
			\end{QOPS}
        \end{QBODY}
        \begin{QFROMS}
        \end{QFROMS}
        \begin{QTAGS}\QTAG{綜合}\end{QTAGS}
        \begin{QANS}
            (2)
        \end{QANS}
        \begin{QSOLLIST}
        \end{QSOLLIST}
        \begin{QEMPTYSPACE}
        \end{QEMPTYSPACE}
    \end{QUESTION}
    \begin{QUESTION}
        \begin{ExamInfo}{097}{學測}{單選}{3}
        \end{ExamInfo}
        \begin{ExamAnsRateInfo}{73}{97}{84}{38}
        \end{ExamAnsRateInfo}
        \begin{QBODY}
			有一個圓形跑道分內、外兩圈,半徑分別為 30、50 公尺。今甲在內圈以等速行走、乙在外圈 以等速跑步,且知甲每走一圈,乙恰跑了兩圈。若甲走了 45 公尺,則同時段乙跑了 \\
			\begin{QOPSINONELINE} 
				\QOP 90 公尺 \QOP 120 公尺 \QOP 135 公尺 \QOP 150 公尺 \QOP 180 公尺
			\end{QOPSINONELINE}
        \end{QBODY}
        \begin{QFROMS}
        \end{QFROMS}
        \begin{QTAGS}\QTAG{綜合}\end{QTAGS}
        \begin{QANS}
            (4)
        \end{QANS}
        \begin{QSOLLIST}
        \end{QSOLLIST}
        \begin{QEMPTYSPACE}
        \end{QEMPTYSPACE}
    \end{QUESTION}
    \begin{QUESTION}
        \begin{ExamInfo}{097}{學測}{單選}{4}
        \end{ExamInfo}
        \begin{ExamAnsRateInfo}{39}{69}{31}{17}
        \end{ExamAnsRateInfo}
        \begin{QBODY}
			某地區的車牌號碼共六碼,其中前兩碼為 O 以外的英文大寫字母,後四碼為 0 到 9 的阿拉伯數字,但規定不能連續出現三個 4。例如: AA1234 , AB4434 為可出現的車牌號碼;而 AO1234 ,AB3444 為不可出現的車牌號碼。則所有第一碼為 A 且最後一碼為 4 的車牌號碼個數為 
			\begin{QOPS}
				\QOP $25 \times 9^3$ 
				\QOP $25\times 9^2 \times 10$        
				\QOP $25 \times 900$
				\QOP $25 \times 990$ 
				\QOP $25 \times 999$ 
			\end{QOPS}
        \end{QBODY}
        \begin{QFROMS}
        \end{QFROMS}
        \begin{QTAGS}\QTAG{B2C2排列組合}\end{QTAGS}
        \begin{QANS}
            (4)
        \end{QANS}
        \begin{QSOLLIST}
        \end{QSOLLIST}
        \begin{QEMPTYSPACE}
        \end{QEMPTYSPACE}
    \end{QUESTION}
    \begin{QUESTION}
        \begin{ExamInfo}{097}{學測}{單選}{5}
        \end{ExamInfo}
        \begin{ExamAnsRateInfo}{41}{74}{32}{17}
        \end{ExamAnsRateInfo}
        \begin{QBODY}
			廣場上插了一支紅旗與一支白旗,小明站在兩支旗子之間。		利用手邊的儀器,小明測出他與正東方紅旗間的距離為他與正西方白旗間距離的 6 倍;			小明往正北方走了 10 公尺之後再量一 次,發現他與紅旗的距離變成他與白旗距離的 4 倍。			試問紅白兩旗之間的距離最接近下列哪個選項?
			\begin{QOPS} 
				\QOP 60 公尺 
				\QOP 65 公尺    
				\QOP 70 公尺 
				\QOP 75 公尺 
				\QOP 80 公尺
			\end{QOPS}
        \end{QBODY}
        \begin{QFROMS}
        \end{QFROMS}
        \begin{QTAGS}\QTAG{B3C1三角}\end{QTAGS}
        \begin{QANS}
            (1)
        \end{QANS}
        \begin{QSOLLIST}
        \end{QSOLLIST}
        \begin{QEMPTYSPACE}
        \end{QEMPTYSPACE}
    \end{QUESTION}
\end{QUESTIONS}
\begin{QUESTIONS}
    \begin{QUESTION}
        \begin{ExamInfo}{097}{學測}{多選}{6}
        \end{ExamInfo}
        \begin{ExamAnsRateInfo}{56}{92}{58}{18}
        \end{ExamAnsRateInfo}
        \begin{QBODY}
			試問:在坐標平面上,下列哪些選項中的函數圖形完全落在 $x$ 軸的上方? 
			\begin{QOPS} 
				\QOP $y=x+100$
				\QOP $y=x^2+1$ 
				\QOP $y=2+\sin x$
				\QOP $y=2x$
				\QOP $y=\log x$
			\end{QOPS}
        \end{QBODY}
        \begin{QFROMS}
        \end{QFROMS}
        \begin{QTAGS}\QTAG{綜合}\end{QTAGS}
        \begin{QANS}
            (2)(3)(4)
        \end{QANS}
        \begin{QSOLLIST}
        \end{QSOLLIST}
        \begin{QEMPTYSPACE}
        \end{QEMPTYSPACE}
    \end{QUESTION}
    \begin{QUESTION}
        \begin{ExamInfo}{097}{學測}{多選}{7}
        \end{ExamInfo}
        \begin{ExamAnsRateInfo}{39}{60}{40}{17}
        \end{ExamAnsRateInfo}
        \begin{QBODY}
			某高中共有 20 個班級,每班各有 40 位學生,其中男生 25 人,女生 15 人。若從全校 800 人中以簡單隨機抽樣抽出 80 人,試問下列哪些選項是正確的? 
			\begin{QOPS} 
				\QOP 每班至少會有一人被抽中 
				\QOP 抽出來的男生人數一定比女生人數多 
				\QOP 已知小文是男生,小美是女生,則小文被抽中的機率大於小美被抽中的機率。 
				\QOP 若學生甲和學生乙在同一班,學生丙在另外一班,則甲、乙兩人同時被抽中的機率跟甲、 丙兩人同時被抽中的機率一樣 
				\QOP 學生 $A$ 和學生 $B$ 是兄弟,他們同時被抽中的機率小於 $\frac{1}{100}$
            \end{QOPS}
        \end{QBODY}
        \begin{QFROMS}
        \end{QFROMS}
        \begin{QTAGS}\QTAG{B2C3機率}\end{QTAGS}
        \begin{QANS}
            (4)(5)
        \end{QANS}
        \begin{QSOLLIST}
        \end{QSOLLIST}
        \begin{QEMPTYSPACE}
        \end{QEMPTYSPACE}
    \end{QUESTION}
    \begin{QUESTION}
        \begin{ExamInfo}{097}{學測}{多選}{8}
        \end{ExamInfo}
        \begin{ExamAnsRateInfo}{36}{57}{36}{15}
        \end{ExamAnsRateInfo}
        \begin{QBODY}
			已知 $a_1 , a_2 , a_3$ 為一等差數列,而 $b_1 , b_2 , b_3$ 為一等比數列,且此六數皆為實數。試問下列哪些選項是正確的?
			\begin{QOPS} 
				\QOP  $a_1 <a_2$ 與 $a_2 >a_3$ 可能同時成立 
				\QOP $b_1 <b_2$ 與 $b_2 >b_3$ 可能同時成立 \quad 
				\QOP 若 $a_1 +a_2 <0$,則 $a_2 +a_3 <0$ 
				\QOP 若 $b_1b_2 <0$ ,則 $b_2b_3 <0$ 
				\QOP 若 $b_1,b_2,b_3$ 皆為正整數且 $b_1 <b_2$,則 $b_1$ 整除 $b_2$
			\end{QOPS}
        \end{QBODY}
        \begin{QFROMS}
        \end{QFROMS}
        \begin{QTAGS}\QTAG{B2C1數列級數}\end{QTAGS}
        \begin{QANS}
            (2)(4)
        \end{QANS}
        \begin{QSOLLIST}
        \end{QSOLLIST}
        \begin{QEMPTYSPACE}
        \end{QEMPTYSPACE}
    \end{QUESTION}
    \begin{QUESTION}
        \begin{ExamInfo}{097}{學測}{多選}{9}
        \end{ExamInfo}
        \begin{ExamAnsRateInfo}{33}{65}{24}{10}
        \end{ExamAnsRateInfo}
        \begin{QBODY}
			已知在一容器中有 $A,B$ 兩種菌,且在任何時刻 $A,B$ 兩種菌的個數乘積為定值 $10^{10}$ 。
            為了簡單起見,科學家用 $P_A =\log (n_A)$ 來記錄 $A$ 菌個數的資料,其中 $n$  為 $A$ 菌的個數。
            試問下列哪些選項是正確的?
			\begin{QOPS}
				\QOP $1 \leq P_A \leq 10$
				\QOP 當 $P_A =5$ 時,$B$菌的個數與 $A$ 菌的個數相同 
				\QOP 如果上週一測得 $P_A$ 值為 4 而上週五測得 $P_A$ 值為 8,表示上週五 $A$ 菌的個數是上週一 $A$ 菌
			個數的 2 倍 
				\QOP 若今天的 $P_A$ 值比昨天增加 1,則今天的 A 菌比昨天多了 10 個 
				\QOP 假設科學家將 $B$ 菌的個數控制為 5 萬個,則此時 $5<P_A <5.5$
			\end{QOPS}
        \end{QBODY}
        \begin{QFROMS}
        \end{QFROMS}
        \begin{QTAGS}\QTAG{B1C3指對數函數}\end{QTAGS}
        \begin{QANS}
            (2)(5)
        \end{QANS}
        \begin{QSOLLIST}
        \end{QSOLLIST}
        \begin{QEMPTYSPACE}
        \end{QEMPTYSPACE}
    \end{QUESTION}
    \begin{QUESTION}
        \begin{ExamInfo}{097}{學測}{多選}{10}
        \end{ExamInfo}
        \begin{ExamAnsRateInfo}{26}{52}{16}{10}
        \end{ExamAnsRateInfo}
        \begin{QBODY}
			已知實係數多項式 $f(x)$ 與 $g(x) = x^3 + x^2 - 2$ 有次數大於 $0$ 的公因式。
			試問下列哪些選項是正確的?
			\begin{QOPS}
				\QOP $g(x)=0$ 恰有一實根 
				\QOP $f(x)=0$ 必有實根 
				\QOP 若 $f(x)=0$ 與 $g(x)=0$ 有共同實根,則此實根必為1 
				\QOP 若 $f(x)=0$ 與 $g(x)=0$ 有共同實根,則 $f(x)$ 與 $g(x)$ 的最高公因式為一次式 
				\QOP 若 $f(x)=0$與 $g(x)=0$ 沒有共同實根,則 $f(x)$ 與 $g(x)$ 的最高公因式為二次式
			\end{QOPS}
        \end{QBODY}
        \begin{QFROMS}
        \end{QFROMS}
        \begin{QTAGS}\QTAG{B1C2多項式函數}\end{QTAGS}
        \begin{QANS}
            (1)(3)(5)
        \end{QANS}
        \begin{QSOLLIST}
        \end{QSOLLIST}
        \begin{QEMPTYSPACE}
        \end{QEMPTYSPACE}
    \end{QUESTION}
    \begin{QUESTION}
        \begin{ExamInfo}{097}{學測}{多選}{11}
        \end{ExamInfo}
        \begin{ExamAnsRateInfo}{27}{51}{19}{11}
        \end{ExamAnsRateInfo}
        \begin{QBODY}
			設坐標空間中三條直線 $L_1$, $L_2$, $L_3$ 的方程式分別為 $L_1: \frac{x}{1} = \frac{y+3}{6} = \frac{z+4}{8}$;$L_2: \frac{x}{1} = \frac{y+3}{3} = \frac{z+4}{4}$; $L_3: \frac{x}{1} = \frac{y}{3} = \frac{z}{4}$;
			試問下列哪些選項是正確的?
			\begin{QOPS} 
				\QOP $L_1$ 與 $L_2$ 相交 
				\QOP $L_2$ 與 $L_3$ 平行 
				\QOP 點 $P(0,-3,-4)$ 與$Q(0,0,0)$ 的距離即為點 $P$ 到 $L_3$ 的最短距離 
				\QOP 直線 $L: \left\{ \begin{array}{rcl} x & = & 0 \\ \frac{y+3}{4} &=& \frac{z+4}{-3} \end{array} \right.$ 與直線 $L_{1}$, $L_{2}$皆垂直
				\QOP 三直線 $L_1$ , $L_2$ , $L_3$ 共平面 
			\end{QOPS}
        \end{QBODY}
        \begin{QFROMS}
        \end{QFROMS}
        \begin{QTAGS}\QTAG{B4C1空間向量}\end{QTAGS}
        \begin{QANS}
            (1)(2)(4)(5)
        \end{QANS}
        \begin{QSOLLIST}
        \end{QSOLLIST}
        \begin{QEMPTYSPACE}
        \end{QEMPTYSPACE}
    \end{QUESTION}
    \begin{QUESTION}
        \begin{ExamInfo}{097}{學測}{多選}{12}
        \end{ExamInfo}
        \begin{ExamAnsRateInfo}{34}{62}{29}{11}
        \end{ExamAnsRateInfo}
        \begin{QBODY}
			設 $\Gamma : x^2 + y^2 - 10x + 9 = 0$ 為坐標平面上的圓。試問下列哪些選項是正確的? 
			\begin{QOPS} 
				\QOP $\Gamma$ 的圓心坐標為 $(5,0)$ 
				\QOP $\Gamma:$上的點與直線 $L: 3x+4y-15=0$ 的最遠距離等於 $4 $
				\QOP 直線 $L_1 :3x+4y+15=0$ 與 $\Gamma$ 相切    
                \QOP $\Gamma$ 上恰有兩個點與直線 $L_2 :3x+4y=0 $的距離等於$2$ 
				\QOP $\Gamma$ 上恰有四個點與直線 $L_3 :3x+4y - 5=0$ 的距離等於$2$
			\end{QOPS}
        \end{QBODY}
        \begin{QFROMS}
        \end{QFROMS}
        \begin{QTAGS}\QTAG{B3C2直線與圓}\end{QTAGS}
        \begin{QANS}
            (1)(2)(4)
        \end{QANS}
        \begin{QSOLLIST}
        \end{QSOLLIST}
        \begin{QEMPTYSPACE}
        \end{QEMPTYSPACE}
    \end{QUESTION}
\end{QUESTIONS}
\begin{QUESTIONS}
    \begin{QUESTION}
        \begin{ExamInfo}{097}{學測}{填充}{A}
        \end{ExamInfo}
        \begin{ExamAnsRateInfo}{47}{82}{51}{8}
        \end{ExamAnsRateInfo}
        \begin{QBODY}
			令 $A(-1,6,0)$, $B(3,-1,-2)$, $C(4,4,5)$ 為坐標空間中三點。
			若 $D$ 為空間中的一點且滿足 $3 \cdot \lvec{DA} -  4 \cdot \lvec{DB} + 2\cdot\lvec{DC}= \vec{0}$ ,
			則點 $D$ 的坐標為 $\TCNBOX{(\TCN\TCN,\TCN\TCN,\TCN\TCN)}$ 。
        \end{QBODY}
        \begin{QFROMS}
        \end{QFROMS}
        \begin{QTAGS}\QTAG{B4C1空間向量}\end{QTAGS}
        \begin{QANS}
            $(-7,30,18)$
        \end{QANS}
        \begin{QSOLLIST}
        \end{QSOLLIST}
        \begin{QEMPTYSPACE}
        \end{QEMPTYSPACE}
    \end{QUESTION}
    \begin{QUESTION}
        \begin{ExamInfo}{097}{學測}{填充}{B}
        \end{ExamInfo}
        \begin{ExamAnsRateInfo}{63}{96}{75}{18}
        \end{ExamAnsRateInfo}
        \begin{QBODY}
			在坐標平面上,設$A$為直線 $3x-y=0$ 上一點, $B$為 $x$ 軸上一點。若線段 $AB$ 的中點坐標為 $(\frac{7}{2},6)$, 則點 $A$ 的坐標為 $\TCNBOX{(\TCN,\TCN\TCN)}$,點 $B$ 的坐標為 $\TCNBOX{\TCN,\TCN}$。
        \end{QBODY}
        \begin{QFROMS}
        \end{QFROMS}
        \begin{QTAGS}\QTAG{B3C3平面向量}\end{QTAGS}
        \begin{QANS}
            $(4,12), (3,0)$
        \end{QANS}
        \begin{QSOLLIST}
        \end{QSOLLIST}
        \begin{QEMPTYSPACE}
        \end{QEMPTYSPACE}
    \end{QUESTION}
    \begin{QUESTION}
        \begin{ExamInfo}{097}{學測}{填充}{C}
        \end{ExamInfo}
        \begin{ExamAnsRateInfo}{19}{51}{6}{0}
        \end{ExamAnsRateInfo}
        \begin{QBODY}
			坐標平面上,以原點 $O$ 為圓心的圓上有三個相異點 $A(1, 0)$, $B$, $C$ ,且 $\overline{AB} = \overline{BC}$ 。已知銳角三角形 $OAB$ 的面積為 $\frac{3}{10}$,則 $\triangle OAC$ 的面積為 $\TCNBOX{\FR{\TCN\TCN}{\TCN\TCN}}$。
        \end{QBODY}
        \begin{QFROMS}
        \end{QFROMS}
        \begin{QTAGS}\QTAG{B3C1三角}\end{QTAGS}
        \begin{QANS}
            $\dfrac{12}{25}$
        \end{QANS}
        \begin{QSOLLIST}
        \end{QSOLLIST}
        \begin{QEMPTYSPACE}
        \end{QEMPTYSPACE}
    \end{QUESTION}
    \begin{QUESTION}
        \begin{ExamInfo}{097}{學測}{填充}{D}
        \end{ExamInfo}
        \begin{ExamAnsRateInfo}{35}{72}{27}{6}
        \end{ExamAnsRateInfo}
        \begin{QBODY}
			設 $F_1$ 與 $F_2$ 為坐標平面上雙曲線 $\Gamma : \frac{x^2}{8} - y^2 =1$ 的兩個焦點,且 $P(-4,1)$ 為 $\Gamma$ 上一點。若 $\angle F_1PF_2$ 的角平分線與 $x$ 軸交於點 $D$,則 $D$ 的 $x$ 坐標為 
$\TCNBOX{\TCN\TCN}$ 。
        \end{QBODY}
        \begin{QFROMS}
        \end{QFROMS}
        \begin{QTAGS}\QTAG{B4C4二次曲線}\end{QTAGS}
        \begin{QANS}
            $-2$
        \end{QANS}
        \begin{QSOLLIST}
        \end{QSOLLIST}
        \begin{QEMPTYSPACE}
        \end{QEMPTYSPACE}
    \end{QUESTION}
    \begin{QUESTION}
        \begin{ExamInfo}{097}{學測}{填充}{E}
        \end{ExamInfo}
        \begin{ExamAnsRateInfo}{13}{37}{2}{0}
        \end{ExamAnsRateInfo}
        \begin{QBODY}
			設 $O(0, 0, 0)$ 為坐標空間中某長方體的一個頂點,且知 $(2, 2,1)$, $(2, -1, -2)$, $(3, -6, 6)$ 為此長方體中與 $O$ 相鄰的三頂點。若平面 $E : x + by + cz = d$ 將此長方體截成兩部分,其中包含頂點 $O$ 的那一部分是個正立方體,則 $(b,c,d)= 
\TCNBOX{(\TCN\TCN,\TCN,\TCN)}$ 。
        \end{QBODY}
        \begin{QFROMS}
        \end{QFROMS}
        \begin{QTAGS}\QTAG{B4C1空間向量}\end{QTAGS}
        \begin{QANS}
            $(-2,2,9)$
        \end{QANS}
        \begin{QSOLLIST}
        \end{QSOLLIST}
        \begin{QEMPTYSPACE}
        \end{QEMPTYSPACE}
    \end{QUESTION}
    \begin{QUESTION}
        \begin{ExamInfo}{097}{學測}{填充}{F}
        \end{ExamInfo}
        \begin{ExamAnsRateInfo}{20}{34}{19}{7}
        \end{ExamAnsRateInfo}
        \begin{QBODY}
			設 $a ,b$ 為正整數。若 $b^2 =9a$,且 $a+2b>280$,則 $a$ 的最小可能值為 
$\TCNBOX{\TCN\TCN}$。
        \end{QBODY}
        \begin{QFROMS}
        \end{QFROMS}
        \begin{QTAGS}\QTAG{B1C2多項式函數}\end{QTAGS}
        \begin{QANS}
            $225$
        \end{QANS}
        \begin{QSOLLIST}
        \end{QSOLLIST}
        \begin{QEMPTYSPACE}
        \end{QEMPTYSPACE}
    \end{QUESTION}
    \begin{QUESTION}
        \begin{ExamInfo}{097}{學測}{填充}{G}
        \end{ExamInfo}
        \begin{ExamAnsRateInfo}{30}{65}{19}{6}
        \end{ExamAnsRateInfo}
        \begin{QBODY}
			坐標平面上有一質點沿方向 $\lvec{u} = (1, 2)$ 前進。現欲在此平面上置一直線 $L$,
			使得此質點碰到 $L$ 時 依光學原理(入射角等於反射角)反射,之後沿方向 $\lvec{v} = (-2, 1)$ 前進,則直線 $L$ 的方向向量應為 $(1, 
			\TCNBOX{\TCN\TCN})$。
        \end{QBODY}
        \begin{QFROMS}
        \end{QFROMS}
        \begin{QTAGS}\QTAG{B3C3平面向量}\end{QTAGS}
        \begin{QANS}
            $-3$
        \end{QANS}
        \begin{QSOLLIST}
        \end{QSOLLIST}
        \begin{QEMPTYSPACE}
        \end{QEMPTYSPACE}
    \end{QUESTION}
    \begin{QUESTION}
        \begin{ExamInfo}{097}{學測}{填充}{H}
        \end{ExamInfo}
        \begin{ExamAnsRateInfo}{10}{29}{1}{0}
        \end{ExamAnsRateInfo}
        \begin{QBODY}
			已知坐標平面上圓 $O_1 :(x-7)^2 +(y-1)^2 =144$ 與 $O_2 : (x+2)^2 +(y-13)^2 =9$ 相切,且此兩圓均與直線 $L:	x = -5$ 相切。若 $\Gamma$ 為以 $L$ 為準線的拋物線,且同時通過 $O_1$ 與 $O_2$ 的圓心,則 $\Gamma$ 的焦點坐標為 
$\TCNBOX{(\frac{\TCN\TCN}{\TCN},\frac{\TCN\TCN}{\TCN})}$ 。
        \end{QBODY}
        \begin{QFROMS}
        \end{QFROMS}
        \begin{QTAGS}\QTAG{B4C4二次曲線}\end{QTAGS}
        \begin{QANS}
            $(\frac{-1}{5},\frac{53}{5})$
        \end{QANS}
        \begin{QSOLLIST}
        \end{QSOLLIST}
        \begin{QEMPTYSPACE}
        \end{QEMPTYSPACE}
    \end{QUESTION}
\end{QUESTIONS}
