% !TEX encoding = UTF-8 Unicode
% !TEX TS-program = xelatex 
\begin{QUESTIONS}
    \begin{QUESTION}
        \begin{ExamInfo}{83}{學測}{單選}{1}
        \end{ExamInfo}
        \begin{ExamAnsRateInfo}{}{}{}{}
        \end{ExamAnsRateInfo}
        \begin{QBODY}
            設$a=\sqrt{7+\sqrt{47}}$,則a在那兩個連續整數之間?(單選)
            \begin{QOPS}
                \QOP 0與1
                \QOP 1與2
                \QOP 2與3
                \QOP 3與4
                \QOP 4與5
            \end{QOPS}            
        \end{QBODY}
        \begin{QFROMS}
        \end{QFROMS}
        \begin{QTAGS}\end{QTAGS}
        \begin{QANS}
            (4)
        \end{QANS}
        \begin{QSOLLIST}
        \end{QSOLLIST}
        \begin{QEMPTYSPACE}
        \end{QEMPTYSPACE}
    \end{QUESTION}
    \begin{QUESTION}
        \begin{ExamInfo}{83}{學測}{單選}{2}
        \end{ExamInfo}
        \begin{ExamAnsRateInfo}{}{}{}{}
        \end{ExamAnsRateInfo}
        \begin{QBODY}
            設直線L的方程式為$\frac{x-2}{3}=\frac{y+1}{-1}=\frac{z-1}{2}$,則下列那一個平面與L平行。(單選)
            \begin{QOPS}
                \QOP $2x-y+z=1$
                \QOP $x+y-z=2$
                \QOP $3x-y+2z=1$
                \QOP $3x+2y+z=2$
                \QOP $x-3y+z=1$    
            \end{QOPS}        
        \end{QBODY}
        \begin{QFROMS}
        \end{QFROMS}
        \begin{QTAGS}\end{QTAGS}
        \begin{QANS}
            (2)
        \end{QANS}
        \begin{QSOLLIST}
        \end{QSOLLIST}
        \begin{QEMPTYSPACE}
        \end{QEMPTYSPACE}
    \end{QUESTION}
    \begin{QUESTION}
        \begin{ExamInfo}{83}{學測}{單選}{3}
        \end{ExamInfo}
        \begin{ExamAnsRateInfo}{}{}{}{}
        \end{ExamAnsRateInfo}
        \begin{QBODY}
            同時擲兩枚均勻的硬幣,連續擲兩次,問至少有一次出現一正面一反面的機率為多少?(單選)
            \begin{QOPS}
            \QOP 0
            \QOP $\frac{1}{4}$
            \QOP $\frac{1}{2}$
            \QOP $\frac{3}{4}$
            \QOP 1
            \end{QOPS}        
        \end{QBODY}
        \begin{QFROMS}
        \end{QFROMS}
        \begin{QTAGS}\end{QTAGS}
        \begin{QANS}
            (4)
        \end{QANS}
        \begin{QSOLLIST}
        \end{QSOLLIST}
        \begin{QEMPTYSPACE}
        \end{QEMPTYSPACE}
    \end{QUESTION}
    \begin{QUESTION}
        \begin{ExamInfo}{83}{學測}{單選}{4}
        \end{ExamInfo}
        \begin{ExamAnsRateInfo}{}{}{}{}
        \end{ExamAnsRateInfo}
        \begin{QBODY}
            設圖1中,A、B、C三點共線,D、E、F三點共線。利用這六點的3個點作頂點所形成的三角形共有多少個?(單選)
            \begin{QOPS}
            \QOP 9
            \QOP 14
            \QOP 16
            \QOP 18
            \QOP 20
            \end{QOPS}        
            %TODO: 補圖
        \end{QBODY}
        \begin{QFROMS}
        \end{QFROMS}
        \begin{QTAGS}\end{QTAGS}
        \begin{QANS}
            (4)
        \end{QANS}
        \begin{QSOLLIST}
        \end{QSOLLIST}
        \begin{QEMPTYSPACE}
        \end{QEMPTYSPACE}
    \end{QUESTION}
    \begin{QUESTION}
        \begin{ExamInfo}{83}{學測}{單選}{5}
        \end{ExamInfo}
        \begin{ExamAnsRateInfo}{}{}{}{}
        \end{ExamAnsRateInfo}
        \begin{QBODY}
            %TODO: 補表
            	甲、乙、丙三位同學參加推薦甄選學科能力測驗,五科的成績如表一所示。設${{S}_{\text{甲}}}$、${{S}_{\text{乙}}}$、${{S}_{\text{丙}}}$分別代表甲、乙、丙三位同學五科成績的標準差。請仔細觀察表中數據,判斷下列那一選項表示${{S}_{\text{甲}}}$、${{S}_{\text{乙}}}$、${{S}_{\text{丙}}}$的大小關係?(單選)
                \begin{QOPS}
                    \QOP ${{S}_{\text{甲}}}>{{S}_{\text{丙}}}>{{S}_{\text{乙}}}$	
                    \QOP ${{S}_{\text{丙}}}>{{S}_{\text{甲}}}={{S}_{\text{乙}}}$
                    \QOP ${{S}_{\text{甲}}}>{{S}_{\text{丙}}}={{S}_{\text{乙}}}$	
                    \QOP ${{S}_{\text{乙}}}>{{S}_{\text{甲}}}={{S}_{\text{丙}}}$
                    \QOP ${{S}_{\text{甲}}}={{S}_{\text{乙}}}>{{S}_{\text{丙}}}$
                \end{QOPS}        
        \end{QBODY}
        \begin{QFROMS}
        \end{QFROMS}
        \begin{QTAGS}\end{QTAGS}
        \begin{QANS}
            (5)
        \end{QANS}
        \begin{QSOLLIST}
        \end{QSOLLIST}
        \begin{QEMPTYSPACE}
        \end{QEMPTYSPACE}
    \end{QUESTION}
    \begin{QUESTION}
        \begin{ExamInfo}{83}{學測}{單選}{6}
        \end{ExamInfo}
        \begin{ExamAnsRateInfo}{}{}{}{}
        \end{ExamAnsRateInfo}
        \begin{QBODY}
            若$ x=\frac{\sqrt[3]{88.3}}{\sqrt{2.56}}$ 則下列那一個敘述是正確的?(可用查表法)(單選)
            \begin{QOPS}
            \QOP $2.8<x<2.9$	
            \QOP $2.7<x<2.8$	
            \QOP $2.6<x<2.7$
            \QOP $2.5<x<2.6$	
            \QOP $2.4<x<2.5$ 
            \end{QOPS}        
        \end{QBODY}
        \begin{QFROMS}
        \end{QFROMS}
        \begin{QTAGS}\end{QTAGS}
        \begin{QANS}
            (2)
        \end{QANS}
        \begin{QSOLLIST}
        \end{QSOLLIST}
        \begin{QEMPTYSPACE}
        \end{QEMPTYSPACE}
    \end{QUESTION}
    \begin{QUESTION}
        \begin{ExamInfo}{83}{學測}{單選}{7}
        \end{ExamInfo}
        \begin{ExamAnsRateInfo}{}{}{}{}
        \end{ExamAnsRateInfo}
        \begin{QBODY}
            %TODO:補圖
            武林高手上官琴魔,幸獲至寶「斷腸一弦琴」。如圖2實線部分,琴身為一圓弧,琴弦$\overline{AB}$長為1.6尺。今欲增其威力,需加一長為1.2尺的平行琴弦,乃在P及Q點鑽孔,加裝琴弦$\overline{PQ}$。若知圓心在O點,半徑為1尺,敢問少(女)俠$\angle AOP$大小若干?(單選)
            \begin{QOPS}
                \QOP ${{13}^{{}^\circ }}<\angle AOP\le {{14}^{{}^\circ }}$	
                \QOP ${{14}^{{}^\circ }}<\angle AOP\le {{15}^{{}^\circ }}$
                \QOP ${{15}^{{}^\circ }}<\angle AOP\le {{16}^{{}^\circ }}$	
                \QOP ${{16}^{{}^\circ }}<\angle AOP\le {{17}^{{}^\circ }}$
                \QOP ${{17}^{{}^\circ }}<\angle AOP\le {{18}^{{}^\circ }}$
            \end{QOPS}        
        \end{QBODY}
        \begin{QFROMS}
        \end{QFROMS}
        \begin{QTAGS}\end{QTAGS}
        \begin{QANS}
            (4)
        \end{QANS}
        \begin{QSOLLIST}
        \end{QSOLLIST}
        \begin{QEMPTYSPACE}
        \end{QEMPTYSPACE}
    \end{QUESTION}
\end{QUESTIONS}\begin{QUESTIONS}
    \begin{QUESTION}
        \begin{ExamInfo}{83}{學測}{多選}{8}
        \end{ExamInfo}
        \begin{ExamAnsRateInfo}{}{}{}{}
        \end{ExamAnsRateInfo}
        \begin{QBODY}
            %TODO:補圖
            若函數$f\left( x \right)=a{{x}^{2}}+bx+c$的圖形如圖3,則下列各數那些為負數?(多選)
            \begin{QOPS}
            \QOP a
            \QOP b
            \QOP c
            \QOP ${{b}^{2}}-4ac$ 
            \QOP $a-b+c$
            \end{QOPS}        
        \end{QBODY}
        \begin{QFROMS}
        \end{QFROMS}
        \begin{QTAGS}\end{QTAGS}
        \begin{QANS}
            (3)(5)
        \end{QANS}
        \begin{QSOLLIST}
        \end{QSOLLIST}
        \begin{QEMPTYSPACE}
        \end{QEMPTYSPACE}
    \end{QUESTION}
    \begin{QUESTION}
        \begin{ExamInfo}{83}{學測}{多選}{9}
        \end{ExamInfo}
        \begin{ExamAnsRateInfo}{}{}{}{}
        \end{ExamAnsRateInfo}
        \begin{QBODY}
            下列有關空間的敘述,那些是正確的?(多選)
            \begin{QOPS}
            \QOP 過已知直線外一點,「恰有」一平面與此直線垂直
            \QOP 過已知直線外一點,「恰有」一平面與此直線平行
            \QOP 過已知平面外一點,「恰有」一直線與此平面平行
            \QOP 過已知平面外一點,「恰有」一平面與此平面垂直
            \QOP 過已知平面外一點,「恰有」一平面與此平面平行
            \end{QOPS}        
        \end{QBODY}
        \begin{QFROMS}
        \end{QFROMS}
        \begin{QTAGS}\end{QTAGS}
        \begin{QANS}
            (1)(5)
        \end{QANS}
        \begin{QSOLLIST}
        \end{QSOLLIST}
        \begin{QEMPTYSPACE}
        \end{QEMPTYSPACE}
    \end{QUESTION}
    \begin{QUESTION}
        \begin{ExamInfo}{83}{學測}{多選}{10}
        \end{ExamInfo}
        \begin{ExamAnsRateInfo}{}{}{}{}
        \end{ExamAnsRateInfo}
        \begin{QBODY}
            下列那些方程式的部分圖形「不可能」出現在圖4中?(多選)
            \begin{QOPS}
            \QOP $y={{\left( \frac{1}{2} \right)}^{x}}$
            \QOP $y={{\log }_{2}}x$
            \QOP $y=\cot x$
            \QOP $5{{x}^{2}}+4x-6y-3=0$
            \QOP ${{x}^{2}}-{{y}^{2}}+4x-6y-10=0$ 
        \end{QOPS}        
        \end{QBODY}
        \begin{QFROMS}
        \end{QFROMS}
        \begin{QTAGS}\end{QTAGS}
        \begin{QANS}
            (3)(4)(5)
        \end{QANS}
        \begin{QSOLLIST}
        \end{QSOLLIST}
        \begin{QEMPTYSPACE}
        \end{QEMPTYSPACE}
    \end{QUESTION}
\end{QUESTIONS}\begin{QUESTIONS}
    \begin{QUESTION}
        \begin{ExamInfo}{83}{學測}{選填}{11}
        \end{ExamInfo}
        \begin{ExamAnsRateInfo}{}{}{}{}
        \end{ExamAnsRateInfo}
        \begin{QBODY}
            函數$y={{4}^{x}}$與$y={{2}^{3x+2}}$的圖形之交點坐標為 $\TCNBOX{(\TCN\TCN,\FR{\TCN}{\TCN\TCN})}$。
        \end{QBODY}
        \begin{QFROMS}
        \end{QFROMS}
        \begin{QTAGS}\end{QTAGS}
        \begin{QANS}
            $(-2, \FR{1}{16})$
        \end{QANS}
        \begin{QSOLLIST}
        \end{QSOLLIST}
        \begin{QEMPTYSPACE}
        \end{QEMPTYSPACE}
    \end{QUESTION}
    \begin{QUESTION}
        \begin{ExamInfo}{83}{學測}{選填}{12}
        \end{ExamInfo}
        \begin{ExamAnsRateInfo}{}{}{}{}
        \end{ExamAnsRateInfo}
        \begin{QBODY}
            一皮球自離地面10公尺高處落下。首次反彈高度為$\frac{10}{3}$公尺,此後每次反彈高度為其前次反彈高度的$\frac{1}{3}$,則此球到完全靜止前,所經過路徑的總長度為$\TCNBOX{\TCN\TCN}$公尺。
        \end{QBODY}
        \begin{QFROMS}
        \end{QFROMS}
        \begin{QTAGS}\end{QTAGS}
        \begin{QANS}
            $20$
        \end{QANS}
        \begin{QSOLLIST}
        \end{QSOLLIST}
        \begin{QEMPTYSPACE}
        \end{QEMPTYSPACE}
    \end{QUESTION}
    \begin{QUESTION}
        \begin{ExamInfo}{83}{學測}{選填}{13}
        \end{ExamInfo}
        \begin{ExamAnsRateInfo}{}{}{}{}
        \end{ExamAnsRateInfo}
        \begin{QBODY}
            平面上四點$A\left( -1,\,2 \right)$,$B\left( 4,\,2 \right)$,$C\left( 2,\,-1 \right)$和$O\left( 0,\,0 \right)$。過B點作直線OC的平行線交直線OA於D點,則D點的坐標為 $\TCNBOX {(\FR{\TCN\TCN}{\TCN},\FR{\TCN\TCN}{\TCN})}$。
        \end{QBODY}
        \begin{QFROMS}
        \end{QFROMS}
        \begin{QTAGS}\end{QTAGS}
        \begin{QANS}
            $(\FR{-8}{3}, \FR{16}{3})$
        \end{QANS}
        \begin{QSOLLIST}
        \end{QSOLLIST}
        \begin{QEMPTYSPACE}
        \end{QEMPTYSPACE}
    \end{QUESTION}
    \begin{QUESTION}
        \begin{ExamInfo}{83}{學測}{選填}{14}
        \end{ExamInfo}
        \begin{ExamAnsRateInfo}{}{}{}{}
        \end{ExamAnsRateInfo}
        \begin{QBODY}
            已知$A\left( 1,\,2 \right)$與$B\left( 3,\,4 \right)$為兩定點,$P\left( x,\,y \right)$為直線$x+2y=3$上一點。問$\overline{PA}=\overline{PB}$時,P的坐標為 $\TCNBOX{(\TCN,\TCN\TCN)}$ 。
        \end{QBODY}
        \begin{QFROMS}
        \end{QFROMS}
        \begin{QTAGS}\end{QTAGS}
        \begin{QANS}
            $(7,-2)$
        \end{QANS}
        \begin{QSOLLIST}
        \end{QSOLLIST}
        \begin{QEMPTYSPACE}
        \end{QEMPTYSPACE}
    \end{QUESTION}
    \begin{QUESTION}
        \begin{ExamInfo}{83}{學測}{選填}{15}
        \end{ExamInfo}
        \begin{ExamAnsRateInfo}{}{}{}{}
        \end{ExamAnsRateInfo}
        \begin{QBODY}
            5.	若直線$L:y=mx+3$與圓${{x}^{2}}+{{y}^{2}}+2x=3$相切,則$m=\TCNBOX{\FR{\TCN\TCN \pm \TCN \sqrt{\TCN}}{\TCN}}$ 。
        \end{QBODY}
        \begin{QFROMS}
        \end{QFROMS}
        \begin{QTAGS}\end{QTAGS}
        \begin{QANS}
            $\FR{-3\pm 2\sqrt{6}}{3}$
        \end{QANS}
        \begin{QSOLLIST}
        \end{QSOLLIST}
        \begin{QEMPTYSPACE}
        \end{QEMPTYSPACE}
    \end{QUESTION}
    \begin{QUESTION}
        \begin{ExamInfo}{83}{學測}{選填}{16}
        \end{ExamInfo}
        \begin{ExamAnsRateInfo}{}{}{}{}
        \end{ExamAnsRateInfo}
        \begin{QBODY}
            平面 $x+3y+z=1$與球面$x^2 +y^2 +z^2 +2x-4y-11=0$ 相交成一個圓。則此圓的面積為$\TCNBOX{\FR{\TCN\TCN\TCN}{11}\pi}$ 。
        \end{QBODY}
        \begin{QFROMS}
        \end{QFROMS}
        \begin{QTAGS}\end{QTAGS}
        \begin{QANS}
            $\FR{160 \pi }{11}$
        \end{QANS}
        \begin{QSOLLIST}
        \end{QSOLLIST}
        \begin{QEMPTYSPACE}
        \end{QEMPTYSPACE}
    \end{QUESTION}
    \begin{QUESTION}
        \begin{ExamInfo}{83}{學測}{選填}{17}
        \end{ExamInfo}
        \begin{ExamAnsRateInfo}{}{}{}{}
        \end{ExamAnsRateInfo}
        \begin{QBODY}
            設$L$為$x-y+z=1$與$x+y-z=1$兩平面的交線,則直線$L$上與點$\left( 1,\,2,\,3 \right)$距離最近之點的坐標為  $\TCNBOX{(\TCN, \FR{\TCN}{\TCN}, \FR{\TCN}{\TCN})}$ 。
        \end{QBODY}
        \begin{QFROMS}
        \end{QFROMS}
        \begin{QTAGS}\end{QTAGS}
        \begin{QANS}
            $(1,\FR{5}{2}, \FR{5}{2})$
        \end{QANS}
        \begin{QSOLLIST}
        \end{QSOLLIST}
        \begin{QEMPTYSPACE}
        \end{QEMPTYSPACE}
    \end{QUESTION}
    \begin{QUESTION}
        \begin{ExamInfo}{83}{學測}{選填}{18}
        \end{ExamInfo}
        \begin{ExamAnsRateInfo}{}{}{}{}
        \end{ExamAnsRateInfo}
        \begin{QBODY}
            每次用20根相同的火柴棒圍成一個三角形,共可圍成 $\TCNBOX{\TCN}$種不全等的三角形。
        \end{QBODY}
        \begin{QFROMS}
        \end{QFROMS}
        \begin{QTAGS}\end{QTAGS}
        \begin{QANS}
            $8$
        \end{QANS}
        \begin{QSOLLIST}
        \end{QSOLLIST}
        \begin{QEMPTYSPACE}
        \end{QEMPTYSPACE}
    \end{QUESTION}
    \begin{QUESTION}
        \begin{ExamInfo}{83}{學測}{選填}{19}
        \end{ExamInfo}
        \begin{ExamAnsRateInfo}{}{}{}{}
        \end{ExamAnsRateInfo}
        \begin{QBODY}
        \end{QBODY}
        \begin{QFROMS}
            	若$\frac{3\pi }{2}<\theta <2\pi $且$\sin \theta +\cos \theta =\frac{1}{5}$,則$\cos \theta =\TCNBOX{\FR{\TCN}{\TCN}}$ 。
        \end{QFROMS}
        \begin{QTAGS}\end{QTAGS}
        \begin{QANS}
            $\FR{4}{5}$
        \end{QANS}
        \begin{QSOLLIST}
        \end{QSOLLIST}
        \begin{QEMPTYSPACE}
        \end{QEMPTYSPACE}
    \end{QUESTION}
    \begin{QUESTION}
        \begin{ExamInfo}{83}{學測}{選填}{20}
        \end{ExamInfo}
        \begin{ExamAnsRateInfo}{}{}{}{}
        \end{ExamAnsRateInfo}
        \begin{QBODY}
            	已知p為常數,若${{x}^{2}}+px+6$與${{x}^{3}}+px+6$的最低公倍式為四次式,則$p=\TCNBOX{\TCN\TCN}$。
        \end{QBODY}
        \begin{QFROMS}
        \end{QFROMS}
        \begin{QTAGS}\end{QTAGS}
        \begin{QANS}
            $-7$
        \end{QANS}
        \begin{QSOLLIST}
        \end{QSOLLIST}
        \begin{QEMPTYSPACE}
        \end{QEMPTYSPACE}
    \end{QUESTION}
\end{QUESTIONS}