% !TEX encoding = UTF-8 Unicode
% !TEX TS-program = xelatex 
\begin{QUESTIONS}
    \begin{QUESTION}
        \begin{ExamInfo}{86}{學測}{單選}{1}
        \end{ExamInfo}
        \begin{ExamAnsRateInfo}{}{}{}{}
        \end{ExamAnsRateInfo}
        \begin{QBODY}坐標平面上兩直線之斜率分別為$\sqrt{3}$ 及$\frac{1}{\sqrt{3}}$,則下列何者為其一交角?
            \begin{QOPS}
                \QOP ${{30}^{{}^\circ }}$	
                \QOP ${{36}^{{}^\circ }}$
                \QOP ${{45}^{{}^\circ }}$	
                \QOP ${{60}^{{}^\circ }}$
                \QOP ${{90}^{{}^\circ }}$
            \end{QOPS}
            
        \end{QBODY}
        \begin{QFROMS}
        \end{QFROMS}
        \begin{QTAGS}\end{QTAGS}
        \begin{QANS}
            (1)
        \end{QANS}
        \begin{QSOLLIST}
        \end{QSOLLIST}
        \begin{QEMPTYSPACE}
        \end{QEMPTYSPACE}
    \end{QUESTION}
    \begin{QUESTION}
        \begin{ExamInfo}{86}{學測}{單選}{2}
        \end{ExamInfo}
        \begin{ExamAnsRateInfo}{}{}{}{}
        \end{ExamAnsRateInfo}
        \begin{QBODY}
            設$P$,$Q$ 為平面$ax+by+cz=5$ 上相異兩點,且 $\lvec{PQ}=\left( {{x}_{0}},{{y}_{0}},{{z}_{0}} \right)$,則 $\lvec{PQ} \cdot \left( a,b,c \right)$ 為
            \begin{QOPS}
                \QOP 不定值,隨$\left( {{x}_{0}},{{y}_{0}},{{z}_{0}} \right)$而改變	
                \QOP $25$
                \QOP $5$	
                \QOP $0$
                \QOP $-1$            
            \end{QOPS}

        \end{QBODY}
        \begin{QFROMS}
        \end{QFROMS}
        \begin{QTAGS}\end{QTAGS}
        \begin{QANS}
            (4)
        \end{QANS}
        \begin{QSOLLIST}
        \end{QSOLLIST}
        \begin{QEMPTYSPACE}
        \end{QEMPTYSPACE}
    \end{QUESTION}
    \begin{QUESTION}
        \begin{ExamInfo}{86}{學測}{單選}{3}
        \end{ExamInfo}
        \begin{ExamAnsRateInfo}{}{}{}{}
        \end{ExamAnsRateInfo}
        \begin{QBODY}            
            設$f\left( x \right)$ 為二次函數,且不等式$f\left( x \right)>0$ 之解為$-2<x<4$,則$f\left( 2x \right)<0$ 之解為
            \begin{QOPS}
                \QOP  $-1<x<2$	
                \QOP  $x<-1$ 或$x>2$
                \QOP  $x<-2$ 或$x>4$	
                \QOP  $-4<x<8$
                \QOP  $x<-4$ 或$x>8$
            \end{QOPS}            
        \end{QBODY}
        \begin{QFROMS}
        \end{QFROMS}
        \begin{QTAGS}\end{QTAGS}
        \begin{QANS}
            (2)
        \end{QANS}
        \begin{QSOLLIST}
        \end{QSOLLIST}
        \begin{QEMPTYSPACE}
        \end{QEMPTYSPACE}
    \end{QUESTION}
    \begin{QUESTION}
        \begin{ExamInfo}{86}{學測}{單選}{4}
        \end{ExamInfo}
        \begin{ExamAnsRateInfo}{}{}{}{}
        \end{ExamAnsRateInfo}
        \begin{QBODY}
            有一個無窮等比級數,其和為$\frac{8}{9}$,第四項為$\frac{3}{32}$。已知公比為一有理數,則當公比以最簡分數表示時,其分母為
            \begin{QOPS}
                \QOP $2$	
                \QOP $3$
                \QOP $4$	
                \QOP $6$
                \QOP $8$
            \end{QOPS}
        \end{QBODY}
        \begin{QFROMS}
        \end{QFROMS}
        \begin{QTAGS}\end{QTAGS}
        \begin{QANS}
            (3)
        \end{QANS}
        \begin{QSOLLIST}
        \end{QSOLLIST}
        \begin{QEMPTYSPACE}
        \end{QEMPTYSPACE}
    \end{QUESTION}
    \begin{QUESTION}
        \begin{ExamInfo}{86}{學測}{單選}{5}
        \end{ExamInfo}
        \begin{ExamAnsRateInfo}{}{}{}{}
        \end{ExamAnsRateInfo}
        \begin{QBODY}
            有一邊長為$3$ 的正六邊形紙板,今在每一個角各剪掉一個小三角形,使其成為正十二邊形之紙板,則此正十二邊形之一邊長為
            \begin{QOPS}
                \QOP $1$	
                \QOP $\frac{3}{2}$
                \QOP $\sqrt{3}$	
                \QOP $\frac{3\sqrt{3}-3}{2}$
                \QOP $6\sqrt{3}-9$
            \end{QOPS}
            
        \end{QBODY}
        \begin{QFROMS}
        \end{QFROMS}
        \begin{QTAGS}\end{QTAGS}
        \begin{QANS}
            (5)
        \end{QANS}
        \begin{QSOLLIST}
        \end{QSOLLIST}
        \begin{QEMPTYSPACE}
        \end{QEMPTYSPACE}
    \end{QUESTION}
    \begin{QUESTION}
        \begin{ExamInfo}{86}{學測}{單選}{6}
        \end{ExamInfo}
        \begin{ExamAnsRateInfo}{}{}{}{}
        \end{ExamAnsRateInfo}
        \begin{QBODY}
            有一正立方體,其邊長都是$1$。如果向量$\lvec{a}$的起點與終點都是此正立方體的頂點,且$\left|\lvec{a}\right|=1$ ,則共有多少個不相等的向量 ?
            \begin{QOPS}
                \QOP $3$	
                \QOP $6$
                \QOP $12$	
                \QOP $24$
                \QOP $28$                
            \end{QOPS}
        \end{QBODY}
        \begin{QFROMS}
        \end{QFROMS}
        \begin{QTAGS}\end{QTAGS}
        \begin{QANS}
            (2)
        \end{QANS}
        \begin{QSOLLIST}
        \end{QSOLLIST}
        \begin{QEMPTYSPACE}
        \end{QEMPTYSPACE}
    \end{QUESTION}
    \begin{QUESTION}
        \begin{ExamInfo}{86}{學測}{單選}{7}
        \end{ExamInfo}
        \begin{ExamAnsRateInfo}{}{}{}{}
        \end{ExamAnsRateInfo}
        \begin{QBODY}
            考慮一正立方體六個面的各中心點,則以其中四個中心點為頂點的正方形一共有幾個?
            \begin{QOPS}
                \QOP $3$	
                \QOP $4$
                \QOP $6$	
                \QOP $8$
                \QOP $12$                
            \end{QOPS}
        \end{QBODY}
        \begin{QFROMS}
        \end{QFROMS}
        \begin{QTAGS}\end{QTAGS}
        \begin{QANS}
            (1)
        \end{QANS}
        \begin{QSOLLIST}
        \end{QSOLLIST}
        \begin{QEMPTYSPACE}
        \end{QEMPTYSPACE}
    \end{QUESTION}
    \begin{QUESTION}
        \begin{ExamInfo}{86}{學測}{單選}{8}
        \end{ExamInfo}
        \begin{ExamAnsRateInfo}{}{}{}{}
        \end{ExamAnsRateInfo}
        \begin{QBODY}
            有一種丟銅板的遊戲,其規則為:出現正面則繼續丟,出現反面就出局。那麼連續丟$5$ 次後還可繼續丟的機率為${{\left( \frac{1}{2} \right)}^{5}}=\frac{1}{32}$。某班有$40$ 名學生,每人各玩一局,設班上至少有一人連續丟$5$ 次後還可繼續丟的機率為$p$,則
            \begin{QOPS}
                \QOP $0.4\le p<0.5$	
                \QOP $0.5\le p<0.6$
                \QOP $0.6\le p<0.7$	
                \QOP $0.7\le p<0.8$
                \QOP $0.8\le p<0.9$
            \end{QOPS}
        \end{QBODY}
        \begin{QFROMS}
        \end{QFROMS}
        \begin{QTAGS}\end{QTAGS}
        \begin{QANS}
            (4)
        \end{QANS}
        \begin{QSOLLIST}
        \end{QSOLLIST}
        \begin{QEMPTYSPACE}
        \end{QEMPTYSPACE}
    \end{QUESTION}
\end{QUESTIONS}\begin{QUESTIONS}
    \begin{QUESTION}
        \begin{ExamInfo}{86}{學測}{多選}{9}
        \end{ExamInfo}
        \begin{ExamAnsRateInfo}{}{}{}{}
        \end{ExamAnsRateInfo}
        \begin{QBODY}
            設$f\left( x \right)=\sum\nolimits_{n=1}^{3}{{{\left( x-n \right)}^{2}}+\sum\nolimits_{n=8}^{10}{{{\left( x-n \right)}^{2}}}}$。若$f\left( x \right)$在$x=a$ 處有最小值,則
            \begin{QOPS}
                \QOP $a$為整數
                \QOP $a<5.9$
                \QOP $a>5.1$
                \QOP $\left| a-4 \right|<0.5$
                \QOP $\left| a-6 \right|<0.5$                
            \end{QOPS}
        \end{QBODY}
        \begin{QFROMS}
        \end{QFROMS}
        \begin{QTAGS}\end{QTAGS}
        \begin{QANS}
            (2)(3)
        \end{QANS}
        \begin{QSOLLIST}
        \end{QSOLLIST}
        \begin{QEMPTYSPACE}
        \end{QEMPTYSPACE}
    \end{QUESTION}
    \begin{QUESTION}
        \begin{ExamInfo}{86}{學測}{多選}{10}
        \end{ExamInfo}
        \begin{ExamAnsRateInfo}{}{}{}{}
        \end{ExamAnsRateInfo}
        \begin{QBODY}
            關於方程式$\left| \frac{3x+y-19}{\sqrt{10}} \right|=\sqrt{{{\left( x+1 \right)}^{2}}+{{\left( y-2 \right)}^{2}}}$所代表的錐線圖形$\Gamma $,下列何者為真?
            \begin{QOPS}
                \QOP $\Gamma $ 為拋物線
                \QOP $\left( 1,-2 \right)$ 為$\Gamma $ 的焦點
                \QOP $3x+y-19=0$ 為$\Gamma $ 的漸近線
                \QOP $x-3y+7=0$ 為$\Gamma $ 的對稱軸
                \QOP $\left( 3,1 \right)$為$\Gamma $ 的頂點
            \end{QOPS}
        \end{QBODY}
        \begin{QFROMS}
        \end{QFROMS}
        \begin{QTAGS}\end{QTAGS}
        \begin{QANS}
            (1)(4)
        \end{QANS}
        \begin{QSOLLIST}
        \end{QSOLLIST}
        \begin{QEMPTYSPACE}
        \end{QEMPTYSPACE}
    \end{QUESTION}
    \begin{QUESTION}
        \begin{ExamInfo}{86}{學測}{多選}{11}
        \end{ExamInfo}
        \begin{ExamAnsRateInfo}{}{}{}{}
        \end{ExamAnsRateInfo}
        \begin{QBODY}
            下列各選項中的曲線$\Gamma $,何者是一個橢圓?
            %TODO:補圖
        \end{QBODY}
        \begin{QFROMS}
        \end{QFROMS}
        \begin{QTAGS}\end{QTAGS}
        \begin{QANS}
            (3)(4)(5)
        \end{QANS}
        \begin{QSOLLIST}
        \end{QSOLLIST}
        \begin{QEMPTYSPACE}
        \end{QEMPTYSPACE}
    \end{QUESTION}
    \begin{QUESTION}
        \begin{ExamInfo}{86}{學測}{多選}{12}
        \end{ExamInfo}
        \begin{ExamAnsRateInfo}{}{}{}{}
        \end{ExamAnsRateInfo}
        \begin{QBODY}
            下圖中,有五組數據,每組各有$A$,$B$,$C$,$D$,$E$,$F$ 等六個資料點。
            
            設各組的相關係數由左至右分別為${{r}_{1}}$,${{r}_{2}}$,${{r}_{3}}$,${{r}_{4}}$,${{r}_{5}}$,則下列關係式何者為真?
            \begin{QOPS}
                \QOP ${{r}_{1}}={{r}_{2}}$
                \QOP ${{r}_{2}}<{{r}_{3}}$
                \QOP ${{r}_{3}}>{{r}_{4}}$
                \QOP ${{r}_{3}}<{{r}_{5}}$
                \QOP ${{r}_{4}}={{r}_{5}}$
            \end{QOPS}
            %TODO:補圖        
        \end{QBODY}
        \begin{QFROMS}
        \end{QFROMS}
        \begin{QTAGS}\end{QTAGS}
        \begin{QANS}
            (1)(2)(5)
        \end{QANS}
        \begin{QSOLLIST}
        \end{QSOLLIST}
        \begin{QEMPTYSPACE}
        \end{QEMPTYSPACE}
    \end{QUESTION}
\end{QUESTIONS}\begin{QUESTIONS}
    \begin{QUESTION}
        \begin{ExamInfo}{86}{學測}{選填}{13}
        \end{ExamInfo}
        \begin{ExamAnsRateInfo}{}{}{}{}
        \end{ExamAnsRateInfo}
        \begin{QBODY}
            設$f\left( x \right)={{x}^{5}}+6{{x}^{4}}-4{{x}^{3}}+25{{x}^{2}}+30x+20$,則$f\left( -7 \right)= \TCNBOX{\TCN}$。
        \end{QBODY}
        \begin{QFROMS}
        \end{QFROMS}
        \begin{QTAGS}\end{QTAGS}
        \begin{QANS}
            $6$
        \end{QANS}
        \begin{QSOLLIST}
        \end{QSOLLIST}
        \begin{QEMPTYSPACE}
        \end{QEMPTYSPACE}
    \end{QUESTION}
    \begin{QUESTION}
        \begin{ExamInfo}{86}{學測}{選填}{14}
        \end{ExamInfo}
        \begin{ExamAnsRateInfo}{}{}{}{}
        \end{ExamAnsRateInfo}
        \begin{QBODY}
            設$\theta $ 為兩平面$2x-y+2z=6$ 與$3x-4z=2$ 的夾角(取銳角),則$\theta $ 最接近的整數度數為$\TCNBOX{\TCN\TCN}$度。
        \end{QBODY}
        \begin{QFROMS}
        \end{QFROMS}
        \begin{QTAGS}\end{QTAGS}
        \begin{QANS}
            $82$
        \end{QANS}
        \begin{QSOLLIST}
        \end{QSOLLIST}
        \begin{QEMPTYSPACE}
        \end{QEMPTYSPACE}
    \end{QUESTION}
    \begin{QUESTION}
        \begin{ExamInfo}{86}{學測}{選填}{15}
        \end{ExamInfo}
        \begin{ExamAnsRateInfo}{}{}{}{}
        \end{ExamAnsRateInfo}
        \begin{QBODY}
            在四邊形$ABCD$ 中,$\angle A={{120}^{{}^\circ }}$,$\overline{AB}=1$,$\overline{AD}=2$,且$\lvec{AC} = 3 \lvec{AB} +2\lvec{AD}$ ,則$\overline{AC}$的長度為$\TCNBOX{\sqrt{TCN\TCN}}$。
        \end{QBODY}
        \begin{QFROMS}
        \end{QFROMS}
        \begin{QTAGS}\end{QTAGS}
        \begin{QANS}
            $\sqrt{13}$
        \end{QANS}
        \begin{QSOLLIST}
        \end{QSOLLIST}
        \begin{QEMPTYSPACE}
        \end{QEMPTYSPACE}
    \end{QUESTION}
    \begin{QUESTION}
        \begin{ExamInfo}{86}{學測}{選填}{16}
        \end{ExamInfo}
        \begin{ExamAnsRateInfo}{}{}{}{}
        \end{ExamAnsRateInfo}
        \begin{QBODY}
            已知三角形由三直線$y=0$,$3x-2y+3=0$,$x+y-4=0$所圍成,則其外接圓之直徑為$\TCNBOX{\sqrt{\TCN\TCN}}$。
        \end{QBODY}
        \begin{QFROMS}
        \end{QFROMS}
        \begin{QTAGS}\end{QTAGS}
        \begin{QANS}
            $\sqrt{26}$
        \end{QANS}
        \begin{QSOLLIST}
        \end{QSOLLIST}
        \begin{QEMPTYSPACE}
        \end{QEMPTYSPACE}
    \end{QUESTION}
    \begin{QUESTION}
        \begin{ExamInfo}{86}{學測}{選填}{17}
        \end{ExamInfo}
        \begin{ExamAnsRateInfo}{}{}{}{}
        \end{ExamAnsRateInfo}
        \begin{QBODY}
            已知圓內接四邊形的各邊長為$\overline{AB}=1$,$\overline{BC}=2$,$\overline{CD}=3$,$\overline{DA}=4$。則對角線$\overline{BD}$ 的長度為$\TCNBOX{\sqrt{\FR{\TCN\TCN}{\TCN}}}$。
        \end{QBODY}
        \begin{QFROMS}
        \end{QFROMS}
        \begin{QTAGS}\end{QTAGS}
        \begin{QANS}
            $\sqrt{\FR{77}{5}}$
        \end{QANS}
        \begin{QSOLLIST}
        \end{QSOLLIST}
        \begin{QEMPTYSPACE}
        \end{QEMPTYSPACE}
    \end{QUESTION}
    \begin{QUESTION}
        \begin{ExamInfo}{86}{學測}{選填}{18}
        \end{ExamInfo}
        \begin{ExamAnsRateInfo}{}{}{}{}
        \end{ExamAnsRateInfo}
        \begin{QBODY}
            將${{3}^{100}}$以科學記號表示:${{3}^{100}}=a\times {{10}^{m}}$,其中$1\le a<10$,$m$ 為整數,則$a$ 的整數部分為$\TCNBOX{\TCN}$。
        \end{QBODY}
        \begin{QFROMS}
        \end{QFROMS}
        \begin{QTAGS}\end{QTAGS}
        \begin{QANS}
            $5$
        \end{QANS}
        \begin{QSOLLIST}
        \end{QSOLLIST}
        \begin{QEMPTYSPACE}
        \end{QEMPTYSPACE}
    \end{QUESTION}
    \begin{QUESTION}
        \begin{ExamInfo}{86}{學測}{選填}{19}
        \end{ExamInfo}
        \begin{ExamAnsRateInfo}{}{}{}{}
        \end{ExamAnsRateInfo}
        \begin{QBODY}
            某人上班有甲、乙兩條路線可供選擇。早上定時從家裡出發,走甲路線有$\frac{1}{10}$ 的機率會遲到,走乙路線則有$\frac{1}{5}$ 的機率會遲到。無論走哪一條路線,只要不遲到,下次就走同一條路線,否則就換另一條路線。假設他第一天走甲路線,則第三天也走甲路線的機率為$\FR{\TCN\TCN}{100}$。
        \end{QBODY}
        \begin{QFROMS}
        \end{QFROMS}
        \begin{QTAGS}\end{QTAGS}
        \begin{QANS}
            $0.83$
        \end{QANS}
        \begin{QSOLLIST}
        \end{QSOLLIST}
        \begin{QEMPTYSPACE}
        \end{QEMPTYSPACE}
    \end{QUESTION}
    \begin{QUESTION}
        \begin{ExamInfo}{86}{學測}{選填}{20}
        \end{ExamInfo}
        \begin{ExamAnsRateInfo}{}{}{}{}
        \end{ExamAnsRateInfo}
        \begin{QBODY}
            有一種遊戲,每次輸贏規則如下:先從$1$ 至$6$ 中選定一個號碼$n$,再擲三粒均勻的骰子。若三粒骰子的點數全都是$n$,則可贏$3$ 元;恰有兩個點數為$n$,則可贏$2$ 元;恰有一個點數為$n$,則可贏$1$ 元;而沒有點數為$n$,則輸$1$ 元。如此,玩一次的期望值(贏為正,輸為負)為$\TCNBOX{\FR{\TCN\TCN\TCN}{216}}$元。
        \end{QBODY}
        \begin{QFROMS}
        \end{QFROMS}
        \begin{QTAGS}\end{QTAGS}
        \begin{QANS}
            $\FR{-17}{216}$
        \end{QANS}
        \begin{QSOLLIST}
        \end{QSOLLIST}
        \begin{QEMPTYSPACE}
        \end{QEMPTYSPACE}
    \end{QUESTION}
\end{QUESTIONS}