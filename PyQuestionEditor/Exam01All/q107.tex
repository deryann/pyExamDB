% !TEX encoding = UTF-8 Unicode
% !TEX TS-program = xelatex 
\begin{QUESTIONS}
    \begin{QUESTION}
        \begin{ExamInfo}{107}{學測}{單選}{1}
        \end{ExamInfo}
        \begin{ExamAnsRateInfo}{}{}{}{}
        \end{ExamAnsRateInfo}
        \begin{QBODY}
            給定相異兩點 $A$、 $B$,試問空間中能使$\triangle PAB$成一正三角形的所有點 $P$ 所成集合
            為下列哪一選項?
            \begin{QOPS}
                \QOP 兩個點
                \QOP 一線段
                \QOP 一直線
                \QOP 一圓
                \QOP 一平面
            \end{QOPS}
        \end{QBODY}
        \begin{QFROMS}
        \end{QFROMS}
        \begin{QTAGS}\end{QTAGS}
        \begin{QANS}
            $(4)$
        \end{QANS}
        \begin{QSOLLIST}
        \end{QSOLLIST}
        \begin{QEMPTYSPACE}
        \end{QEMPTYSPACE}
    \end{QUESTION}
    \begin{QUESTION}
        \begin{ExamInfo}{107}{學測}{單選}{2}
        \end{ExamInfo}
        \begin{ExamAnsRateInfo}{}{}{}{}
        \end{ExamAnsRateInfo}
        \begin{QBODY}
            一份試卷共有 $10$ 題單選題, 每題有$5$ 個選項, 其中只有一個選項是正確答案。
            假設小明以隨機猜答的方式回答此試卷,且各題猜答方式互不影響。試估計小
            明全部答對的機率最接近下列哪一選項?
            \begin{QOPS}
                \QOP $10^{-5}$
                \QOP $10^{-6}$
                \QOP $10^{-7}$
                \QOP $10^{-8}$
                \QOP $10^{-9}$
            \end{QOPS}
        \end{QBODY}
        \begin{QFROMS}
        \end{QFROMS}
        \begin{QTAGS}\end{QTAGS}
        \begin{QANS}
            $(3)$
        \end{QANS}
        \begin{QSOLLIST}
        \end{QSOLLIST}
        \begin{QEMPTYSPACE}
        \end{QEMPTYSPACE}
    \end{QUESTION}
    \begin{QUESTION}
        \begin{ExamInfo}{107}{學測}{單選}{3}
        \end{ExamInfo}
        \begin{ExamAnsRateInfo}{}{}{}{}
        \end{ExamAnsRateInfo}
        \begin{QBODY}
            某公司規定員工可在一星期( 七天)當中選擇兩天休假。若甲、乙兩人隨機選
            擇休假日且兩人的選擇互不相關,試問一星期當中發生兩人在同一天休假的機
            率為何?
            \begin{QOPS}
                \QOP $\FR{1}{3}$
                \QOP $\FR{8}{21}$
                \QOP $\FR{3}{7}$
                \QOP $\FR{10}{21}$
                \QOP $\FR{11}{21}$
            \end{QOPS}
        \end{QBODY}
        \begin{QFROMS}
        \end{QFROMS}
        \begin{QTAGS}\end{QTAGS}
        \begin{QANS}
            $(5)$
        \end{QANS}
        \begin{QSOLLIST}
        \end{QSOLLIST}
        \begin{QEMPTYSPACE}
        \end{QEMPTYSPACE}
    \end{QUESTION}
    \begin{QUESTION}
        \begin{ExamInfo}{107}{學測}{單選}{4}
        \end{ExamInfo}
        \begin{ExamAnsRateInfo}{}{}{}{}
        \end{ExamAnsRateInfo}
        \begin{QBODY}
            試問有多少個整數 $x$ 滿足 $10^{9} < 2^{x} < 9^{10}$?
            \begin{QOPS}
                \QOP $1$ 個
                \QOP $2$ 個
                \QOP $3$ 個
                \QOP $4$ 個
                \QOP $0$ 個
            \end{QOPS}
        \end{QBODY}
        \begin{QFROMS}
        \end{QFROMS}
        \begin{QTAGS}\end{QTAGS}
        \begin{QANS}
            $(2)$
        \end{QANS}
        \begin{QSOLLIST}
        \end{QSOLLIST}
        \begin{QEMPTYSPACE}
        \end{QEMPTYSPACE}
    \end{QUESTION}
    \begin{QUESTION}
        \begin{ExamInfo}{107}{學測}{單選}{5}
        \end{ExamInfo}
        \begin{ExamAnsRateInfo}{}{}{}{}
        \end{ExamAnsRateInfo}
        \begin{QBODY}
            試 問 共 有 幾 個 角 度 $\theta$ 滿足 $0^\circ < \theta < 180^\circ$ ,且 $\cos(3\theta-60^\circ), \cos(3\theta), \cos(3\theta +60^\circ) $ 依序排成一等差數列?
            \begin{QOPS}
                \QOP $1$ 個
                \QOP $2$ 個
                \QOP $3$ 個
                \QOP $4$ 個
                \QOP $5$ 個
            \end{QOPS}            
        \end{QBODY}
        \begin{QFROMS}
        \end{QFROMS}
        \begin{QTAGS}\end{QTAGS}
        \begin{QANS}
            $(3)$
        \end{QANS}
        \begin{QSOLLIST}
        \end{QSOLLIST}
        \begin{QEMPTYSPACE}
        \end{QEMPTYSPACE}
    \end{QUESTION}
    \begin{QUESTION}
        \begin{ExamInfo}{107}{學測}{單選}{6}
        \end{ExamInfo}
        \begin{ExamAnsRateInfo}{}{}{}{}
        \end{ExamAnsRateInfo}
        \begin{QBODY}
            某貨品為避免因成本變動而造成售價波動太過劇烈,當週售價相對於前一週售
            價的漲跌幅定為當週成本相對於前一週成本的漲跌幅的一半。例如下表中第二
            週成本上漲100\%,所以第二週售價上漲50\%。依此定價方式以及下表的資訊,
            試選出正確的選項。
            【註:成本漲跌幅 $=\FR{\text{當週成本}-\text{前週成本}}{\text{前週成本}}$,售價漲跌幅 $ = \FR{\text{當週售價}-\text{前週售價}}{\text{前週售價}} $。】
            \begin{tabular}{|c|c|c|c|c|}
                \hline 
                & 第一週 & 第二週 & 第三週 & 第四周 \\ 
                \hline 
                成本 & 50 & 100 & 50 & 90 \\ 
                \hline 
                售價 & 120 & 180 & x & y \\ 
                \hline 
            \end{tabular} 
            \begin{QOPS}
                \QOP $120 = x < y < 180$
                \QOP $120 < x < y < 180$
                \QOP $x < 120 < y < 180$
                \QOP $120 = x < 180 < y$
                \QOP $120 < x < 180 < y$
            \end{QOPS}
        \end{QBODY}
        \begin{QFROMS}
        \end{QFROMS}
        \begin{QTAGS}\end{QTAGS}
        \begin{QANS}
            $(5)$
        \end{QANS}
        \begin{QSOLLIST}
        \end{QSOLLIST}
        \begin{QEMPTYSPACE}
        \end{QEMPTYSPACE}
    \end{QUESTION}
    \begin{QUESTION}
        \begin{ExamInfo}{107}{學測}{單選}{7}
        \end{ExamInfo}
        \begin{ExamAnsRateInfo}{}{}{}{}
        \end{ExamAnsRateInfo}
        \begin{QBODY}
            $\triangle ABC$ 內接於圓心為 $O$ 之單位圓。若$\lvec{OA} +\lvec{OB}+\sqrt{3}\lvec{OC} = \lvec{0}$,則 $\angle BAC$ 之度數為何?
            \begin{QOPS}
                \QOP $30^\circ$
                \QOP $45^\circ$
                \QOP $60^\circ$ 
                \QOP $75^\circ$
                \QOP $90^\circ$
            \end{QOPS}
        \end{QBODY}
        \begin{QFROMS}
        \end{QFROMS}
        \begin{QTAGS}\end{QTAGS}
        \begin{QANS}
            $(4)$
        \end{QANS}
        \begin{QSOLLIST}
        \end{QSOLLIST}
        \begin{QEMPTYSPACE}
        \end{QEMPTYSPACE}
    \end{QUESTION}
\end{QUESTIONS}\begin{QUESTIONS}
    \begin{QUESTION}
        \begin{ExamInfo}{107}{學測}{多選}{8}
        \end{ExamInfo}
        \begin{ExamAnsRateInfo}{}{}{}{}
        \end{ExamAnsRateInfo}
        \begin{QBODY}
            某年學科能力測驗小華的成績為: 國文11級分、英文12 級分、數學9 級分、自
            然9 級分、社會12 級分。他考慮申請一些校系,表1 為大考中心公布的學測各
            科成績標準; 表2 是他最有興趣的五個校系規定的申請檢定標準, 依規定申
            請者需通過該校系所有檢定標準才會被列入篩選。例如甲校系規定國文成績
            須達均標、英文須達前標、且社會須達均標;丙校系則規定英文成績須達均標、
            且數學或自然至少有一科達前標。表2 空白者表示該校系對該科成績未規定
            檢定標準。
            
            表 1 學測各科成績標準
            \begin{tabular}{|c|c|c|c|c|c|}
            	\hline
            	     & 頂 標 & 前 標 & 均 標 & 後 標 & 底 標 \\ \hline
            	社 會 & 13  & 12  & 10  &  8  &  7  \\ \hline
            	國 文 & 13  & 12  & 10  &  9  &  7  \\ \hline
            	英 文 & 14  & 12  &  9  &  6  &  4  \\ \hline
            	數 學 & 12  & 10  &  7  &  4  &  3  \\ \hline
            	自 然 & 13  & 11  &  9  &  6  &  5  \\ \hline
            \end{tabular} 
    
            表 2 校系篩選規定
%            \begin{tabular}{|c|c|c|c|c|c|}
%                \hline 
%                &  國文 & 英文  & 數學  & 自然  & 社會 \\ 
%                \hline 
%                甲 校 系& &  &  &  &  \\                 \hline 
%                乙 校 系& &  &  &  &  \\                 \hline 
%                丙 校 系& &  &  &  &  \\                 \hline 
%                丁 校 系& &  &  &  &  \\                 \hline 
%                戊 校 系& &  &  &  &  \\                 \hline 
%            \end{tabular} 
            %TODO 表二
            根 據 以 上 資訊, 試問小華可以考慮申請哪些校系( 會被列入篩選) ?
            \begin{QOPS}
                \QOP 甲校系 
                \QOP 乙校系 
                \QOP 丙校系 
                \QOP 丁校系
                \QOP 戊校系
            \end{QOPS}
            
        \end{QBODY}
        \begin{QFROMS}
        \end{QFROMS}
        \begin{QTAGS}\end{QTAGS}
        \begin{QANS}
            $(1)(4)$
        \end{QANS}
        \begin{QSOLLIST}
        \end{QSOLLIST}
        \begin{QEMPTYSPACE}
        \end{QEMPTYSPACE}
    \end{QUESTION}
    \begin{QUESTION}
        \begin{ExamInfo}{107}{學測}{多選}{9}
        \end{ExamInfo}
        \begin{ExamAnsRateInfo}{}{}{}{}
        \end{ExamAnsRateInfo}
        \begin{QBODY}
            已知多項式 $f(x)$除以 $x^2-1$ 之餘式為 $2x+1$。試選出正確的選項。
            \begin{QOPS}
                \QOP $f(0)=1$
                \QOP $f(1)=3$
                \QOP $f(x)$ 可能為一次式
                \QOP $f(x)$ 可能為 $4x^4+2x^2-3$
                \QOP $f(x)$ 可能為 $4x^4+2x^3-3$
            \end{QOPS}
        \end{QBODY}
        \begin{QFROMS}
        \end{QFROMS}
        \begin{QTAGS}\end{QTAGS}
        \begin{QANS}
            $(2)(3)(5)$
        \end{QANS}
        \begin{QSOLLIST}
        \end{QSOLLIST}
        \begin{QEMPTYSPACE}
        \end{QEMPTYSPACE}
    \end{QUESTION}
    \begin{QUESTION}
        \begin{ExamInfo}{107}{學測}{多選}{10}
        \end{ExamInfo}
        \begin{ExamAnsRateInfo}{}{}{}{}
        \end{ExamAnsRateInfo}
        \begin{QBODY}
            已知坐標平面上 $\triangle ABC$ ,其中 $\lvec{AB} = (-4,3)$ ,且 $\lvec{AC} = \left(\FR{2}{5}, \FR{4}{5} \right)$ 。試選出正確的選項。
            \begin{QOPS}
                \QOP $\overline{BC}=5$ 
                \QOP $\triangle ABC$ 是直角三角形
                \QOP $\triangle ABC$ 的面積是 $\FR{11}{5}$
                \QOP $\sin B > \sin C$
                \QOP $\cos A > \cos B$
            \end{QOPS}
        \end{QBODY}
        \begin{QFROMS}
        \end{QFROMS}
        \begin{QTAGS}\end{QTAGS}
        \begin{QANS}
            $(2)(3)$
        \end{QANS}
        \begin{QSOLLIST}
        \end{QSOLLIST}
        \begin{QEMPTYSPACE}
        \end{QEMPTYSPACE}
    \end{QUESTION}
    \begin{QUESTION}
        \begin{ExamInfo}{107}{學測}{多選}{11}
        \end{ExamInfo}
        \begin{ExamAnsRateInfo}{}{}{}{}
        \end{ExamAnsRateInfo}
        \begin{QBODY}
            坐標空間中,設直線 $L: \FR{x-1}{2}=\FR{y-2}{-3} =\FR{z}{-1} $ , 平面$E_1:2x-3y-z=0$ ,平面 $E_2:x+y-z=0$
            。試選出正確的選項。
            \begin{QOPS}
                \QOP 點 $(3,0,-1)$ 在直線 $L$ 上
                \QOP 點 $(1,2,3)$ 在平面 $E_1$ 上
                \QOP 直線$L$ 與平面 $E_1$ 垂直
                \QOP 直線$L$ 在平面 $E_2$ 上
                \QOP 平面 $E_1$ 與 $E_2$ 交於一直線
            \end{QOPS}
        \end{QBODY}
        \begin{QFROMS}
        \end{QFROMS}
        \begin{QTAGS}\end{QTAGS}
        \begin{QANS}
            $(3)(5)$
        \end{QANS}
        \begin{QSOLLIST}
        \end{QSOLLIST}
        \begin{QEMPTYSPACE}
        \end{QEMPTYSPACE}
    \end{QUESTION}
    \begin{QUESTION}
        \begin{ExamInfo}{107}{學測}{多選}{12}
        \end{ExamInfo}
        \begin{ExamAnsRateInfo}{}{}{}{}
        \end{ExamAnsRateInfo}
        \begin{QBODY}
            試問下列哪些選項中的二次曲線, 其焦點( 之一) 是拋物線 $y^2=2x$的焦點?
            \begin{QOPS}
                \QOP $y = \left(x-\FR{1}{2}\right)^2 - \FR{1}{4}$
                \QOP $\FR{x^2}{4} + \FR{y^2}{3} = 1 $
                \QOP $x^2 + \FR{4y^2}{3} = 1 $
                \QOP $8x^2 - 8y^2 = 1 $
                \QOP $4x^2 - 4y^2 = 1 $
            \end{QOPS}
        \end{QBODY}
        \begin{QFROMS}
        \end{QFROMS}
        \begin{QTAGS}\end{QTAGS}
        \begin{QANS}
            $(1)(3)(4)$
        \end{QANS}
        \begin{QSOLLIST}
        \end{QSOLLIST}
        \begin{QEMPTYSPACE}
        \end{QEMPTYSPACE}
    \end{QUESTION}
\end{QUESTIONS}\begin{QUESTIONS}
    \begin{QUESTION}
        \begin{ExamInfo}{107}{學測}{選填}{A}
        \end{ExamInfo}
        \begin{ExamAnsRateInfo}{}{}{}{}
        \end{ExamAnsRateInfo}
        \begin{QBODY}
            已知坐標平面上三點 $(3, \log 3)$、$(6,\log 6)$與 $(12, y)$ 在同一直線上,則 $y = \log \TCNBOX{\TCN\TCN}$。
        \end{QBODY}
        \begin{QFROMS}
        \end{QFROMS}
        \begin{QTAGS}\end{QTAGS}
        \begin{QANS}
            $24$
        \end{QANS}
        \begin{QSOLLIST}
        \end{QSOLLIST}
        \begin{QEMPTYSPACE}
        \end{QEMPTYSPACE}
    \end{QUESTION}
    \begin{QUESTION}
        \begin{ExamInfo}{107}{學測}{選填}{B}
        \end{ExamInfo}
        \begin{ExamAnsRateInfo}{}{}{}{}
        \end{ExamAnsRateInfo}
        \begin{QBODY}
            如右圖所示( 只是示意圖),將梯子$\overline{AB}$靠在與地面垂
            直的牆 $AC$上,測得與水平地面的夾角$\angle ABC$ 為 $60^\circ$。將
            在地面上的底 $B$ 沿著地面向外拉 $51$ 公分到點 $F$ ( 即
            $\overline{FB} = 51$ 公分),此時梯子 $\overline{EF}$ 與地面的夾角 $\angle EFC$ 之正弦
            值為 $\sin \angle EFC = 0.6$,則梯子長 $\overline{AB}=\TCNBOX{\TCN\TCN\TCN}$ 公分。
            %TODO:補圖
        \end{QBODY}
        \begin{QFROMS}
        \end{QFROMS}
        \begin{QTAGS}\end{QTAGS}
        \begin{QANS}
            $170$
        \end{QANS}
        \begin{QSOLLIST}
        \end{QSOLLIST}
        \begin{QEMPTYSPACE}
        \end{QEMPTYSPACE}
    \end{QUESTION}
    \begin{QUESTION}
        \begin{ExamInfo}{107}{學測}{選填}{C}
        \end{ExamInfo}
        \begin{ExamAnsRateInfo}{}{}{}{}
        \end{ExamAnsRateInfo}
        \begin{QBODY}
            平面上兩點 $A$、 $B$之距離為 5,以 $A$為圓心作一半徑為 $r$( $0 < r < 5$ )的圓 $\Gamma$,過
            $B$作圓$\Gamma$的切線,切點(之一)為 $P$。當 $r$ 變動時,$\triangle PAB$ 的面積最大可能值為 $\TCNBOX{\FR{\TCN\TCN}{\TCN}}$ 。( 化成最簡分數)
        \end{QBODY}
        \begin{QFROMS}
        \end{QFROMS}
        \begin{QTAGS}\end{QTAGS}
        \begin{QANS}
            $\FR{25}{4}$
        \end{QANS}
        \begin{QSOLLIST}
        \end{QSOLLIST}
        \begin{QEMPTYSPACE}
        \end{QEMPTYSPACE}
    \end{QUESTION}
    \begin{QUESTION}
        \begin{ExamInfo}{107}{學測}{選填}{D}
        \end{ExamInfo}
        \begin{ExamAnsRateInfo}{}{}{}{}
        \end{ExamAnsRateInfo}
        \begin{QBODY}
            坐標平面上,圓$\Gamma$完全落在四個不等式::$x - y \le 4$、$x + y \le 18$、$x - y \ge -2$、$x + y \ge -24$
            所圍成的區域內。則 $\Gamma$ 最大可能面積為 $\TCNBOX{\FR{\TCN}{\TCN}} \pi$ 。( 化成最簡分數)
        \end{QBODY}
        \begin{QFROMS}
        \end{QFROMS}
        \begin{QTAGS}\end{QTAGS}
        \begin{QANS}
            $\FR{9}{2}$
        \end{QANS}
        \begin{QSOLLIST}
        \end{QSOLLIST}
        \begin{QEMPTYSPACE}
        \end{QEMPTYSPACE}
    \end{QUESTION}
    \begin{QUESTION}
        \begin{ExamInfo}{107}{學測}{選填}{E}
        \end{ExamInfo}
        \begin{ExamAnsRateInfo}{}{}{}{}
        \end{ExamAnsRateInfo}
        \begin{QBODY}
                        坐 標 平 面 上 , 若 拋 物 線 $ y = x^2 +2x-3$的 頂 點 為 $C$ , 與 $x$ 軸 的 交 點 為 $A$、 $B$, 則
            $\cos \angle ACB = \TCNBOX{\FR{\TCN}{\TCN}} $。( 化成最簡分數)
        \end{QBODY}
        \begin{QFROMS}
        \end{QFROMS}
        \begin{QTAGS}\end{QTAGS}
        \begin{QANS}
            $\FR{3}{5}$
        \end{QANS}
        \begin{QSOLLIST}
        \end{QSOLLIST}
        \begin{QEMPTYSPACE}
        \end{QEMPTYSPACE}
    \end{QUESTION}
    \begin{QUESTION}
        \begin{ExamInfo}{107}{學測}{選填}{F}
        \end{ExamInfo}
        \begin{ExamAnsRateInfo}{}{}{}{}
        \end{ExamAnsRateInfo}
        \begin{QBODY}
             設 $a,b,c,d,e, x, y, z$ 皆為實數,考慮矩陣相乘:
            $\begin{bmatrix} 
            a & b \\
            c & d \\
            1 & 2 
            \end{bmatrix} 
            \begin{bmatrix} 
            -3 & 5 & 7 \\
            -4 & 6 & e 
            \end{bmatrix}  =
            \begin{bmatrix} 
            3 & x & 7 \\
            0 & y & 7 \\
            -11 & z & 23 
            \end{bmatrix}                         
            $ 則 $y = \TCNBOX{\FR{\TCN}{\TCN}}$ 。( 化成最簡分數)
        \end{QBODY}
        \begin{QFROMS}
        \end{QFROMS}
        \begin{QTAGS}\end{QTAGS}
        \begin{QANS}
            $\FR{7}{2}$
        \end{QANS}
        \begin{QSOLLIST}
        \end{QSOLLIST}
        \begin{QEMPTYSPACE}
        \end{QEMPTYSPACE}
    \end{QUESTION}
    \begin{QUESTION}
        \begin{ExamInfo}{107}{學測}{選填}{G}
        \end{ExamInfo}
        \begin{ExamAnsRateInfo}{}{}{}{}
        \end{ExamAnsRateInfo}
        \begin{QBODY}
            設$D$為 $\triangle ABC$ 中 $\overline{BC}BC$ 邊上的一點,已知 $\angle ABC = 75^\circ$、 $\angle ACB = 45^\circ$、 $\angle ADB = 60^\circ$。
            若 $\lvec{AD} = s \lvec{AB} + t \lvec{AC}$ 則 $s = \TCNBOX{\FR{\TCN}{\TCN}}, t = \TCNBOX{\FR{\TCN}{\TCN}}$。( 化成最簡分數)
        \end{QBODY}
        \begin{QFROMS}
        \end{QFROMS}
        \begin{QTAGS}\end{QTAGS}
        \begin{QANS}
            $\FR{1}{3}, \FR{2}{3}$
        \end{QANS}
        \begin{QSOLLIST}
        \end{QSOLLIST}
        \begin{QEMPTYSPACE}
        \end{QEMPTYSPACE}
    \end{QUESTION}
    \begin{QUESTION}
        \begin{ExamInfo}{107}{學測}{選填}{H}
        \end{ExamInfo}
        \begin{ExamAnsRateInfo}{}{}{}{}
        \end{ExamAnsRateInfo}
        \begin{QBODY}
            將一塊邊長 $\overline{AB} = 15$公分、 $\overline{BC} = 20$公分的長方形鐵片 $ABCD$ 沿對角線 $\overline{BD}$對摺後豎立,
            使得平面 $ABD$ 與平面 $CBD$ 垂直,則$A$、$C$ 兩點(在空間)的距離$\overline{AC}=\TCNBOX{\sqrt{\TCN\TCN\TCN}}$
            公 分 。( 化成最簡根式)
        \end{QBODY}
        \begin{QFROMS}
        \end{QFROMS}
        \begin{QTAGS}\end{QTAGS}
        \begin{QANS}
            $\sqrt{337}$
        \end{QANS}
        \begin{QSOLLIST}
        \end{QSOLLIST}
        \begin{QEMPTYSPACE}
        \end{QEMPTYSPACE}
    \end{QUESTION}
\end{QUESTIONS}