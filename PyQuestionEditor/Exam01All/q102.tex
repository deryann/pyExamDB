    \begin{QUESTION}
        \begin{ExamInfo}{102}{學測}{單選}{1}
        \end{ExamInfo}
        \begin{ExamAnsRateInfo}{70}{88}{73}{49}
        \end{ExamAnsRateInfo}
        \begin{QBODY}
            學校規定上學期成績需同時滿足以下兩項要求,才有資格參選模範生。\\
				一、國文成績或英文成績70分(含)以上;\\
				二、數學成績及格。\\
				已知小文上學期國文65分而且他不符合參選模範生資格。請問下列哪一個選項的推論是正確的?
				\begin{QOPS}
					\QOP 小文的英文成績未達70分
					\QOP 小文的數學成績不及格
					\QOP 小文的英文成績70分以上但數學成績不及格
					\QOP 小文的英文成績未達70分且數學成績不及格
					\QOP 小文的英文成績未達70分或數學成績不及格
				\end{QOPS}
        \end{QBODY}
        \begin{QFROMS}
        \end{QFROMS}
        \begin{QTAGS}\QTAG{邏輯}\QTAG{B2C2-1簡單的邏輯與集合}\QTAG{B2C2排列組合}\end{QTAGS}
        \begin{QANS}
            (5)
        \end{QANS}
        \begin{QSOLLIST}
        \end{QSOLLIST}
        \begin{QEMPTYSPACE}
        \end{QEMPTYSPACE}
    \end{QUESTION}
    \begin{QUESTION}
        \begin{ExamInfo}{102}{學測}{單選}{2}
        \end{ExamInfo}
        \begin{ExamAnsRateInfo}{66}{90}{72}{36}
        \end{ExamAnsRateInfo}
        \begin{QBODY}
            令$a={{2.6}^{10}}-{{2.6}^{9}}$,$b={{2.6}^{11}}-{{2.6}^{10}}$,$c=\FR{{{2.6}^{11}}-{{2.6}^{9}}}{2}$。請選出正確的大小關係。
			\begin{QOPS}
				\QOP $a>b>c$
				\QOP $a>c>b$
				\QOP $b>a>c$
				\QOP $b>c>a$
				\QOP $c>b>a$
			\end{QOPS}
        \end{QBODY}
        \begin{QFROMS}
        \end{QFROMS}
        \begin{QTAGS}\QTAG{圖形}\QTAG{B1C3-2指數函數}\QTAG{B1C3指對數函數}\end{QTAGS}
        \begin{QANS}
            (4)
        \end{QANS}
        \begin{QSOLLIST}
        \end{QSOLLIST}
        \begin{QEMPTYSPACE}
        \end{QEMPTYSPACE}
    \end{QUESTION}
    \begin{QUESTION}
        \begin{ExamInfo}{102}{學測}{單選}{3}
        \end{ExamInfo}
        \begin{ExamAnsRateInfo}{35}{67}{26}{12}
        \end{ExamAnsRateInfo}
        \begin{QBODY}
            袋子裡有3顆白球,$\,2\,$顆黑球。由甲、乙、丙三人依序各抽取1顆球,抽取後不放回。若每顆球被取出的機會相等,請問在甲和乙抽到相同顏色球的條件下,丙抽到白球之條件機率為何?
			\begin{QOPS}
				\QOP $\FR{1}{3}$	
				\QOP $\FR{5}{12}$	
				\QOP $\FR{1}{2}$
				\QOP $\FR{3}{5}$	
				\QOP $\FR{2}{3}$
			\end{QOPS}
        \end{QBODY}
        \begin{QFROMS}
        \end{QFROMS}
        \begin{QTAGS}\QTAG{B2C3機率}\QTAG{B2C3-3條件機率與貝氏定理}\end{QTAGS}
        \begin{QANS}
            (3)
        \end{QANS}
        \begin{QSOLLIST}
        \end{QSOLLIST}
        \begin{QEMPTYSPACE}
        \end{QEMPTYSPACE}
    \end{QUESTION}
    \begin{QUESTION}
        \begin{ExamInfo}{102}{學測}{單選}{4}
        \end{ExamInfo}
        \begin{ExamAnsRateInfo}{30}{46}{25}{19}
        \end{ExamAnsRateInfo}
        \begin{QBODY}
            已知以下各選項資料的迴歸直線(最適合直線)皆相同且皆為負相關,請選出相關係數最小的選項。
			\begin{QOPS}
				\QOP $\begin{matrix}
				   x & 2 & 3 & 5  \\
				   y & 1 & 13 & 1  \\
				\end{matrix}$	
				\QOP $\begin{matrix}
				   x & 2 & 3 & 5  \\
				   y & 3 & 10 & 2  \\
				\end{matrix}$ 	
				\QOP $\begin{matrix}
				   x & 2 & 3 & 5  \\
				   y & 5 & 7 & 3  \\
				\end{matrix}$ 
				\QOP $\begin{matrix}
				   x & 2 & 3 & 5  \\
				   y & 9 & 1 & 5  \\
				\end{matrix}$	
				\QOP $\begin{matrix}
				   x & 2 & 3 & 5  \\
				   y & 7 & 4 & 4  \\
				\end{matrix}$
			\end{QOPS}
        \end{QBODY}
        \begin{QFROMS}
        \end{QFROMS}
        \begin{QTAGS}\QTAG{B2C4-2二維數據分析}\QTAG{B2C4數據分析}\QTAG{相關係數}\QTAG{迴歸直線}\end{QTAGS}
        \begin{QANS}
            (5)
        \end{QANS}
        \begin{QSOLLIST}
        \end{QSOLLIST}
        \begin{QEMPTYSPACE}
        \end{QEMPTYSPACE}
    \end{QUESTION}
    \begin{QUESTION}
        \begin{ExamInfo}{102}{學測}{單選}{5}
        \end{ExamInfo}
        \begin{ExamAnsRateInfo}{50}{75}{50}{25}
        \end{ExamAnsRateInfo}
        \begin{QBODY}
            將24顆雞蛋分裝到紅、黃、綠的三個籃子。每個籃子都要有雞蛋,且黃、綠兩
			個籃子裡都裝奇數顆。請選出分裝的方法數。
			\begin{QOPS}
					\QOP $55  $
					\QOP $66  $
					\QOP $132 $
					\QOP $198 $
					\QOP $253 $
			\end{QOPS}
        \end{QBODY}
        \begin{QFROMS}
        \end{QFROMS}
        \begin{QTAGS}\QTAG{重複組合}\QTAG{B2C2排列組合}\QTAG{B2C2-3組合}\end{QTAGS}
        \begin{QANS}
            (2)
        \end{QANS}
        \begin{QSOLLIST}
        \end{QSOLLIST}
        \begin{QEMPTYSPACE}
        \end{QEMPTYSPACE}
    \end{QUESTION}
    \begin{QUESTION}
        \begin{ExamInfo}{102}{學測}{單選}{6}
        \end{ExamInfo}
        \begin{ExamAnsRateInfo}{39}{71}{31}{15}
        \end{ExamAnsRateInfo}
        \begin{QBODY}
            莎韻觀測遠方等速率垂直上升的熱氣球。在上午$10:00$熱氣球的仰角為$30{}^\circ $,到上午$10:10$仰角變成$34{}^\circ $。請利用下表判斷到上午$10:30$時,熱氣球的仰角最接近下列哪一個度數?
			\begin{tabular}{|c|c|c|c|c|c|c|c|}
			\hline
			$\theta $		& $30{}^\circ $ & $34{}^\circ $ & 	$39{}^\circ $	&  $40{}^\circ $   & $41{}^\circ $ & $42{}^\circ $	& $43{}^\circ $  \\\hline
			$\sin \theta $	& 0.500	        &    0.559	    &       0.629	    &     0.643	       &   0.656	   &       0.669	&      0.682     \\\hline
			$\cos \theta $	& 0.866	        &    0.829	    &       0.777	    &     0.766	       &   0.755	   &       0.743	&      0.731     \\\hline
			$\tan \theta $	& 0.577	        &    0.675	    &       0.810	    &     0.839	       &   0.869	   &       0.900	&      0.933     \\\hline
			\end{tabular}
			\begin{QOPS}
				\QOP $39{}^\circ $
				\QOP $40{}^\circ $
				\QOP $41{}^\circ $
				\QOP $42{}^\circ $
				\QOP $43{}^\circ $
			\end{QOPS}
        \end{QBODY}
        \begin{QFROMS}
        \end{QFROMS}
        \begin{QTAGS}\QTAG{B3C1-5三角測量}\QTAG{B3C1三角}\end{QTAGS}
        \begin{QANS}
            (3)
        \end{QANS}
        \begin{QSOLLIST}
        \end{QSOLLIST}
        \begin{QEMPTYSPACE}
        \end{QEMPTYSPACE}
    \end{QUESTION}
    \begin{QUESTION}
        \begin{ExamInfo}{102}{學測}{多選}{7}
        \end{ExamInfo}
        \begin{ExamAnsRateInfo}{60}{85}{67}{28}
        \end{ExamAnsRateInfo}
        \begin{QBODY}
            設$n$為正整數,符號${{\left[ \begin{matrix}
			   1 & 1  \\
			   0 & 2  \\
			\end{matrix} \right]}^{n}}$代表矩陣$\left[ \begin{matrix}
			   1 & 1  \\
			   0 & 2  \\
			\end{matrix} \right]$自乘$n$次。令${{\left[ \begin{matrix}
			   1 & 1  \\
			   0 & 2  \\
			\end{matrix} \right]}^{n}}=\left[ \begin{matrix}
			   {{a}_{n}} & {{b}_{n}}  \\
			   {{c}_{n}} & {{d}_{n}}  \\
			\end{matrix} \right]$,請選出正確的選項。
			\begin{QOPS}
				\QOP ${{a}_{2}}=1$
				\QOP ${{a}_{1}},{{a}_{2}},{{a}_{3}}$為等比數列
				\QOP ${{d}_{1}},{{d}_{2}},{{d}_{3}}$為等比數列
				\QOP ${{b}_{1}},{{b}_{2}},{{b}_{3}}$為等差數列
				\QOP ${{c}_{1}},{{c}_{2}},{{c}_{3}}$為等差數列
			\end{QOPS}
        \end{QBODY}
        \begin{QFROMS}
        \end{QFROMS}
        \begin{QTAGS}\QTAG{矩陣乘法}\QTAG{B4C3矩陣}\QTAG{等差數列}\QTAG{B2C1數列級數}\QTAG{等比數列}\QTAG{B2C1-1數列}\QTAG{B4C3-2矩陣的運算}\end{QTAGS}
        \begin{QANS}
            (1)(2)(3)(5)
        \end{QANS}
        \begin{QSOLLIST}
        \end{QSOLLIST}
        \begin{QEMPTYSPACE}
        \end{QEMPTYSPACE}
    \end{QUESTION}
    \begin{QUESTION}
        \begin{ExamInfo}{102}{學測}{多選}{8}
        \end{ExamInfo}
        \begin{ExamAnsRateInfo}{47}{68}{50}{23}
        \end{ExamAnsRateInfo}
        \begin{QBODY}
            設$a>1>b>0$,關於下列不等式,請選出正確的選項。
			\begin{QOPS}
				\QOP ${{(-a)}^{7}}>{{(-a)}^{9}}$
				\QOP ${{b}^{-9}}>{{b}^{-7}}$
				\QOP ${{\log }_{10}}\FR{1}{a}>{{\log }_{10}}\FR{1}{b}$
				\QOP ${{\log }_{a}}1>{{\log }_{b}}1$
				\QOP ${{\log }_{a}}b\ge {{\log }_{b}}a$
			\end{QOPS}
        \end{QBODY}
        \begin{QFROMS}
        \end{QFROMS}
        \begin{QTAGS}\QTAG{B1C3-4對數函數}\QTAG{B1C3指對數函數}\QTAG{不等式}\end{QTAGS}
        \begin{QANS}
            (1)(2)
        \end{QANS}
        \begin{QSOLLIST}
        \end{QSOLLIST}
        \begin{QEMPTYSPACE}
        \end{QEMPTYSPACE}
    \end{QUESTION}
    \begin{QUESTION}
        \begin{ExamInfo}{102}{學測}{多選}{9}
        \end{ExamInfo}
        \begin{ExamAnsRateInfo}{43}{64}{40}{25}
        \end{ExamAnsRateInfo}
        \begin{QBODY}
            設$a<b<c$。已知實係數多項式函數$y=f(x)$的圖形為一開口向上的拋物線,且與$x$軸交於$(a,0)$、$(b,0)$兩點;實係數多項式函數$y=g(x)$的圖形亦為一開口向上的拋物線,且跟$x$軸相交於$(b,0)$、$(c,0)$兩點。請選出$y=f(x)+g(x)$的圖形可能的選項。
			\begin{QOPS}
				\QOP 水平直線
				\QOP 和$x$軸僅交於一點的直線
				\QOP 和$x$軸無交點的拋物線
				\QOP 和$x$軸僅交於一點的拋物線
				\QOP 和$x$軸交於兩點的拋物線
			\end{QOPS}
        \end{QBODY}
        \begin{QFROMS}
        \end{QFROMS}
        \begin{QTAGS}\QTAG{B1C2多項式函數}\QTAG{圖形}\QTAG{二次多項式}\QTAG{B1C2-1簡單的多項式及圖形}\end{QTAGS}
        \begin{QANS}
            (4)(5)
        \end{QANS}
        \begin{QSOLLIST}
        \end{QSOLLIST}
        \begin{QEMPTYSPACE}
        \end{QEMPTYSPACE}
    \end{QUESTION}
    \begin{QUESTION}
        \begin{ExamInfo}{102}{學測}{多選}{10}
        \end{ExamInfo}
        \begin{ExamAnsRateInfo}{36}{60}{30}{18}
        \end{ExamAnsRateInfo}
        \begin{QBODY}
            坐標平面上考慮兩點${{Q}_{1}}(1,0),{{Q}_{2}}(-1,0)$。在下列各方程式的圖形中,請選出其上至少有一點$P$滿足內積 $\lvec{PQ_{1}}\cdot \lvec{PQ_{2}}<0$的選項。
		\begin{QOPS}
			\QOP $y=\FR{1}{2}$  
			\QOP $y={{x}^{2}}+1$ 
			\QOP $-{{x}^{2}}+2{{y}^{2}}=1$
			\QOP $4{{x}^{2}}+{{y}^{2}}=1$
			\QOP $\FR{{{x}^{2}}}{2}-\FR{{{y}^{2}}}{2}=1$
		\end{QOPS}
        \end{QBODY}
        \begin{QFROMS}
        \end{QFROMS}
        \begin{QTAGS}\QTAG{B3C3-2平面向量的內積}\QTAG{B3C3平面向量}\QTAG{夾角}\QTAG{圖形}\end{QTAGS}
        \begin{QANS}
            (1)(3)(4)
        \end{QANS}
        \begin{QSOLLIST}
        \end{QSOLLIST}
        \begin{QEMPTYSPACE}
        \end{QEMPTYSPACE}
    \end{QUESTION}
    \begin{QUESTION}
        \begin{ExamInfo}{102}{學測}{多選}{11}
        \end{ExamInfo}
        \begin{ExamAnsRateInfo}{42}{58}{45}{23}
        \end{ExamAnsRateInfo}
        \begin{QBODY}
            設${{F}_{1}},{{F}_{2}}$為橢圓$\Gamma $的兩個焦點。$S$為以${{F}_{1}}$為中心的正方形($S$的各邊可不與$\Gamma $的對稱軸平行)。試問$S$可能有幾個頂點落在$\Gamma $上?
			\begin{QOPS}
				\QOP 1
				\QOP 2
				\QOP 3
				\QOP 4
				\QOP 0
			\end{QOPS}
        \end{QBODY}
        \begin{QFROMS}
        \end{QFROMS}
        \begin{QTAGS}\QTAG{B4C4二次曲線}\QTAG{B4C4-2橢圓}\end{QTAGS}
        \begin{QANS}
            (1)(2)(5)
        \end{QANS}
        \begin{QSOLLIST}
        \end{QSOLLIST}
        \begin{QEMPTYSPACE}
        \end{QEMPTYSPACE}
    \end{QUESTION}
    \begin{QUESTION}
        \begin{ExamInfo}{102}{學測}{多選}{12}
        \end{ExamInfo}
        \begin{ExamAnsRateInfo}{45}{67}{46}{22}
        \end{ExamAnsRateInfo}
        \begin{QBODY}
            設實數組成的數列$\left\langle {{a}_{n}} \right\rangle $是公比為$-\,0.8$的等比數列,實數組成的數列$\left\langle {{b}_{n}} \right\rangle $是首項為10的等差數列。已知${{a}_{9}}>{{b}_{9}}$且${{a}_{10}}>{{b}_{10}}$。請選出正確的選項。
			\begin{QOPS}
				\QOP ${{a}_{9}}\times {{a}_{10}}<0$
				\QOP ${{b}_{10}}>0$
				\QOP ${{b}_{9}}>{{b}_{10}}$
				\QOP ${{a}_{9}}>{{a}_{10}}$
				\QOP ${{a}_{8}}>{{b}_{8}}$
			\end{QOPS}
        \end{QBODY}
        \begin{QFROMS}
        \end{QFROMS}
        \begin{QTAGS}\QTAG{B2C1數列級數}\QTAG{等比數列}\QTAG{B2C1-1數列}\QTAG{等差數列}\end{QTAGS}
        \begin{QANS}
            (1)(3)
        \end{QANS}
        \begin{QSOLLIST}
        \end{QSOLLIST}
        \begin{QEMPTYSPACE}
        \end{QEMPTYSPACE}
    \end{QUESTION}
    \begin{QUESTION}
        \begin{ExamInfo}{102}{學測}{填充}{A}
        \end{ExamInfo}
        \begin{ExamAnsRateInfo}{68}{95}{82}{27}
        \end{ExamAnsRateInfo}
        \begin{QBODY}
            設$k$為一整數。已知$\FR{k}{3}<\sqrt{31}<\FR{k+1}{3}$,則$k= \TCNBOX{\TCN\TCN}$        。
        \end{QBODY}
        \begin{QFROMS}
        \end{QFROMS}
        \begin{QTAGS}\QTAG{B1C1-1數與數線}\QTAG{B1C1數與式}\end{QTAGS}
        \begin{QANS}
            $16$
        \end{QANS}
        \begin{QSOLLIST}
        \end{QSOLLIST}
        \begin{QEMPTYSPACE}
        \end{QEMPTYSPACE}
    \end{QUESTION}
    \begin{QUESTION}
        \begin{ExamInfo}{102}{學測}{填充}{B}
        \end{ExamInfo}
        \begin{ExamAnsRateInfo}{59}{97}{71}{9}
        \end{ExamAnsRateInfo}
        \begin{QBODY}
            設$a,b$為實數且$(a+bi)(2+6i)=-80$,其中${{i}^{2}}=-1$。則$(a,b)=\TCNBOX{(\TCN\TCN,\TCN\TCN)}$。
        \end{QBODY}
        \begin{QFROMS}
        \end{QFROMS}
        \begin{QTAGS}\QTAG{B1C2-3多項式方程式}\QTAG{B1C2多項式函數}\QTAG{複數}\end{QTAGS}
        \begin{QANS}
            $(-4,12)$
        \end{QANS}
        \begin{QSOLLIST}
        \end{QSOLLIST}
        \begin{QEMPTYSPACE}
        \end{QEMPTYSPACE}
    \end{QUESTION}
    \begin{QUESTION}
        \begin{ExamInfo}{102}{學測}{填充}{C}
        \end{ExamInfo}
        \begin{ExamAnsRateInfo}{64}{94}{74}{24}
        \end{ExamAnsRateInfo}
        \begin{QBODY}
            坐標平面中$A(a,3),\text{ }B(16,b),\text{ }C(19,12)$三點共線。已知$C$不在$A,B$之間,且$\overline{AC}:\overline{BC}=3:1$,則$a+b=\TCNBOX{\TCN\TCN}$
        \end{QBODY}
        \begin{QFROMS}
        \end{QFROMS}
        \begin{QTAGS}\QTAG{共線定理}\QTAG{B3C3-1平面向量的表示法}\QTAG{B3C3平面向量}\end{QTAGS}
        \begin{QANS}
            $19$
        \end{QANS}
        \begin{QSOLLIST}
        \end{QSOLLIST}
        \begin{QEMPTYSPACE}
        \end{QEMPTYSPACE}
    \end{QUESTION}
    \begin{QUESTION}
        \begin{ExamInfo}{102}{學測}{填充}{D}
        \end{ExamInfo}
        \begin{ExamAnsRateInfo}{23}{53}{15}{1}
        \end{ExamAnsRateInfo}
        \begin{QBODY}
            阿德賣100公斤的香蕉,第一天每公斤賣40元;沒賣完的部份,第二天降價為每公斤36元;第三天再降為每公斤32元,到第三天全部賣完,三天所得共為3720元。假設阿德在第三天所賣香蕉的公斤數為$t$,可算得第二天賣出香蕉的公斤數為$at+b$,其中$a= \TCNBOX{\TCN\TCN}, b= \TCNBOX{\TCN\TCN}$。
        \end{QBODY}
        \begin{QFROMS}
        \end{QFROMS}
        \begin{QTAGS}\QTAG{一次多項式}\QTAG{B1C2多項式函數}\QTAG{B1C2-1簡單的多項式及圖形}\end{QTAGS}
        \begin{QANS}
            $-2,70$
        \end{QANS}
        \begin{QSOLLIST}
        \end{QSOLLIST}
        \begin{QEMPTYSPACE}
        \end{QEMPTYSPACE}
    \end{QUESTION}
    \begin{QUESTION}
        \begin{ExamInfo}{102}{學測}{填充}{E}
        \end{ExamInfo}
        \begin{ExamAnsRateInfo}{29}{67}{19}{1}
        \end{ExamAnsRateInfo}
        \begin{QBODY}
            坐標平面上,一圓與直線$x-y=1$以及直線$x-y=5$所截的弦長皆為14。則此圓的面積為$\TCNBOX{\TCN\TCN} \pi $。
        \end{QBODY}
        \begin{QFROMS}
        \end{QFROMS}
        \begin{QTAGS}\QTAG{B3C2-3圓與直線的關係}\QTAG{B3C2直線與圓}\end{QTAGS}
        \begin{QANS}
            $51$
        \end{QANS}
        \begin{QSOLLIST}
        \end{QSOLLIST}
        \begin{QEMPTYSPACE}
        \end{QEMPTYSPACE}
    \end{QUESTION}
    \begin{QUESTION}
        \begin{ExamInfo}{102}{學測}{填充}{F}
        \end{ExamInfo}
        \begin{ExamAnsRateInfo}{11}{29}{3}{1}
        \end{ExamAnsRateInfo}
        \begin{QBODY}
            令$\lvec{A}, \lvec{B}$  為坐標平面上兩向量。已知 $\lvec{A}$ 的長度為 $1$, $\lvec{B}$ 的長度為 $2$ 且 $\lvec{A}, \lvec{B}$ 之間的夾角為 $60 ^\circ$ ,令 $\lvec{u} = \lvec{A} + \lvec{B}$,$\lvec{v} = x\lvec{A} + y \lvec{B} $,其中 $x,y$ 為實數且符合 $6 \le x+y  \le 8 $ 以及 $-2 \le x-y \le 0$,則內積 $\lvec{u} \cdot \lvec{v}$ 的最大值為 $\TCNBOX{ \TCN\TCN }$ 。
        \end{QBODY}
        \begin{QFROMS}
        \end{QFROMS}
        \begin{QTAGS}\QTAG{B3C3-2平面向量的內積}\QTAG{夾角}\QTAG{B3C3平面向量}\QTAG{B3C2-2線性規劃}\QTAG{B3C2直線與圓}\end{QTAGS}
        \begin{QANS}
            $31$
        \end{QANS}
        \begin{QSOLLIST}
        \end{QSOLLIST}
        \begin{QEMPTYSPACE}
        \end{QEMPTYSPACE}
    \end{QUESTION}
    \begin{QUESTION}
        \begin{ExamInfo}{102}{學測}{填充}{G}
        \end{ExamInfo}
        \begin{ExamAnsRateInfo}{19}{46}{10}{1}
        \end{ExamAnsRateInfo}
        \begin{QBODY}
            設銳角三角形$ABC$的外接圓半徑為8。已知外接圓圓心到$\overline{AB}$的距離為2,而到$\overline{BC}$的距離為7,則$\overline{AC}=\TCNBOX{\TCN\sqrt{\TCN\TCN}}$ 。(化成最簡根式)
        \end{QBODY}
        \begin{QFROMS}
        \end{QFROMS}
        \begin{QTAGS}\QTAG{B3C1-4差角公式}\QTAG{B3C1三角}\end{QTAGS}
        \begin{QANS}
            $4\sqrt{15}$
        \end{QANS}
        \begin{QSOLLIST}
        \end{QSOLLIST}
        \begin{QEMPTYSPACE}
        \end{QEMPTYSPACE}
    \end{QUESTION}
    \begin{QUESTION}
        \begin{ExamInfo}{102}{學測}{填充}{H}
        \end{ExamInfo}
        \begin{ExamAnsRateInfo}{18}{48}{6}{0}
        \end{ExamAnsRateInfo}
        \begin{QBODY}
            如下圖,在坐標空間中,$A,B,C,D,E,F,G,H$為正立方體的八個頂點,已知其中四個點的坐標A(0,0,0)、B(6,0,0)、D(0,6,0)及E(0,0,6),$P$在線段$\overline{CG}$上且$\overline{CP}:\overline{PG}=1:5$,$R$在線段$\overline{EH}$上且$\overline{ER}:\overline{RH}=1:1$,$Q$在線段$\overline{AD}$上。若空間中通過P,Q,R這三點的平面,與直線$AG$不相交,則$Q$點的$y$坐標為$\TCNBOX{\FR{\TCN\TCN}{\TCN\TCN}}$ 。(化成最簡分數) 
			
			\begin{tikzpicture}[every edge quotes/.append style={auto, text=blue},
				x={(-0.25cm,-0.15cm)},
				y={(0.5cm,0cm)},
				z={(0cm,0.5cm)}]

				\coordinate (O) at (0,0,0);
				\coordinate (x) at (10,0,0);
				\coordinate (y) at (0,8,0);
				\coordinate (z) at (0,0,8);

				\coordinate (Base1) at (0,0,0);
				\coordinate (Base2) at (6,0,0);
				\coordinate (Base3) at (6,6,0);
				\coordinate (Base4) at (0,6,0);
				\coordinate (Base1Up) at (0,0,6);
				\coordinate (Base2Up) at (6,0,6);
				\coordinate (Base3Up) at (6,6,6);
				\coordinate (Base4Up) at (0,6,6);

				\coordinate (A) at (Base1);
				\coordinate (B) at (Base2);
				\coordinate (C) at (Base3);
				\coordinate (D) at (Base4);
				\coordinate (E) at (Base1Up);
				\coordinate (F) at (Base2Up);
				\coordinate (G) at (Base3Up);
				\coordinate (H) at (Base4Up);

				\coordinate (P) at (6,6,1);
				\coordinate (Q) at (0,15/11,0);
				\coordinate (R) at (0,3,6);

				\draw [draw=black, every edge/.append style={draw=black, dashed}]
				(Base1) edge (Base2)
				(Base2) -- (Base3)
				(Base3) -- (Base4)
				(Base4) edge (Base1)
				(Base1Up) -- (Base2Up)
				(Base2Up) -- (Base3Up)
				(Base3Up) -- (Base4Up)
				(Base4Up) -- (Base1Up)
				(Base1) edge (Base1Up)
				(Base2) -- (Base2Up)
				(Base3) -- (Base3Up)
				(Base4) -- (Base4Up);

				\draw[-{Stealth[scale=1.3,angle'=45]},semithick] (D) -- (y) node[right] {$y$};
				\draw[-{Stealth[scale=1.3,angle'=45]},semithick] (B) -- (x) node[left] {$x$};
				\draw[-{Stealth[scale=1.3,angle'=45]},semithick] (E) -- (z) node[above] {$z$};;


				\foreach \v/\u/\t in 
				{   P/left/$P$,
					Q/below/$Q$,
					R/above/$R$,
					A/left/$A$,
					B/left/$B$,
					C/below/$C$,
					D/below/$D$,
					E/left/$E$,
					F/left/$F$,
					G/right/$G$,
					H/right/$H$
				}
				{
					\draw[ultra thick,fill] (\v) circle (1.5pt);
					\node[\u] at (\v){\t};
				}; 

			\end{tikzpicture}
        \end{QBODY}
        \begin{QFROMS}
        \end{QFROMS}
        \begin{QTAGS}\QTAG{B4C2空間中的平面與直線}\QTAG{B4C2-2空間直線方程式}\end{QTAGS}
        \begin{QANS}
            $\frac{15}{11}$
        \end{QANS}
        \begin{QSOLLIST}
        \end{QSOLLIST}
        \begin{QEMPTYSPACE}
        \end{QEMPTYSPACE}
    \end{QUESTION}
