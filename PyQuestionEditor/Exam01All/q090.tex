% !TEX encoding = UTF-8 Unicode
% !TEX TS-program = xelatex 
\begin{QUESTIONS}
    \begin{QUESTION}
        \begin{ExamInfo}{90}{學測}{單選}{1}
        \end{ExamInfo}
        \begin{ExamAnsRateInfo}{}{}{}{}
        \end{ExamAnsRateInfo}
        \begin{QBODY}
            設$a={{\left( \frac{1}{2} \right)}^{\frac{1}{2}}},\ \ b={{\left( \frac{1}{3} \right)}^{\frac{1}{3}}},\ \ c={{\left( \frac{1}{4} \right)}^{\frac{1}{4}}}$。下列選項何者為真?
            \begin{QOPS}         
                \QOP $ a > b > c $
                \QOP $ a < b < c $
                \QOP $ a = c > b $
                \QOP $ a = c < b $
            \end{QOPS}
        \end{QBODY}
        \begin{QFROMS}
        \end{QFROMS}
        \begin{QTAGS}\end{QTAGS}
        \begin{QANS}
            (3)
        \end{QANS}
        \begin{QSOLLIST}
        \end{QSOLLIST}
        \begin{QEMPTYSPACE}
        \end{QEMPTYSPACE}
    \end{QUESTION}
    \begin{QUESTION}
        \begin{ExamInfo}{90}{學測}{單選}{2}
        \end{ExamInfo}
        \begin{ExamAnsRateInfo}{}{}{}{}
        \end{ExamAnsRateInfo}
        \begin{QBODY}
            右下圖為一拋物線的部分圖形,且$ A $、$ B $、$ C $、$ D $、$ E $五個點中有一為其焦點。試判斷哪一點是其焦點?(可利用你手邊現有簡易測量工具)
			%TODO: 圖
			\begin{QOPS}
				\QOP $A$
				\QOP $B$
				\QOP $C$
				\QOP $D$
				\QOP $E$
			\end{QOPS}
        \end{QBODY}
        \begin{QFROMS}
        \end{QFROMS}
        \begin{QTAGS}\end{QTAGS}
        \begin{QANS}
            (3)
        \end{QANS}
        \begin{QSOLLIST}
            \begin{QSOL}
            \end{QSOL}
        \end{QSOLLIST}
        \begin{QEMPTYSPACE}
        \end{QEMPTYSPACE}
    \end{QUESTION}
    \begin{QUESTION}
        \begin{ExamInfo}{90}{學測}{單選}{3}
        \end{ExamInfo}
        \begin{ExamAnsRateInfo}{}{}{}{}
        \end{ExamAnsRateInfo}
        \begin{QBODY}
				令$X$代表每個高中生平均每天研讀數學的時間(以小時計),則$W=7(24-X)$代表每個高中生平均每週花在研讀數學以外的時間。令$Y$代表每個高中生數學學科能力測驗的成績。設$X,Y$之相關係數為$R_{XY}$,$W,Y$之相關係數為$R_{WY}$,則$R_{XY}$與$R_{WY}$ 兩數之間的關係,下列選項何者為真?
				\begin{QOPS}
					\QOP ${{R}_{WY}}=7\left( 24-{{R}_{XY}} \right)$
					\QOP ${{R}_{WY}}=7{{R}_{XY}}$
					\QOP ${{R}_{WY}}=-7{{R}_{XY}}$
					\QOP ${{R}_{WY}}={{R}_{XY}}$
					\QOP ${{R}_{WY}}=-{{R}_{XY}}$
				\end{QOPS}
        \end{QBODY}
        \begin{QFROMS}
        \end{QFROMS}
        \begin{QTAGS}\end{QTAGS}
        \begin{QANS}
            (5)
        \end{QANS}
        \begin{QSOLLIST}
        \end{QSOLLIST}
        \begin{QEMPTYSPACE}
        \end{QEMPTYSPACE}
    \end{QUESTION}
\end{QUESTIONS}
\begin{QUESTIONS}
    \begin{QUESTION}
        \begin{ExamInfo}{90}{學測}{多選}{4}
        \end{ExamInfo}
        \begin{ExamAnsRateInfo}{}{}{}{}
            (3)(4)(5)
        \end{ExamAnsRateInfo}
        \begin{QBODY}
			若$\sin x=\frac{3}{5},\quad \frac{\pi }{2}<x<\pi $,則下列選項何者為真?
			\begin{QOPS}
				\QOP $\cos x=\frac{4}{5}$
				\QOP $\tan x=\frac{3}{4}$
				\QOP $\cot x=-\frac{4}{3}$
				\QOP $\sec x=-\frac{5}{4}$
				\QOP $\csc x=\frac{5}{3}$
			\end{QOPS}

        \end{QBODY}
        \begin{QFROMS}
        \end{QFROMS}
        \begin{QTAGS}\end{QTAGS}
        \begin{QANS}
        \end{QANS}
        \begin{QSOLLIST}
        \end{QSOLLIST}
        \begin{QEMPTYSPACE}
        \end{QEMPTYSPACE}
    \end{QUESTION}
    \begin{QUESTION}
        \begin{ExamInfo}{90}{學測}{多選}{5}
        \end{ExamInfo}
        \begin{ExamAnsRateInfo}{}{}{}{}
        \end{ExamAnsRateInfo}
        \begin{QBODY}
			設$ a,b,c $ 為實數。若二次函數\\
			$f\left( x \right)=a{{x}^{2}}+bx+c$\\
			的圖形通過$(0,-1)$且與$x$軸相切,則下列選項何者為真?
			\begin{QOPS}
				\QOP $a<0$
				\QOP $b>0$
				\QOP $c=-1$
				\QOP ${{b}^{2}}+4ac=0$
				\QOP $a+b+c\le 0$
			\end{QOPS}
        \end{QBODY}
        \begin{QFROMS}
        \end{QFROMS}
        \begin{QTAGS}\end{QTAGS}
        \begin{QANS}
            (1)(3)(5)
        \end{QANS}
        \begin{QSOLLIST}
        \end{QSOLLIST}
        \begin{QEMPTYSPACE}
        \end{QEMPTYSPACE}
    \end{QUESTION}
    \begin{QUESTION}
        \begin{ExamInfo}{90}{學測}{多選}{6}
        \end{ExamInfo}
        \begin{ExamAnsRateInfo}{}{}{}{}
        \end{ExamAnsRateInfo}
        \begin{QBODY}
			若正整數$a,b,q,r$滿足
			$a=bq+r$
			且令$\left( a,b \right)$表示a與b的最大公因數,則下列選項何者為真?
			\begin{QOPS}
				\QOP $\left( a,b \right)=\left( b,r \right)$
				\QOP $\left( a,b \right)=\left( q,r \right)$
				\QOP $\left( a,q \right)=\left( b,r \right)$
				\QOP $\left( a,q \right)=\left( q,r \right)$
				\QOP $\left( a,r \right)=\left( b,q \right)$
			\end{QOPS}
        \end{QBODY}
        \begin{QFROMS}
        \end{QFROMS}
        \begin{QTAGS}\end{QTAGS}
        \begin{QANS}
            (1)(4)
        \end{QANS}
        \begin{QSOLLIST}
        \end{QSOLLIST}
        \begin{QEMPTYSPACE}
        \end{QEMPTYSPACE}
    \end{QUESTION}
    \begin{QUESTION}
        \begin{ExamInfo}{90}{學測}{多選}{7}
        \end{ExamInfo}
        \begin{ExamAnsRateInfo}{}{}{}{}
        \end{ExamAnsRateInfo}
        \begin{QBODY}
			古代的足球運動,有一種計分法,規定踢進一球得$16$分,犯規後的罰踢,進一球得$6$分。請問下列哪些得分數有可能在計分板上出現?
			\begin{QOPS}
				\QOP $26$
				\QOP $28$
				\QOP $82$
				\QOP $103$
				\QOP $284$
			\end{QOPS}
        \end{QBODY}
        \begin{QFROMS}
        \end{QFROMS}
        \begin{QTAGS}\end{QTAGS}
        \begin{QANS}
            (2)(3)(5)
        \end{QANS}
        \begin{QSOLLIST}
        \end{QSOLLIST}
        \begin{QEMPTYSPACE}
        \end{QEMPTYSPACE}
    \end{QUESTION}
    \begin{QUESTION}
        \begin{ExamInfo}{90}{學測}{多選}{8}
        \end{ExamInfo}
        \begin{ExamAnsRateInfo}{}{}{}{}
        \end{ExamAnsRateInfo}
        \begin{QBODY}
			在坐標平面上,$A(150,200),B(146,203),C(-4,3),O(0,0)$,則下列選項何者為真?
			\begin{QOPS}
				\QOP 四邊形$ABCO$是一個平行四邊形
				\QOP 四邊形$ABCO$是一個長方形
				\QOP 四邊形$ABCO$ 的兩對角線互相垂直
				\QOP 四邊形$ABCO$的對角線$AC$長度大於$251$
				\QOP 四邊形$ABCO$ 的面積為$1250$
			\end{QOPS}
        \end{QBODY}
        \begin{QFROMS}
        \end{QFROMS}
        \begin{QTAGS}\end{QTAGS}
        \begin{QANS}
            (1)(2)(5)
        \end{QANS}
        \begin{QSOLLIST}
        \end{QSOLLIST}
        \begin{QEMPTYSPACE}
        \end{QEMPTYSPACE}
    \end{QUESTION}
    \begin{QUESTION}
        \begin{ExamInfo}{90}{學測}{多選}{9}
        \end{ExamInfo}
        \begin{ExamAnsRateInfo}{}{}{}{}
        \end{ExamAnsRateInfo}
        \begin{QBODY}
			在坐標平面上,請問下列哪些直線與雙曲線 $\frac{{{x}^{2}}}{25}-\frac{{{y}^{2}}}{4}=1$ 不相交?
			\begin{QOPS}
				\QOP $5y=2x$
				\QOP $5y=3x$
				\QOP $5y=2x+1$
				\QOP $5y=-2x$
				\QOP $y=100$
			\end{QOPS}
        \end{QBODY}
        \begin{QFROMS}
        \end{QFROMS}
        \begin{QTAGS}\end{QTAGS}
        \begin{QANS}
            (1)(2)(4)
        \end{QANS}
        \begin{QSOLLIST}
        \end{QSOLLIST}
        \begin{QEMPTYSPACE}
        \end{QEMPTYSPACE}
    \end{QUESTION}
    \begin{QUESTION}
        \begin{ExamInfo}{90}{學測}{多選}{10}
        \end{ExamInfo}
        \begin{ExamAnsRateInfo}{}{}{}{}
        \end{ExamAnsRateInfo}
        \begin{QBODY}
		令$z$為複數且${{z}^{6}}=1,\quad z\ne 1$,則下列選項何者為真?
			\begin{QOPS}
				\QOP $\left| \,z\, \right|=1$
				\QOP ${{z}^{2}}=1$
				\QOP ${{z}^{3}}=1$或${{z}^{3}}=-1$
				\QOP $\left| \,{{z}^{4}}\, \right|=1$
				\QOP $1+z+{{z}^{2}}+{{z}^{3}}+{{z}^{4}}+{{z}^{5}}=0$
			\end{QOPS}
        \end{QBODY}
        \begin{QFROMS}
        \end{QFROMS}
        \begin{QTAGS}\end{QTAGS}
        \begin{QANS}
            (1)(3)(4)(5)
        \end{QANS}
        \begin{QSOLLIST}
        \end{QSOLLIST}
        \begin{QEMPTYSPACE}
        \end{QEMPTYSPACE}
    \end{QUESTION}
\end{QUESTIONS}
\begin{QUESTIONS}
    \begin{QUESTION}
        \begin{ExamInfo}{90}{學測}{選填}{11}
        \end{ExamInfo}
        \begin{ExamAnsRateInfo}{}{}{}{}
        \end{ExamAnsRateInfo}
        \begin{QBODY}
		將一張B4的長方形紙張對折剪開之後,成為B5的紙張,其形狀跟原來B4的形狀相似。已知B4紙張的長邊為36.4公分,則B4紙張的短
邊長為$\TCNBOX{\TCN\TCN .\TCN}$公分。(小數點後第二位四捨五入)

        \end{QBODY}
        \begin{QFROMS}
        \end{QFROMS}
        \begin{QTAGS}\end{QTAGS}
        \begin{QANS}
            $25.7$
        \end{QANS}
        \begin{QSOLLIST}
        \end{QSOLLIST}
        \begin{QEMPTYSPACE}
        \end{QEMPTYSPACE}
    \end{QUESTION}
    \begin{QUESTION}
        \begin{ExamInfo}{90}{學測}{選填}{12}
        \end{ExamInfo}
        \begin{ExamAnsRateInfo}{}{}{}{}
        \end{ExamAnsRateInfo}
        \begin{QBODY}
			調查某新興工業都市的市民對市長施政的滿意情況,依據隨機抽樣,共抽樣男性$600$人、女性$400$人,由甲、乙兩組人分別調查男性與女性市民。調查結果男性中有$36$%滿意市長的施政,女性市民中有$46$%滿意市長的施政,則滿意市長施政的樣本佔全體樣本的百分比為
        $\TCNBOX{\TCN\TCN}$%
        \end{QBODY}
        \begin{QFROMS}
        \end{QFROMS}
        \begin{QTAGS}\end{QTAGS}
        \begin{QANS}
            $40$
        \end{QANS}
        \begin{QSOLLIST}
        \end{QSOLLIST}
        \begin{QEMPTYSPACE}
        \end{QEMPTYSPACE}
    \end{QUESTION}
    \begin{QUESTION}
        \begin{ExamInfo}{90}{學測}{選填}{13}
        \end{ExamInfo}
        \begin{ExamAnsRateInfo}{}{}{}{}
        \end{ExamAnsRateInfo}
        \begin{QBODY}
			從1,2,3,4,5,6,7,8,9中,任取兩相異數,則其積為完全立方數的機率為$\TCNBOX{\FR{1}{\TCN\TCN}}$
          。
        \end{QBODY}
        \begin{QFROMS}
        \end{QFROMS}
        \begin{QTAGS}\end{QTAGS}
        \begin{QANS}
            $12$
        \end{QANS}
        \begin{QSOLLIST}
        \end{QSOLLIST}
        \begin{QEMPTYSPACE}
        \end{QEMPTYSPACE}
    \end{QUESTION}
    \begin{QUESTION}
        \begin{ExamInfo}{90}{學測}{選填}{14}
        \end{ExamInfo}
        \begin{ExamAnsRateInfo}{}{}{}{}
        \end{ExamAnsRateInfo}
        \begin{QBODY}
		設多項式$f\left( x \right)$除以${{x}^{2}}-5x+4$,餘式為$x+2$;除以${{x}^{2}}-5x+6$,餘式為
$3x+4$。則多項式$f\left( x \right)$除以${{x}^{2}}-4x+3$,餘式為   $\TCNBOX{\TCN x - \TCN }$   。

        \end{QBODY}
        \begin{QFROMS}
        \end{QFROMS}
        \begin{QTAGS}\end{QTAGS}
        \begin{QANS}
            $5x-2$
        \end{QANS}
        \begin{QSOLLIST}
        \end{QSOLLIST}
        \begin{QEMPTYSPACE}
        \end{QEMPTYSPACE}
    \end{QUESTION}
    \begin{QUESTION}
        \begin{ExamInfo}{90}{學測}{選填}{15}
        \end{ExamInfo}
        \begin{ExamAnsRateInfo}{}{}{}{}
        \end{ExamAnsRateInfo}
        \begin{QBODY}
			兩條公路$k$及$m$,如果筆直延伸將交會於C處成60夾角,如圖所示。為銜接此二公路,規劃在兩公路各距C處450公尺的A、B兩點間開拓成圓弧型公路,使 $k$,$m$ 分別在A,B與此圓弧相切,
則此圓弧長=$\TCNBOX{\TCN\TCN\TCN}$公尺。
(公尺以下四捨五入)
【$\sqrt{3}\approx 1.732,\ \pi \approx 3.142$】
		%TODO:補圖
        \end{QBODY}
        \begin{QFROMS}
        \end{QFROMS}
        \begin{QTAGS}\end{QTAGS}
        \begin{QANS}
            $544$
        \end{QANS}
        \begin{QSOLLIST}
        \end{QSOLLIST}
        \begin{QEMPTYSPACE}
        \end{QEMPTYSPACE}
    \end{QUESTION}
    \begin{QUESTION}
        \begin{ExamInfo}{90}{學測}{選填}{16}
        \end{ExamInfo}
        \begin{ExamAnsRateInfo}{}{}{}{}
        \end{ExamAnsRateInfo}
        \begin{QBODY}
			如右圖的四角錐展開圖,四角錐底面為邊長2的正方形,四個側面都
是腰長為4的等腰三角形,則此四角錐的高度為$\TCNBOX{\sqrt{\TCN\TCN}}$。
	%TODO: 補圖

        \end{QBODY}
        \begin{QFROMS}
        \end{QFROMS}
        \begin{QTAGS}\end{QTAGS}
        \begin{QANS}
            $14$
        \end{QANS}
        \begin{QSOLLIST}
        \end{QSOLLIST}
        \begin{QEMPTYSPACE}
        \end{QEMPTYSPACE}
    \end{QUESTION}
    \begin{QUESTION}
        \begin{ExamInfo}{90}{學測}{選填}{17}
        \end{ExamInfo}
        \begin{ExamAnsRateInfo}{}{}{}{}
        \end{ExamAnsRateInfo}
        \begin{QBODY}
			在坐標平面的x軸上有A(2,0),B(-4,0)兩觀測站,同時觀察在x軸上方的一目標C點,測得BAC及ABC之值後,通知在$\text{D(}\ \frac{\text{5}}{\text{2}}\text{,}-\text{8)}$的砲台此兩個角的正切值分別為$\frac{8}{9}\frac{8}{3}$。那麼砲台D至目標C的距離為$\TCNBOX{\TCN\TCN}$。
        \end{QBODY}
        \begin{QFROMS}
        \end{QFROMS}
        \begin{QTAGS}\end{QTAGS}
        \begin{QANS}
            $13$
        \end{QANS}
        \begin{QSOLLIST}
        \end{QSOLLIST}
        \begin{QEMPTYSPACE}
        \end{QEMPTYSPACE}
    \end{QUESTION}
    \begin{QUESTION}
        \begin{ExamInfo}{90}{學測}{選填}{18}
        \end{ExamInfo}
        \begin{ExamAnsRateInfo}{}{}{}{}
        \end{ExamAnsRateInfo}
        \begin{QBODY}
		將一個正四面體的四個面上的各邊中點用線段連接,可得四個小正四面體及一個正八面體,如下圖所示。如果原四面體$ABCD$的體積為$12$,
那麼此正八面體的體積為$\TCNBOX{\TCN}$。
		%TODO:補圖
        \end{QBODY}
        \begin{QFROMS}
        \end{QFROMS}
        \begin{QTAGS}\end{QTAGS}
        \begin{QANS}
            $6$
        \end{QANS}
        \begin{QSOLLIST}
        \end{QSOLLIST}
        \begin{QEMPTYSPACE}
        \end{QEMPTYSPACE}
    \end{QUESTION}
    \begin{QUESTION}
        \begin{ExamInfo}{90}{學測}{選填}{19}
        \end{ExamInfo}
        \begin{ExamAnsRateInfo}{}{}{}{}
        \end{ExamAnsRateInfo}
        \begin{QBODY}
			根據過去紀錄知,某電腦工廠檢驗其產品的過程中,將良品檢驗為不良品的機率為0.20,將不良品檢驗為良品的機率為0.16。又知該產品中,不良品佔5%,良品佔95%。若一件產品被檢驗為良品,但該產
品實際上為不良品之機率為 $\TCNBOX {0.\TCN\TCN     }$。(小數點後第三位四捨五入)

        \end{QBODY}
        \begin{QFROMS}
        \end{QFROMS}
        \begin{QTAGS}\end{QTAGS}
        \begin{QANS}
            $0.01$
        \end{QANS}
        \begin{QSOLLIST}
        \end{QSOLLIST}
        \begin{QEMPTYSPACE}
        \end{QEMPTYSPACE}
    \end{QUESTION}
    \begin{QUESTION}
        \begin{ExamInfo}{90}{學測}{選填}{20}
        \end{ExamInfo}
        \begin{ExamAnsRateInfo}{}{}{}{}
        \end{ExamAnsRateInfo}
        \begin{QBODY}
			籃球3人鬥牛賽,共有甲、乙、丙、丁、戊、己、庚、辛、壬9人參加,組成3隊,且甲、乙兩人不在同一隊的組隊方法有多少種?
	答:$\TCNBOX{\TCN\TCN\TCN}$種。
        \end{QBODY}
        \begin{QFROMS}
        \end{QFROMS}
        \begin{QTAGS}\end{QTAGS}
        \begin{QANS}
            $210$
        \end{QANS}
        \begin{QSOLLIST}
        \end{QSOLLIST}
        \begin{QEMPTYSPACE}
        \end{QEMPTYSPACE}
    \end{QUESTION}
\end{QUESTIONS}