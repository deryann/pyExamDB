% !TEX encoding = UTF-8 Unicode
% !TEX TS-program = xelatex 
\begin{QUESTIONS}
    \begin{QUESTION}
        \begin{ExamInfo}{85}{學測}{單選}{1}
        \end{ExamInfo}
        \begin{ExamAnsRateInfo}{}{}{}{}
        \end{ExamAnsRateInfo}
        \begin{QBODY}
            ${{\left( 40 \right)}^{255}}$除以$13$的餘數為
            \begin{QOPS} 
                \QOP$1$	
                \QOP$2$	
                \QOP$4$
                \QOP$6$	
                \QOP$8$
            \end{QOPS}
            
        \end{QBODY}
        \begin{QFROMS}
        \end{QFROMS}
        \begin{QTAGS}\end{QTAGS}
        \begin{QANS}
            (1)
        \end{QANS}
        \begin{QSOLLIST}
        \end{QSOLLIST}
        \begin{QEMPTYSPACE}
        \end{QEMPTYSPACE}
    \end{QUESTION}
    \begin{QUESTION}
        \begin{ExamInfo}{85}{學測}{單選}{2}
        \end{ExamInfo}
        \begin{ExamAnsRateInfo}{}{}{}{}
        \end{ExamAnsRateInfo}
        \begin{QBODY}
            坐標平面上點$A \left( 1,2 \right)$到直線$L$的垂足是$D \left( 3,2 \right)$。問$ A $對於$L$的對稱點是下列那一點?
            \begin{QOPS} 
            \QOP $\left( -2,0 \right)$	
            \QOP $\left( -1,2 \right)$
            \QOP $\left( 2,0 \right)$
            \QOP $\left( 2,2 \right)$	
            \QOP $\left( 5,2 \right)$
            \end{QOPS}            
        \end{QBODY}
        \begin{QFROMS}
        \end{QFROMS}
        \begin{QTAGS}\end{QTAGS}
        \begin{QANS}
            (5)
        \end{QANS}
        \begin{QSOLLIST}
        \end{QSOLLIST}
        \begin{QEMPTYSPACE}
        \end{QEMPTYSPACE}
    \end{QUESTION}
    \begin{QUESTION}
        \begin{ExamInfo}{85}{學測}{單選}{3}
        \end{ExamInfo}
        \begin{ExamAnsRateInfo}{}{}{}{}
        \end{ExamAnsRateInfo}
        \begin{QBODY}
            已知直線${{L}_{1}}$,${{L}_{2}}$交於$\left( 1,0,-1 \right)$,且相互垂直,其中
            ${{L}_{1}}:\left\{ \begin{aligned}
            & x=1+t \\ 
            & y=t \\ 
            & z=-1 \\ 
            \end{aligned} \right.t\in R$,	${{\Lambda }_{2}}:\left\{ \begin{aligned}
            & x =1+t  \\ 
            & y =-t  \\ 
            & z =-1-t  \\ 
            \end{aligned} \right.t \in R {{}_{\circ }}$
            若以${{L}_{1}}$為軸將${{L}_{2}}$旋轉一圈得一平面,則此平面的方程式為何?
            \begin{QOPS} 
                \QOP $x=1$	
                \QOP $y=0$	
                \QOP $x+y-1=0$
                \QOP $x-y-z=2$	
                \QOP $x+y-3=0$
            \end{QOPS}
        \end{QBODY}
        \begin{QFROMS}
        \end{QFROMS}
        \begin{QTAGS}\end{QTAGS}
        \begin{QANS}
            (3)
        \end{QANS}
        \begin{QSOLLIST}
        \end{QSOLLIST}
        \begin{QEMPTYSPACE}
        \end{QEMPTYSPACE}
    \end{QUESTION}
    \begin{QUESTION}
        \begin{ExamInfo}{85}{學測}{單選}{4}
        \end{ExamInfo}
        \begin{ExamAnsRateInfo}{}{}{}{}
        \end{ExamAnsRateInfo}
        \begin{QBODY}
            設$f\left( x \right)$為實係數三次多項式,且$f\left( i \right)=0\left( i=\sqrt{-1} \right)$,則函數$y=f\left( x \right)$的圖形與$X$軸有幾個交點?
            \begin{QOPS} 
                \QOP $0$	
                \QOP $1$	
                \QOP $2$
                \QOP $3$	
                \QOP 因$f\left( x \right)$的不同而異
            \end{QOPS}            
        \end{QBODY}
        \begin{QFROMS}
        \end{QFROMS}
        \begin{QTAGS}\end{QTAGS}
        \begin{QANS}
            (2)
        \end{QANS}
        \begin{QSOLLIST}
        \end{QSOLLIST}
        \begin{QEMPTYSPACE}
        \end{QEMPTYSPACE}
    \end{QUESTION}
    \begin{QUESTION}
        \begin{ExamInfo}{85}{學測}{單選}{5}
        \end{ExamInfo}
        \begin{ExamAnsRateInfo}{}{}{}{}
        \end{ExamAnsRateInfo}
        \begin{QBODY}
            	坐標平面上有一橢圓,已知其長軸平行$Y$軸,短軸的一個頂點為$\left( 0,4 \right)$,且其中一焦點為$\left( 4,0 \right)$。問此橢圓長軸的長度為何?
            \begin{QOPS} 
 \QOP$2$	
            \QOP$2\sqrt{2}$	
            \QOP$6$
            \QOP$6\sqrt{2}$	
            \QOP$8\sqrt{2}$
            \end{QOPS}            
        \end{QBODY}
        \begin{QFROMS}
        \end{QFROMS}
        \begin{QTAGS}\end{QTAGS}
        \begin{QANS}
            (5)
        \end{QANS}
        \begin{QSOLLIST}
        \end{QSOLLIST}
        \begin{QEMPTYSPACE}
        \end{QEMPTYSPACE}
    \end{QUESTION}
    \begin{QUESTION}
        \begin{ExamInfo}{85}{學測}{單選}{6}
        \end{ExamInfo}
        \begin{ExamAnsRateInfo}{}{}{}{}
        \end{ExamAnsRateInfo}
        \begin{QBODY}
            %TODO: 補圖
            已知拋物線$\Gamma $的方程式為$y={{\left( x+1 \right)}^{2}}+1$,且直線$y=2x+2$與$\Gamma $相切。設$L$為斜率等於$2$ 的直線,若$L$與$\Gamma $有兩個交點,則$L$上任一點$P$的坐標$\left( \xi \text{,}\psi  \right)$滿足下列那個關係式?(參考圖1)
            \begin{QOPS} 
 \QOP$y>{{\left( x+1 \right)}^{2}}+1$	
            \QOP$y<{{\left( x+1 \right)}^{2}}+1$
            \QOP$y={{\left( x+1 \right)}^{2}}+1$	
            \QOP$y>2x+2$ 
            \QOP$y<2x+2$
            \end{QOPS}            
            
        \end{QBODY}
        \begin{QFROMS}
        \end{QFROMS}
        \begin{QTAGS}\end{QTAGS}
        \begin{QANS}
            (4)
        \end{QANS}
        \begin{QSOLLIST}
        \end{QSOLLIST}
        \begin{QEMPTYSPACE}
        \end{QEMPTYSPACE}
    \end{QUESTION}
    \begin{QUESTION}
        \begin{ExamInfo}{85}{學測}{單選}{7}
        \end{ExamInfo}
        \begin{ExamAnsRateInfo}{}{}{}{}
        \end{ExamAnsRateInfo}
        \begin{QBODY}
            已知下列五個圖形中有一個是$y=-x\left( \cos x \right)$的部分圖形,判斷那一個選項是該圖形?
            %TODO: 補圖
        \end{QBODY}
        \begin{QFROMS}
        \end{QFROMS}
        \begin{QTAGS}\end{QTAGS}
        \begin{QANS}
            (2)
        \end{QANS}
        \begin{QSOLLIST}
        \end{QSOLLIST}
        \begin{QEMPTYSPACE}
        \end{QEMPTYSPACE}
    \end{QUESTION}
    \begin{QUESTION}
        \begin{ExamInfo}{85}{學測}{單選}{8}
        \end{ExamInfo}
        \begin{ExamAnsRateInfo}{}{}{}{}
        \end{ExamAnsRateInfo}
        \begin{QBODY}
            設想地球是個圓球體,已知沿著赤道,經度$10$度間的距離是$1113$公里,那麼沿北緯${{20}^{{}^\circ }}$線,經度$10$度間的距離最接近下面那個數值?(參考圖2)
            \begin{QOPS} 
 \QOP$1019$
            \QOP$1027$
            \QOP$1035$
            \QOP$1046$
            \QOP$1054$ 
            \end{QOPS}          
            %TODO:補圖
        \end{QBODY}
        \begin{QFROMS}
        \end{QFROMS}
        \begin{QTAGS}\end{QTAGS}
        \begin{QANS}
            (4)
        \end{QANS}
        \begin{QSOLLIST}
        \end{QSOLLIST}
        \begin{QEMPTYSPACE}
        \end{QEMPTYSPACE}
    \end{QUESTION}
\end{QUESTIONS}\begin{QUESTIONS}
    \begin{QUESTION}
        \begin{ExamInfo}{85}{學測}{多選}{9}
        \end{ExamInfo}
        \begin{ExamAnsRateInfo}{}{}{}{}
        \end{ExamAnsRateInfo}
        \begin{QBODY}
            	設$y=f\left( x \right)$及$y=g\left( x \right)$ 的圖形都是拋物線,一個開口向上,一個開口向下,則$y=f\left( x \right)+g\left( x \right)$ 的圖形可能出現下列那些情形?
            \begin{QOPS} 
            \QOP 兩條拋物線
            \QOP 一條拋物線
            \QOP 一條直線
            \QOP 橢圓
            \QOP 雙曲線
            \end{QOPS}            
        \end{QBODY}
        \begin{QFROMS}
        \end{QFROMS}
        \begin{QTAGS}\end{QTAGS}
        \begin{QANS}
            (2)(3)
        \end{QANS}
        \begin{QSOLLIST}
        \end{QSOLLIST}
        \begin{QEMPTYSPACE}
        \end{QEMPTYSPACE}
    \end{QUESTION}
    \begin{QUESTION}
        \begin{ExamInfo}{85}{學測}{多選}{10}
        \end{ExamInfo}
        \begin{ExamAnsRateInfo}{}{}{}{}
        \end{ExamAnsRateInfo}
        \begin{QBODY}
            圖3 為某年級國文、英文、歷史三科成績分佈情形的直方圖。根據該圖,下列那些推論是合理的?
            \begin{QOPS} 
                \QOP 歷史的平均分數比國文的平均分數低
                \QOP 歷史的平均分數最低
                \QOP 英文的標準差比國文的標準差小
                \QOP 英文的標準差最大
                \QOP「國文與歷史之相關係數」比「國文與英文之相關係數」高
            \end{QOPS}            
            %TODO:補圖
        \end{QBODY}
        \begin{QFROMS}
        \end{QFROMS}
        \begin{QTAGS}\end{QTAGS}
        \begin{QANS}
            (1)(2)(4)
        \end{QANS}
        \begin{QSOLLIST}
        \end{QSOLLIST}
        \begin{QEMPTYSPACE}
        \end{QEMPTYSPACE}
    \end{QUESTION}
    \begin{QUESTION}
        \begin{ExamInfo}{85}{學測}{多選}{11}
        \end{ExamInfo}
        \begin{ExamAnsRateInfo}{}{}{}{}
        \end{ExamAnsRateInfo}
        \begin{QBODY}
            某品牌之燈泡由$A$廠及$B$廠各生產$30\%$及$70\%$。$A$廠生產的產品中有$1\%$瑕疵品;$B$廠生產的產品中有$5\%$瑕疵品。某日退貨部門回收一件瑕疵品,則下列敘述那些是正確的?
            \begin{QOPS} 
            \QOP 猜此瑕疵品是由$A$廠製造的,猜對的機率較大
            \QOP 猜此瑕疵品是由$B$廠製造的,猜對的機率較大
            \QOP 此瑕疵品由$A$廠製造的機率為${3}/{38}\;$
            \QOP 此瑕疵品由$A$廠製造的機率為${30}/{10000}\;$
            \QOP 此瑕疵品由$B$廠製造的機率為${350}/{10000}\;$
            \end{QOPS}
            
        \end{QBODY}
        \begin{QFROMS}
        \end{QFROMS}
        \begin{QTAGS}\end{QTAGS}
        \begin{QANS}
            (2)(3)
        \end{QANS}
        \begin{QSOLLIST}
        \end{QSOLLIST}
        \begin{QEMPTYSPACE}
        \end{QEMPTYSPACE}
    \end{QUESTION}
    \begin{QUESTION}
        \begin{ExamInfo}{85}{學測}{多選}{12}
        \end{ExamInfo}
        \begin{ExamAnsRateInfo}{}{}{}{}
        \end{ExamAnsRateInfo}
        \begin{QBODY}
            	設$\alpha >\beta >1000$。令$p=\sqrt{{{\log }_{7}}a\cdot {{\log }_{7}}b}$,$q=\frac{1}{2}\left( {{\log }_{7}}a+{{\log }_{7}}b \right)$,$r={{\log }_{7}}\left( \frac{a+b}{2} \right)$,則下列敘述何者正確?
            \begin{QOPS} 
 \QOP$q={{\log }_{7}}\sqrt{ab}$
            \QOP$q>r$
            \QOP$r<p<q$
            \QOP$p<q<r$
            \QOP$q<p<r$
            \end{QOPS}            
            
        \end{QBODY}
        \begin{QFROMS}
        \end{QFROMS}
        \begin{QTAGS}\end{QTAGS}
        \begin{QANS}
            (1)(4)
        \end{QANS}
        \begin{QSOLLIST}
        \end{QSOLLIST}
        \begin{QEMPTYSPACE}
        \end{QEMPTYSPACE}
    \end{QUESTION}
    \begin{QUESTION}
        \begin{ExamInfo}{85}{學測}{多選}{13}
        \end{ExamInfo}
        \begin{ExamAnsRateInfo}{}{}{}{}
        \end{ExamAnsRateInfo}
        \begin{QBODY}
            %TODO:補圖
            	設$\alpha >\beta >1000$。令$p=\sqrt{{{\log }_{7}}a\cdot {{\log }_{7}}b}$,$q=\frac{1}{2}\left( {{\log }_{7}}a+{{\log }_{7}}b \right)$,$r={{\log }_{7}}\left( \frac{a+b}{2} \right)$,則下列敘述何者正確?
            \begin{QOPS} 
                \QOP $q={{\log }_{7}}\sqrt{ab}$
                \QOP $q>r$
                \QOP $r<p<q$
                \QOP $p<q<r$
                \QOP $q<p<r$
            \end{QOPS}            
        \end{QBODY}
        \begin{QFROMS}
        \end{QFROMS}
        \begin{QTAGS}\end{QTAGS}
        \begin{QANS}
            (1)(4)
        \end{QANS}
        \begin{QSOLLIST}
        \end{QSOLLIST}
        \begin{QEMPTYSPACE}
        \end{QEMPTYSPACE}
    \end{QUESTION}
    \begin{QUESTION}
        \begin{ExamInfo}{85}{學測}{多選}{14}
        \end{ExamInfo}
        \begin{ExamAnsRateInfo}{}{}{}{}
        \end{ExamAnsRateInfo}
        \begin{QBODY}
            有一個$101$項的等差數列${{a}_{1}},{{a}_{2}},{{a}_{3}},\cdots ,{{a}_{101}}$,其和為$0$,且${{a}_{71}}=71$。問下列選項那些正確?
            \begin{QOPS} 
 \QOP${{a}_{1}}+{{a}_{101}}>0$
            \QOP${{a}_{2}}+{{a}_{100}}<0$
            \QOP${{a}_{3}}+{{a}_{99}}=0$
            \QOP${{a}_{51}}=51$
            \QOP${{a}_{1}}<0$
            \end{QOPS}            
        \end{QBODY}
        \begin{QFROMS}
        \end{QFROMS}
        \begin{QTAGS}\end{QTAGS}
        \begin{QANS}
            (3)(5)
        \end{QANS}
        \begin{QSOLLIST}
        \end{QSOLLIST}
        \begin{QEMPTYSPACE}
        \end{QEMPTYSPACE}
    \end{QUESTION}
\end{QUESTIONS}\begin{QUESTIONS}
    \begin{QUESTION}
        \begin{ExamInfo}{85}{學測}{選填}{15}
        \end{ExamInfo}
        \begin{ExamAnsRateInfo}{}{}{}{}
        \end{ExamAnsRateInfo}
        \begin{QBODY}
            設$D$點在$\Delta ABC$的$\overline{BC}$邊上,且$\Delta ABD$的面積$=\frac{2}{3}\Delta ADC$的面積,若$B$的坐標為$\left( 0,5 \right)$,$C$的坐標為$\left( 7,0 \right)$,則$D$的坐標為 $\TCNBOX{\FR{\TCN\TCN}{\TCN}, \TCN}$。
        \end{QBODY}
        \begin{QFROMS}
        \end{QFROMS}
        \begin{QTAGS}\end{QTAGS}
        \begin{QANS}
            $(\FR{14}{5}, 3)$
        \end{QANS}
        \begin{QSOLLIST}
        \end{QSOLLIST}
        \begin{QEMPTYSPACE}
        \end{QEMPTYSPACE}
    \end{QUESTION}
    \begin{QUESTION}
        \begin{ExamInfo}{85}{學測}{選填}{16}
        \end{ExamInfo}
        \begin{ExamAnsRateInfo}{}{}{}{}
        \end{ExamAnsRateInfo}
        \begin{QBODY}
             
           圓心在原點的兩個同心圓,面積分別為$75\pi $和$27\pi $。設$P$點在第一象限。若$P$點到大圓、小圓、$X$軸的距離均相等,則$P$點的坐標為$\TCNBOX{(\TCN\sqrt{\TCN}, \sqrt{\TCN})}$。
            
        \end{QBODY}
        \begin{QFROMS}
        \end{QFROMS}
        \begin{QTAGS}\end{QTAGS}
        \begin{QANS}
            $(3\sqrt{5}, \sqrt{3})$
        \end{QANS}
        \begin{QSOLLIST}
        \end{QSOLLIST}
        \begin{QEMPTYSPACE}
        \end{QEMPTYSPACE}
    \end{QUESTION}
    \begin{QUESTION}
        \begin{ExamInfo}{85}{學測}{選填}{17}
        \end{ExamInfo}
        \begin{ExamAnsRateInfo}{}{}{}{}
        \end{ExamAnsRateInfo}
        \begin{QBODY}
            圖5中,至少包含$A$或$B$兩點之一的長方形共有$\TCNBOX{\TCN\TCN}$個。
            %TODO:補圖
        \end{QBODY}
        \begin{QFROMS}
        \end{QFROMS}
        \begin{QTAGS}\end{QTAGS}
        \begin{QANS}
            $15$
        \end{QANS}
        \begin{QSOLLIST}
        \end{QSOLLIST}
        \begin{QEMPTYSPACE}
        \end{QEMPTYSPACE}
    \end{QUESTION}
    \begin{QUESTION}
        \begin{ExamInfo}{85}{學測}{選填}{18}
        \end{ExamInfo}
        \begin{ExamAnsRateInfo}{}{}{}{}
        \end{ExamAnsRateInfo}
        \begin{QBODY}
            擲一均勻硬幣三次,每出現一個正面得$5$元,一個反面賠$2$元,則所得總額之期望值為$\FR{\TCN}{\TCN}$ 元。
        \end{QBODY}
        \begin{QFROMS}
        \end{QFROMS}
        \begin{QTAGS}\end{QTAGS}
        \begin{QANS}
            $\FR{9}{2}$
        \end{QANS}
        \begin{QSOLLIST}
        \end{QSOLLIST}
        \begin{QEMPTYSPACE}
        \end{QEMPTYSPACE}
    \end{QUESTION}
    \begin{QUESTION}
        \begin{ExamInfo}{85}{學測}{選填}{19}
        \end{ExamInfo}
        \begin{ExamAnsRateInfo}{}{}{}{}
        \end{ExamAnsRateInfo}
        \begin{QBODY}
            空間中三向量 $\lvec{u}=\left( {{\upsilon }_{1}},{{\upsilon }_{2}},{{\upsilon }_{3}} \right)$, $\lvec{v}=\left( {{\varpi }_{1}},{{\varpi }_{2}},{{\varpi }_{3}} \right)$, $\lvec{w}=\left( {{\omega }_{1}},{{\omega }_{2}},{{\omega }_{3}} \right)$,所張平行六面體的體積為$\left| \begin{matrix}
            {{\upsilon }_{1}} & {{\upsilon }_{2}} & {{\upsilon }_{3}}  \\
            {{\varpi }_{1}} & {{\varpi }_{2}} & {{\varpi }_{3}}  \\
            {{\omega }_{1}} & {{\omega }_{2}} & {{\omega }_{3}}  \\
            \end{matrix} \right|$的絕對值。今已知 , , 三向量所張平行六面體的體積為$5$,則$2\lvec{a} +3 \lvec{b}$ , $\lvec{b}$, $\lvec{c}$ 三向量所張平行六面體的體積為\TCNBOX{\TCN\TCN}。
            
        \end{QBODY}
        \begin{QFROMS}
        \end{QFROMS}
        \begin{QTAGS}\end{QTAGS}
        \begin{QANS}
            $10$
        \end{QANS}
        \begin{QSOLLIST}
        \end{QSOLLIST}
        \begin{QEMPTYSPACE}
        \end{QEMPTYSPACE}
    \end{QUESTION}
    \begin{QUESTION}
        \begin{ExamInfo}{85}{學測}{選填}{20}
        \end{ExamInfo}
        \begin{ExamAnsRateInfo}{}{}{}{}
        \end{ExamAnsRateInfo}
        \begin{QBODY}
            %TODO:補圖
            學校蓋了一棟正四面體的玻璃溫室(如圖6)。今欲將一鋼柱橫架在室中,作為吊花的橫樑。其兩端分別固定在兩面牆$ABC$和$ACD$的重心$E$,$F$處。生物老師要先知道這個鋼柱多長,才能請工人製作。雖然$\overline{BD}$的長度很容易量出,卻很難爬到$E$,$F$點測量$\overline{EF}$長。生物老師在上課時說出他的問題,立刻有一位同學舉手說他有辦法。這位同學在紙上劃出圖6,算出$\overline{EF}:\overline{BD }$就解決了問題。$\overline{EF}:\overline{BD }= \TCNBOX{\TCN:\TCN}$。
        \end{QBODY}
        \begin{QFROMS}
        \end{QFROMS}
        \begin{QTAGS}\end{QTAGS}
        \begin{QANS}
            $1:3$
        \end{QANS}
        \begin{QSOLLIST}
        \end{QSOLLIST}
        \begin{QEMPTYSPACE}
        \end{QEMPTYSPACE}
    \end{QUESTION}
\end{QUESTIONS}