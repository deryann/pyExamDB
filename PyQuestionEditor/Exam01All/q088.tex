% !TEX encoding = UTF-8 Unicode
% !TEX TS-program = xelatex 
\begin{QUESTIONS}
    \begin{QUESTION}
        \begin{ExamInfo}{88}{學測}{單選}{1}
        \end{ExamInfo}
        \begin{ExamAnsRateInfo}{}{}{}{}
        \end{ExamAnsRateInfo}
        \begin{QBODY}
            下列何者是$ 2^{100} $ 除以10的餘數?
            \begin{QOPS}
                \QOP $0$
                \QOP $2$
                \QOP $4$
                \QOP $6$
                \QOP $8$
            \end{QOPS}
        \end{QBODY}
        \begin{QFROMS}
        \end{QFROMS}
        \begin{QTAGS}\end{QTAGS}
        \begin{QANS}
            (4)
        \end{QANS}
        \begin{QSOLLIST}
        \end{QSOLLIST}
        \begin{QEMPTYSPACE}
        \end{QEMPTYSPACE}
    \end{QUESTION}
    \begin{QUESTION}
        \begin{ExamInfo}{88}{學測}{單選}{2}
        \end{ExamInfo}
        \begin{ExamAnsRateInfo}{}{}{}{}
        \end{ExamAnsRateInfo}
        \begin{QBODY}
            下列五個數中,何者為最小?
            \begin{QOPS}
                \QOP ${{2}^{\frac{1}{3}}}$
                \QOP ${{\left( \frac{1}{8} \right)}^{-2}}$
                \QOP $2^{-\,\frac{1}{4}}$
                \QOP ${{\left( \frac{1}{2} \right)}^{\frac{1}{2}}}$
                \QOP $8^{-\,\frac{1}{3}}$
            \end{QOPS}
        \end{QBODY}
        \begin{QFROMS}
        \end{QFROMS}
        \begin{QTAGS}\end{QTAGS}
        \begin{QANS}
            (5)
        \end{QANS}
        \begin{QSOLLIST}
        \end{QSOLLIST}
        \begin{QEMPTYSPACE}
        \end{QEMPTYSPACE}
    \end{QUESTION}
    \begin{QUESTION}
        \begin{ExamInfo}{88}{學測}{單選}{3}
        \end{ExamInfo}
        \begin{ExamAnsRateInfo}{}{}{}{}
        \end{ExamAnsRateInfo}
        \begin{QBODY}
            圖一為一正立方體,$A,B,C$分別為所在的邊之中點,通過$A,B,C$三點的平面與此立方體表面相截,問下列何者為其截痕的形狀?
            \begin{QOPS}
                \QOP 直角三角形
                \QOP 非直角的三角形
                \QOP 正方形
                \QOP 非正方形的長方形
                \QOP 六邊形
            \end{QOPS}
            %TODO: 補圖
        \end{QBODY}
        \begin{QFROMS}
        \end{QFROMS}
        \begin{QTAGS}\end{QTAGS}
        \begin{QANS}
            (4)
        \end{QANS}
        \begin{QSOLLIST}
        \end{QSOLLIST}
        \begin{QEMPTYSPACE}
        \end{QEMPTYSPACE}
    \end{QUESTION}
\end{QUESTIONS}\begin{QUESTIONS}
    \begin{QUESTION}
        \begin{ExamInfo}{88}{學測}{多選}{4}
        \end{ExamInfo}
        \begin{ExamAnsRateInfo}{}{}{}{}
        \end{ExamAnsRateInfo}
        \begin{QBODY}
            設$\triangle ABC$的三頂點A,B,C所對邊的邊長分別為a,b,c,$\overline{AH}$為高,則$\overline{AH}$之長為
            \begin{QOPS}
                \QOP $b\cdot \sin B$
                \QOP $c\cdot \sin C$
                \QOP $b\cdot \sin C$
                \QOP $c\cdot \sin B$
                \QOP $a\cdot \sin A$
            \end{QOPS}
        \end{QBODY}
        \begin{QFROMS}
        \end{QFROMS}
        \begin{QTAGS}\end{QTAGS}
        \begin{QANS}
            (3)(4)
        \end{QANS}
        \begin{QSOLLIST}
        \end{QSOLLIST}
        \begin{QEMPTYSPACE}
        \end{QEMPTYSPACE}
    \end{QUESTION}
    \begin{QUESTION}
        \begin{ExamInfo}{88}{學測}{多選}{5}
        \end{ExamInfo}
        \begin{ExamAnsRateInfo}{}{}{}{}
        \end{ExamAnsRateInfo}
        \begin{QBODY}
            試選出正確的選項:
            \begin{QOPS}
                \QOP $0.3\overline{43}$不是有理數
                \QOP $0.\overline{34}>\frac{1}{3}$	
                \QOP $0.\overline{34}>0.343$
                \QOP $0.\overline{34}<0.35$
                \QOP $0.\overline{34}=0.3\overline{43}$
            \end{QOPS}    
        \end{QBODY}
        \begin{QFROMS}
        \end{QFROMS}
        \begin{QTAGS}\end{QTAGS}
        \begin{QANS}
            (2)(3)(4)(5)
        \end{QANS}
        \begin{QSOLLIST}
        \end{QSOLLIST}
        \begin{QEMPTYSPACE}
        \end{QEMPTYSPACE}
    \end{QUESTION}
    \begin{QUESTION}
        \begin{ExamInfo}{88}{學測}{多選}{6}
        \end{ExamInfo}
        \begin{ExamAnsRateInfo}{}{}{}{}
        \end{ExamAnsRateInfo}
        \begin{QBODY}
            三次方程式${{x}^{3}}+{{x}^{2}}-2x-1=0$在下列那些連續整數之間有根?
            \begin{QOPS}
                \QOP $–2$與$–1$之間
                \QOP $–1$與$0$之間
                \QOP $0$與$1$之間
                \QOP $1$與$2$之間
                \QOP $2$與$3$之間
            \end{QOPS}
        \end{QBODY}
        \begin{QFROMS}
        \end{QFROMS}
        \begin{QTAGS}\end{QTAGS}
        \begin{QANS}
            (1)(2)(4)
        \end{QANS}
        \begin{QSOLLIST}
        \end{QSOLLIST}
        \begin{QEMPTYSPACE}
        \end{QEMPTYSPACE}
    \end{QUESTION}
    \begin{QUESTION}
        \begin{ExamInfo}{88}{學測}{多選}{7}
        \end{ExamInfo}
        \begin{ExamAnsRateInfo}{}{}{}{}
        \end{ExamAnsRateInfo}
        \begin{QBODY}
            關於橢圓$\Gamma: \sqrt{{{(x-1)}^{2}}+{{(y-2)}^{2}}}+\sqrt{{{(x+1)}^{2}}+{{(y+2)}^{2}}}=6$,下列何者為真?
            \begin{QOPS}
                \QOP $\,(0\,,\,0)$是 $\Gamma$的中心
                \QOP $\,(1\,,\,2)$,$(-1,-2)$為 $\Gamma$的焦點
                \QOP $\Gamma$的短軸為4
                \QOP $\Gamma$對稱於直線$x=y$
                \QOP $\Gamma$對稱於$(1\,,\,2)$與$(-1,-2)$的連線
            \end{QOPS}
        \end{QBODY}
        \begin{QFROMS}
        \end{QFROMS}
        \begin{QTAGS}\end{QTAGS}
        \begin{QANS}
            (1)(2)(3)(5)
        \end{QANS}
        \begin{QSOLLIST}
        \end{QSOLLIST}
        \begin{QEMPTYSPACE}
        \end{QEMPTYSPACE}
    \end{QUESTION}
    \begin{QUESTION}
        \begin{ExamInfo}{88}{學測}{多選}{8}
        \end{ExamInfo}
        \begin{ExamAnsRateInfo}{}{}{}{}
        \end{ExamAnsRateInfo}
        \begin{QBODY}
            下列各選項中的行列式,那些與行列式$\left| \begin{matrix}
                {{a}_{1}} & {{a}_{2}} & {{a}_{3}}  \\
                {{b}_{1}} & {{b}_{2}} & {{b}_{3}}  \\
                {{c}_{1}} & {{c}_{2}} & {{c}_{3}}  \\
             \end{matrix} \right|$相等?
             \begin{QOPS}
                \QOP $\left| \begin{matrix}
                    {{a}_{1}} & {{a}_{2}} & {{a}_{3}}  \\
                    {{c}_{1}} & {{c}_{2}} & {{c}_{3}}  \\
                    {{b}_{1}} & {{b}_{2}} & {{b}_{3}}  \\
                \end{matrix} \right|\ \ \ \ \ \ \ \ \ \ \ \ \ \ \ \ \ \ \ \ \ \ \ \ \ \ \ \ \ $
                \QOP $\left| \begin{matrix}
                    {{a}_{1}} & {{b}_{1}} & {{c}_{1}}  \\
                    {{a}_{2}} & {{b}_{2}} & {{c}_{2}}  \\
                    {{a}_{3}} & {{b}_{3}} & {{c}_{3}}  \\
                \end{matrix} \right|$
                \QOP $\left| \begin{matrix}
                    {{a}_{1}} & {{a}_{2}} & {{a}_{3}}  \\
                    {{b}_{1}}-{{c}_{1}} & {{b}_{2}}-{{c}_{2}} & {{b}_{3}}-{{c}_{3}}  \\
                    {{c}_{1}} & {{c}_{2}} & {{c}_{3}}  \\
                \end{matrix} \right|$	
                \QOP $\left| \begin{matrix}
                    {{a}_{1}} & {{a}_{2}} & {{a}_{3}}  \\
                    {{b}_{1}}\cdot {{c}_{1}} & {{b}_{2}}\cdot {{c}_{2}} & {{b}_{3}}\cdot {{c}_{3}}  \\
                    {{c}_{1}} & {{c}_{2}} & {{c}_{3}}  \\
                \end{matrix} \right|$
                \QOP $\left| \begin{matrix}
                    {{a}_{3}} & {{a}_{2}} & {{a}_{1}}  \\
                    {{b}_{3}} & {{b}_{2}} & {{b}_{1}}  \\
                    {{c}_{3}} & {{c}_{2}} & {{c}_{1}}  \\
                \end{matrix} \right|$
            \end{QOPS}
        \end{QBODY}
        \begin{QFROMS}
        \end{QFROMS}
        \begin{QTAGS}\end{QTAGS}
        \begin{QANS}
            (2)(3)
        \end{QANS}
        \begin{QSOLLIST}
        \end{QSOLLIST}
        \begin{QEMPTYSPACE}
        \end{QEMPTYSPACE}
    \end{QUESTION}
    \begin{QUESTION}
        \begin{ExamInfo}{88}{學測}{多選}{9}
        \end{ExamInfo}
        \begin{ExamAnsRateInfo}{}{}{}{}
        \end{ExamAnsRateInfo}
        \begin{QBODY}
            測量一物件的長度9次,得其長(公尺)為\\
            2.43,	2.46,	2.41,	2.45,	2.44,	2.48,	2.46,	2.47,	2.45\\
            將上面的數據每一個都乘以100,再減去240得一組新數據為\\
            3,	6,	1,	5,	4,	8,	6,	7,	5\\
            問下列選項,何者為真?
            \begin{QOPS}
                \QOP 新數據的算術平均數為5
                \QOP 新數據的標準差為2
                \QOP 原數據的算術平均數為2.45
                \QOP 原數據的標準差為0.2
                \QOP 原數據的中位數為2.45
            \end{QOPS}
        \end{QBODY}
        \begin{QFROMS}
        \end{QFROMS}
        \begin{QTAGS}\end{QTAGS}
        \begin{QANS}
            (1)(2)(3)(5)
        \end{QANS}
        \begin{QSOLLIST}
        \end{QSOLLIST}
        \begin{QEMPTYSPACE}
        \end{QEMPTYSPACE}
    \end{QUESTION}
    \begin{QUESTION}
        \begin{ExamInfo}{88}{學測}{多選}{10}
        \end{ExamInfo}
        \begin{ExamAnsRateInfo}{}{}{}{}
        \end{ExamAnsRateInfo}
        \begin{QBODY}
            圖為一正立方體,試問下列何者為真? %TODO:補圖
            (1)$\lvec{EA} \cdot \lvec{EG}=0$
            (2)$\lvec{ED} \cdot \lvec{EF}=0$
            (3)$\lvec{EF}+\lvec{EH}=\lvec{AC}$
            (4)$\lvec{EC} \cdot \lvec{AG}=0$
            (5)$\lvec{EF}+\lvec{EA}+\lvec{EH}=\lvec{EC}$            
        \end{QBODY}
        \begin{QFROMS}
        \end{QFROMS}
        \begin{QTAGS}\end{QTAGS}
        \begin{QANS}
            (1)(2)(3)(5)
        \end{QANS}
        \begin{QSOLLIST}
        \end{QSOLLIST}
        \begin{QEMPTYSPACE}
        \end{QEMPTYSPACE}
    \end{QUESTION}
\end{QUESTIONS}\begin{QUESTIONS}
    \begin{QUESTION}
        \begin{ExamInfo}{88}{學測}{選填}{11}
        \end{ExamInfo}
        \begin{ExamAnsRateInfo}{}{}{}{}
        \end{ExamAnsRateInfo}
        \begin{QBODY}
            一個正三角形的面積為$36$,今截去三個角(如圖),使成為正六邊形,此正六邊形的面積為$\TCNBOX{\TCN\TCN}$。
            %TODO:補圖
        \end{QBODY}
        \begin{QFROMS}
        \end{QFROMS}
        \begin{QTAGS}\end{QTAGS}
        \begin{QANS}
            $24$
        \end{QANS}
        \begin{QSOLLIST}
        \end{QSOLLIST}
        \begin{QEMPTYSPACE}
        \end{QEMPTYSPACE}
    \end{QUESTION}
    \begin{QUESTION}
        \begin{ExamInfo}{88}{學測}{選填}{12}
        \end{ExamInfo}
        \begin{ExamAnsRateInfo}{}{}{}{}
        \end{ExamAnsRateInfo}
        \begin{QBODY}
            本金100元,年利率6 \%,每半年複利一次,五年期滿,共得本利和為 $\TCNBOX{\TCN\TCN\TCN }$ 元。(元以下四捨五入)
        \end{QBODY}
        \begin{QFROMS}
        \end{QFROMS}
        \begin{QTAGS}\end{QTAGS}
        \begin{QANS}
            $134$
        \end{QANS}
        \begin{QSOLLIST}
        \end{QSOLLIST}
        \begin{QEMPTYSPACE}
        \end{QEMPTYSPACE}
    \end{QUESTION}
    \begin{QUESTION}
        \begin{ExamInfo}{88}{學測}{選填}{13}
        \end{ExamInfo}
        \begin{ExamAnsRateInfo}{}{}{}{}
        \end{ExamAnsRateInfo}
        \begin{QBODY}
            一位海盜欲將三件珠寶埋藏在一個島上的三個地方,海盜就以島上的一棵大王椰子樹為中心,由大王椰子樹向東走$12$步埋他的第一件珠寶;
            由大王椰子樹向東走$4$步,再往北走$a$步埋他的第二件珠寶;最後由大王椰子樹向東走$a$步,再往南走$8$步埋他的第三件珠寶。事隔多
            年之後,海盜僅記得$a>0$及埋藏珠寶的三個地方在同一直線上。那麼$a =\TCNBOX{\TCN\TCN}$。
        \end{QBODY}
        \begin{QFROMS}
        \end{QFROMS}
        \begin{QTAGS}\end{QTAGS}
        \begin{QANS}
            $16$
        \end{QANS}
        \begin{QSOLLIST}
        \end{QSOLLIST}
        \begin{QEMPTYSPACE}
        \end{QEMPTYSPACE}
    \end{QUESTION}
    \begin{QUESTION}
        \begin{ExamInfo}{88}{學測}{選填}{14}
        \end{ExamInfo}
        \begin{ExamAnsRateInfo}{}{}{}{}
        \end{ExamAnsRateInfo}
        \begin{QBODY}
            設$0<\theta  < \frac{\pi }{4}$,且$2+\sqrt{3}$為${{x}^{2}}-(\tan \theta +\cot \theta )\,x+1=0$的一根,則
            $\tan \theta = \TCNBOX{ \TCN-\sqrt{\TCN} }$。
        \end{QBODY}
        \begin{QFROMS}
        \end{QFROMS}
        \begin{QTAGS}\end{QTAGS}
        \begin{QANS}
            $2-\sqrt{3}$
        \end{QANS}
        \begin{QSOLLIST}
        \end{QSOLLIST}
        \begin{QEMPTYSPACE}
        \end{QEMPTYSPACE}
    \end{QUESTION}
    \begin{QUESTION}
        \begin{ExamInfo}{88}{學測}{選填}{15}
        \end{ExamInfo}
        \begin{ExamAnsRateInfo}{}{}{}{}
        \end{ExamAnsRateInfo}
        \begin{QBODY}
            有一輪子,半徑50公分,讓它在地上滾動200公分的長度,問輪子繞軸轉動$\TCNBOX{\TCN\TCN\TCN}$度。(度以下四捨五入)
        \end{QBODY}
        \begin{QFROMS}
        \end{QFROMS}
        \begin{QTAGS}\end{QTAGS}
        \begin{QANS}
            $229$
        \end{QANS}
        \begin{QSOLLIST}
        \end{QSOLLIST}
        \begin{QEMPTYSPACE}
        \end{QEMPTYSPACE}
    \end{QUESTION}
    \begin{QUESTION}
        \begin{ExamInfo}{88}{學測}{選填}{16}
        \end{ExamInfo}
        \begin{ExamAnsRateInfo}{}{}{}{}
        \end{ExamAnsRateInfo}
        \begin{QBODY}
            在$\triangle ABC$中,已知$\angle C=60^\circ$,$\overline{AC}=3000$公尺,$\overline{BC}=2000$公尺,則 $\angle A$為
            $\TCNBOX{\TCN\TCN}$度。(度以下四捨五入)
            (參考資料:$\sqrt{3}\approx 1.732\ ,\ \ \ \sqrt{7}\approx 2.646\ ,\ \ \ \sqrt{21}\approx 4.583$)
            
        \end{QBODY}
        \begin{QFROMS}
        \end{QFROMS}
        \begin{QTAGS}\end{QTAGS}
        \begin{QANS}
            $41$
        \end{QANS}
        \begin{QSOLLIST}
        \end{QSOLLIST}
        \begin{QEMPTYSPACE}
        \end{QEMPTYSPACE}
    \end{QUESTION}
    \begin{QUESTION}
        \begin{ExamInfo}{88}{學測}{選填}{17}
        \end{ExamInfo}
        \begin{ExamAnsRateInfo}{}{}{}{}
        \end{ExamAnsRateInfo}
        \begin{QBODY}
            袋子裡有3個球,2個球上標1元,1個球上標5元。從袋中任取2個球,即可得到兩個球所標錢數的總和,則此玩法所得錢數的期望值是
            $\TCNBOX{\FR{\TCN\TCN}{3}}$
               元。

        \end{QBODY}
        \begin{QFROMS}
        \end{QFROMS}
        \begin{QTAGS}\end{QTAGS}
        \begin{QANS}
            $\FR{14}{3}$
        \end{QANS}
        \begin{QSOLLIST}
        \end{QSOLLIST}
        \begin{QEMPTYSPACE}
        \end{QEMPTYSPACE}
    \end{QUESTION}
    \begin{QUESTION}
        \begin{ExamInfo}{88}{學測}{選填}{18}
        \end{ExamInfo}
        \begin{ExamAnsRateInfo}{}{}{}{}
        \end{ExamAnsRateInfo}
        \begin{QBODY}
            有一片長方形牆壁,尺寸為$12\times 1$(即:長12單位長,寬1單位長)。若有許多白色及咖啡色壁磚,白色壁磚尺寸為 $2\times 1$,咖啡色壁磚尺寸為$4\times 1$,用這些壁磚貼滿此長方形,問可貼成幾種不同的圖案?
            $\TCNBOX{\TCN\TCN}$ 種。
            
        \end{QBODY}
        \begin{QFROMS}
        \end{QFROMS}
        \begin{QTAGS}\end{QTAGS}
        \begin{QANS}
            $13$
        \end{QANS}
        \begin{QSOLLIST}
        \end{QSOLLIST}
        \begin{QEMPTYSPACE}
        \end{QEMPTYSPACE}
    \end{QUESTION}
    \begin{QUESTION}
        \begin{ExamInfo}{88}{學測}{選填}{19}
        \end{ExamInfo}
        \begin{ExamAnsRateInfo}{}{}{}{}
        \end{ExamAnsRateInfo}
        \begin{QBODY}
            擲3粒公正骰子,問恰好有兩粒點數相同的機率為$\TCNBOX{\FR{\TCN\TCN}{6^3}}$。
        \end{QBODY}
        \begin{QFROMS}
        \end{QFROMS}
        \begin{QTAGS}\end{QTAGS}
        \begin{QANS}
            $\FR{5}{12}$
        \end{QANS}
        \begin{QSOLLIST}
        \end{QSOLLIST}
        \begin{QEMPTYSPACE}
        \end{QEMPTYSPACE}
    \end{QUESTION}
    \begin{QUESTION}
        \begin{ExamInfo}{88}{學測}{選填}{20}
        \end{ExamInfo}
        \begin{ExamAnsRateInfo}{}{}{}{}
        \end{ExamAnsRateInfo}
        \begin{QBODY}
            在空間中,連接點$P(2,1,3)$與點$Q(4,5,5)$的線段PQ之垂直平分面為 $\TCNBOX{\TCN x + \TCN y + \TCN z =13 }$
        \end{QBODY}
        \begin{QFROMS}
        \end{QFROMS}
        \begin{QTAGS}\end{QTAGS}
        \begin{QANS}
            $x+2y+z=13$
        \end{QANS}
        \begin{QSOLLIST}
        \end{QSOLLIST}
        \begin{QEMPTYSPACE}
        \end{QEMPTYSPACE}
    \end{QUESTION}
\end{QUESTIONS}