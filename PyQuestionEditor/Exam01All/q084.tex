% !TEX encoding = UTF-8 Unicode
% !TEX TS-program = xelatex 
\begin{QUESTIONS}
    \begin{QUESTION}
        \begin{ExamInfo}{84}{學測}{單選}{1}
        \end{ExamInfo}
        \begin{ExamAnsRateInfo}{}{}{}{}
        \end{ExamAnsRateInfo}
        \begin{QBODY}
            %TODO:補圖
            圖1中A、B、C、D、E為坐標平面上的五個點。將這五點的坐標$\left( x,\,y \right)$分別代入$x-y=k$,問那一點所得的$k$值最大?
            \begin{QOPS}
                \QOP $A$
                \QOP $B$
                \QOP $C$
                \QOP $D$
                \QOP $E$
            \end{QOPS}
        \end{QBODY}
        \begin{QFROMS}
        \end{QFROMS}
        \begin{QTAGS}\end{QTAGS}
        \begin{QANS}
            (5)
        \end{QANS}
        \begin{QSOLLIST}
        \end{QSOLLIST}
        \begin{QEMPTYSPACE}
        \end{QEMPTYSPACE}
    \end{QUESTION}
    \begin{QUESTION}
        \begin{ExamInfo}{84}{學測}{單選}{2}
        \end{ExamInfo}
        \begin{ExamAnsRateInfo}{}{}{}{}
        \end{ExamAnsRateInfo}
        \begin{QBODY}
            若將$\frac{1}{4369}+\frac{1}{5911}$化為最簡分數,則其分母為何?
            \begin{QOPS}
            \QOP 100487
            \QOP 100489
            \QOP 10280
            \QOP 25825159
            \QOP 25825161
            \end{QOPS}            
        \end{QBODY}
        \begin{QFROMS}
        \end{QFROMS}
        \begin{QTAGS}\end{QTAGS}
        \begin{QANS}
            (1)
        \end{QANS}
        \begin{QSOLLIST}
        \end{QSOLLIST}
        \begin{QEMPTYSPACE}
        \end{QEMPTYSPACE}
    \end{QUESTION}
    \begin{QUESTION}
        \begin{ExamInfo}{84}{學測}{單選}{3}
        \end{ExamInfo}
        \begin{ExamAnsRateInfo}{}{}{}{}
        \end{ExamAnsRateInfo}
        \begin{QBODY}
            圖2表示長方形垛的疊法:
            %TODO:補圖
            某水果販將橘子堆成長方形垛。若最底層長邊有10個橘子,短邊有5個,則此長方形垛最多有幾個橘子?
            \begin{QOPS}
            \QOP 110
            \QOP 120
            \QOP 130
            \QOP 140
            \QOP 150   
            \end{QOPS}         
        \end{QBODY}
        \begin{QFROMS}
        \end{QFROMS}
        \begin{QTAGS}\end{QTAGS}
        \begin{QANS}
            (3)
        \end{QANS}
        \begin{QSOLLIST}
        \end{QSOLLIST}
        \begin{QEMPTYSPACE}
        \end{QEMPTYSPACE}
    \end{QUESTION}
    \begin{QUESTION}
        \begin{ExamInfo}{84}{學測}{單選}{4}
        \end{ExamInfo}
        \begin{ExamAnsRateInfo}{}{}{}{}
        \end{ExamAnsRateInfo}
        \begin{QBODY}
            以下選項所列的各平面,那一個平面與球${{x}^{2}}+{{y}^{2}}+{{z}^{2}}-2x+4y+2z-19=0$相交所成的圓之面積最大?
            \begin{QOPS}
            \QOP $x+y+z=0$
            \QOP $z=-1$
            \QOP $y=1$
            \QOP $x=2$
            \QOP $x=2y$
        \end{QOPS}
            
        \end{QBODY}
        \begin{QFROMS}
        \end{QFROMS}
        \begin{QTAGS}\end{QTAGS}
        \begin{QANS}
            (2)
        \end{QANS}
        \begin{QSOLLIST}
        \end{QSOLLIST}
        \begin{QEMPTYSPACE}
        \end{QEMPTYSPACE}
    \end{QUESTION}
    \begin{QUESTION}
        \begin{ExamInfo}{84}{學測}{單選}{5}
        \end{ExamInfo}
        \begin{ExamAnsRateInfo}{}{}{}{}
        \end{ExamAnsRateInfo}
        \begin{QBODY}
            我國自用小汽車的牌照號碼,前兩位為大寫英文字母,後四位為數字,例如$AB-0950$。若最後一位數字不用4,且後四位數字沒有0000這個號碼,那麼我國可能有的自用小汽車牌照號碼有多少個?
            \begin{QOPS}
            \QOP $26\times 25\times \left( 4320-1 \right)$
            \QOP $ 26\times 25\times 4320-1$ 
            \QOP $26\times 25\times \left( 5040-1 \right)$
            \QOP $26\times 26\times \left( 9000-1 \right)$
            \QOP $26\times 26\times 9000-1$
        \end{QOPS}
        \end{QBODY}
        \begin{QFROMS}
        \end{QFROMS}
        \begin{QTAGS}\end{QTAGS}
        \begin{QANS}
            (4)
        \end{QANS}
        \begin{QSOLLIST}
        \end{QSOLLIST}
        \begin{QEMPTYSPACE}
        \end{QEMPTYSPACE}
    \end{QUESTION}
    \begin{QUESTION}
        \begin{ExamInfo}{84}{學測}{單選}{6}
        \end{ExamInfo}
        \begin{ExamAnsRateInfo}{}{}{}{}
        \end{ExamAnsRateInfo}
        \begin{QBODY}
            某肥皂廠商欲推出一種新產品,在上市前以不同的單價x(單位:十元)調查市場的需求量y(單位:萬盒)。調查結果如下:
            \begin{tabular}{|c|c|c|c|c|c|}
                \hline 
                $x$ &	8&	9&	10&	11&	12 \\ 
                \hline 
                $y$	&  11&	12&	10&	8&	9  \\ 
                \hline 
            \end{tabular} 
            
            
            問$x$和$y$的相關係數最接近下列那一個值?
            \begin{QOPS}
            \QOP $\frac{4}{5}$
            \QOP $\frac{2}{5}$
            \QOP $0$
            \QOP $-\frac{2}{5}$
            \QOP $-\frac{4}{5}$
        \end{QOPS}
        \end{QBODY}
        \begin{QFROMS}
        \end{QFROMS}
        \begin{QTAGS}\end{QTAGS}
        \begin{QANS}
            (5)
        \end{QANS}
        \begin{QSOLLIST}
        \end{QSOLLIST}
        \begin{QEMPTYSPACE}
        \end{QEMPTYSPACE}
    \end{QUESTION}
    \begin{QUESTION}
        \begin{ExamInfo}{84}{學測}{單選}{7}
        \end{ExamInfo}
        \begin{ExamAnsRateInfo}{}{}{}{}
        \end{ExamAnsRateInfo}
        \begin{QBODY}
            設$m$為實數,若二次函數$y=m{{x}^{2}}+10x+m+6$的圖形在直線$y=2$的上方,則$m$的範圍為何?
            \begin{QOPS}
                \QOP $m>0$
                \QOP $m>-2+\sqrt{29}$
                \QOP $0<m<-2+\sqrt{29}$
                \QOP $-2-\sqrt{29}<m<-2+\sqrt{29}$
                \QOP $m>-2+\sqrt{29}$或$m<-2-\sqrt{29}$
            \end{QOPS}
        \end{QBODY}
        \begin{QFROMS}
        \end{QFROMS}
        \begin{QTAGS}\end{QTAGS}
        \begin{QANS}
            (2)
        \end{QANS}
        \begin{QSOLLIST}
        \end{QSOLLIST}
        \begin{QEMPTYSPACE}
        \end{QEMPTYSPACE}
    \end{QUESTION}
\end{QUESTIONS}
\begin{QUESTIONS}
    \begin{QUESTION}
        \begin{ExamInfo}{84}{學測}{多選}{8}
        \end{ExamInfo}
        \begin{ExamAnsRateInfo}{}{}{}{}
        \end{ExamAnsRateInfo}
        \begin{QBODY}
            下面有五組函數,那些組的兩個函數,其圖形互相對稱於y軸?
            \begin{QOPS}
            \QOP $y={{\left( \frac{1}{2} \right)}^{3x}}$和$y={{2}^{3x}}$
            \QOP $y={{2}^{3x}}$和$y={{3}^{2x}}$
            \QOP $y={{x}^{2}}$和$y=-{{x}^{2}}$
            \QOP $y=\log x$ 和$y=\log \left( -x \right)$ 
            \QOP $y=\cos x$ 和$y=\sin \,\left( x-\frac{\pi }{2} \right)$ 
        \end{QOPS}
        \end{QBODY}
        \begin{QFROMS}
        \end{QFROMS}
        \begin{QTAGS}\end{QTAGS}
        \begin{QANS}
            (1)(4)
        \end{QANS}
        \begin{QSOLLIST}
        \end{QSOLLIST}
        \begin{QEMPTYSPACE}
        \end{QEMPTYSPACE}
    \end{QUESTION}
    \begin{QUESTION}
        \begin{ExamInfo}{84}{學測}{多選}{9}
        \end{ExamInfo}
        \begin{ExamAnsRateInfo}{}{}{}{}
        \end{ExamAnsRateInfo}
        \begin{QBODY}
            $ \cos 74{}^\circ -\cos 14{}^\circ  $ 等於下列那些式子?
            \begin{QOPS}
            \QOP $\cos 60{}^\circ  $ 
            \QOP $2\sin 30{}^\circ \sin 44{}^\circ  $ 
            \QOP $2\cos 30{}^\circ \cos 44{}^\circ  $ 
            \QOP $\sin 16{}^\circ -\sin 76{}^\circ  $ 
            \QOP $\sin 164{}^\circ +\cos 166{}^\circ $  
        \end{QOPS}
    

        \end{QBODY}
        \begin{QFROMS}
        \end{QFROMS}
        \begin{QTAGS}\end{QTAGS}
        \begin{QANS}
            (4)(5)
        \end{QANS}
        \begin{QSOLLIST}
        \end{QSOLLIST}
        \begin{QEMPTYSPACE}
        \end{QEMPTYSPACE}
    \end{QUESTION}
    \begin{QUESTION}
        \begin{ExamInfo}{84}{學測}{多選}{10}
        \end{ExamInfo}
        \begin{ExamAnsRateInfo}{}{}{}{}
        \end{ExamAnsRateInfo}
        \begin{QBODY}
            	已知等軸雙曲線$\Gamma $的一條漸近線為$x-y=0$,中心的坐標為$\left( 1,\,1 \right)$,且$\Gamma $通過點$\left( 3,\,0 \right)$。試問下列敘述那些是正確的?
            \begin{QOPS}
            \QOP $\Gamma $的兩條漸近線互相垂直
            \QOP $x+y=0$為$\Gamma $的另外一條漸近線
            \QOP $\Gamma $的貫軸在直線$y=1$上
            \QOP 點$\left( 1,\,\sqrt{3}-1 \right)$為$\Gamma $的一個頂點
            \QOP 點$\left( 1,\,\sqrt{6}-1 \right)$為$\Gamma $的一個焦點
        \end{QOPS}
        \end{QBODY}
        \begin{QFROMS}
        \end{QFROMS}
        \begin{QTAGS}\end{QTAGS}
        \begin{QANS}
            (1)(3)
        \end{QANS}
        \begin{QSOLLIST}
        \end{QSOLLIST}
        \begin{QEMPTYSPACE}
        \end{QEMPTYSPACE}
    \end{QUESTION}
    \begin{QUESTION}
        \begin{ExamInfo}{84}{學測}{多選}{11}
        \end{ExamInfo}
        \begin{ExamAnsRateInfo}{}{}{}{}
        \end{ExamAnsRateInfo}
        \begin{QBODY}
            圖3中ABCD為正四面體,M為$\overline{CD}$的中點,試問下列那些敘述是正確的?
            %TODO:補圖
            \begin{QOPS}
            \QOP 直線CD與平面ABM垂直
            \QOP 向量AB與向量CD垂直
            \QOP $\angle AMB>\angle ADB$
            \QOP 平面ACD與平面BCD的二面角(銳角)大於${{60}^{{}^\circ }}$
            \QOP $\overline{BA}=\overline{BM}$
            \end{QOPS}
        \end{QBODY}
        \begin{QFROMS}
        \end{QFROMS}
        \begin{QTAGS}\end{QTAGS}
        \begin{QANS}
            (1)(2)(3)(4)
        \end{QANS}
        \begin{QSOLLIST}
        \end{QSOLLIST}
        \begin{QEMPTYSPACE}
        \end{QEMPTYSPACE}
    \end{QUESTION}
\end{QUESTIONS}\begin{QUESTIONS}
    \begin{QUESTION}
        \begin{ExamInfo}{84}{學測}{選填}{12}
        \end{ExamInfo}
        \begin{ExamAnsRateInfo}{}{}{}{}
        \end{ExamAnsRateInfo}
        \begin{QBODY}
            已知兩拋物線$x={{y}^{2}}+3y-2$與$y={{x}^{2}}+kx+19$有交點,其中兩個交點在直線$x+y=3$上,則k的值等於多少?$\TCNBOX{\TCN\TCN\TCN}$
        \end{QBODY}
        \begin{QFROMS}
        \end{QFROMS}
        \begin{QTAGS}\end{QTAGS}
        \begin{QANS}
            $-11$
        \end{QANS}
        \begin{QSOLLIST}
        \end{QSOLLIST}
        \begin{QEMPTYSPACE}
        \end{QEMPTYSPACE}
    \end{QUESTION}
    \begin{QUESTION}
        \begin{ExamInfo}{84}{學測}{選填}{13}
        \end{ExamInfo}
        \begin{ExamAnsRateInfo}{}{}{}{}
        \end{ExamAnsRateInfo}
        \begin{QBODY}
            已知二多項式
            $P\left( x \right)=1+2x+3{{x}^{2}}+\cdots +10{{x}^{9}}+11{{x}^{10}}=\sum\limits_{i=0}^{10}{\left( i+1 \right){{x}^{i}}}$,
            與$Q\left( x \right)=1+3{{x}^{2}}+5{{x}^{4}}+\cdots +9{{x}^{8}}+11{{x}^{10}}=\sum\limits_{i=0}^{5}{\left( 2i+1 \right){{x}^{2i}}}$。
            則$P\left( x \right)$和$Q\left( x \right)$的乘積中,${{x}^{9}}$的係數為\TCNBOX{}。
            
        \end{QBODY}
        \begin{QFROMS}
        \end{QFROMS}
        \begin{QTAGS}\end{QTAGS}
        \begin{QANS}
            $110$
        \end{QANS}
        \begin{QSOLLIST}
        \end{QSOLLIST}
        \begin{QEMPTYSPACE}
        \end{QEMPTYSPACE}
    \end{QUESTION}
    \begin{QUESTION}
        \begin{ExamInfo}{84}{學測}{選填}{14}
        \end{ExamInfo}
        \begin{ExamAnsRateInfo}{}{}{}{}
        \end{ExamAnsRateInfo}
        \begin{QBODY}
            林先生和陳小姐一起到遊樂場玩打靶遊戲。林先生射擊命中靶的機率是2/5,陳小姐的機率是1/2。林先生先射,陳小姐後射;林先生射中與否不會影響陳小姐的命中率。若他們兩人向靶各射一次,問只有陳小姐射中的機率為多少?$\FR{\TCN}{\TCN\TCN}$
        \end{QBODY}
        \begin{QFROMS}
        \end{QFROMS}
        \begin{QTAGS}\end{QTAGS}
        \begin{QANS}
            $\FR{3}{10}$
        \end{QANS}
        \begin{QSOLLIST}
        \end{QSOLLIST}
        \begin{QEMPTYSPACE}
        \end{QEMPTYSPACE}
    \end{QUESTION}
    \begin{QUESTION}
        \begin{ExamInfo}{84}{學測}{選填}{15}
        \end{ExamInfo}
        \begin{ExamAnsRateInfo}{}{}{}{}
        \end{ExamAnsRateInfo}
        \begin{QBODY}
            設n為自然數,則滿足${{10}^{n-1}}>{{9}^{n}}$的n值中最小的為
            $\TCNBOX{\TCN\TCN}$。
            
        \end{QBODY}
        \begin{QFROMS}
        \end{QFROMS}
        \begin{QTAGS}\end{QTAGS}
        \begin{QANS}
            $22$
        \end{QANS}
        \begin{QSOLLIST}
        \end{QSOLLIST}
        \begin{QEMPTYSPACE}
        \end{QEMPTYSPACE}
    \end{QUESTION}
    \begin{QUESTION}
        \begin{ExamInfo}{84}{學測}{選填}{16}
        \end{ExamInfo}
        \begin{ExamAnsRateInfo}{}{}{}{}
        \end{ExamAnsRateInfo}
        \begin{QBODY}
            有四條直線${{L}_{1}}:x-y=1$,${{L}_{2}}:x+y=4$,${{L}_{3}}:8x+y=-10$和${{L}_{4}}:x=2$。這四條直線圍出一個四邊形。請問此四邊形較短的對角線長度為多少?$\TCNBOX{\TCN}$
        \end{QBODY}
        \begin{QFROMS}
        \end{QFROMS}
        \begin{QTAGS}\end{QTAGS}
        \begin{QANS}
            $5$
        \end{QANS}
        \begin{QSOLLIST}
        \end{QSOLLIST}
        \begin{QEMPTYSPACE}
        \end{QEMPTYSPACE}
    \end{QUESTION}
    \begin{QUESTION}
        \begin{ExamInfo}{84}{學測}{選填}{17}
        \end{ExamInfo}
        \begin{ExamAnsRateInfo}{}{}{}{}
        \end{ExamAnsRateInfo}
        \begin{QBODY}
            一汽艇在湖上沿直線前進,有人用儀器在岸上先測得汽艇在正前方偏左${{50}^{{}^\circ }}$,距離為200公尺。一分鐘後,於原地再測,知汽艇駛到正前方偏右${{70}^{{}^\circ }}$,距離為300公尺。那麼此艇在這一分鐘內行駛了 $\TCNBOX{\TCN\TCN\TCN\sqrt{\TCN\TCN}}$公尺。
        \end{QBODY}
        \begin{QFROMS}
        \end{QFROMS}
        \begin{QTAGS}\end{QTAGS}
        \begin{QANS}
            $100\sqrt{19}$
        \end{QANS}
        \begin{QSOLLIST}
        \end{QSOLLIST}
        \begin{QEMPTYSPACE}
        \end{QEMPTYSPACE}
    \end{QUESTION}
    \begin{QUESTION}
        \begin{ExamInfo}{84}{學測}{選填}{18}
        \end{ExamInfo}
        \begin{ExamAnsRateInfo}{}{}{}{}
        \end{ExamAnsRateInfo}
        \begin{QBODY}
            假設某鎮每年的人口數逐年成長,且成一等比數列。已知此鎮十年前有25萬人,現在有30萬人,那麼二十年後,此鎮人口應有$\TCNBOX{\TCN\TCN.\TCN}$萬人。(求到小數點後一位)
        \end{QBODY}
        \begin{QFROMS}
        \end{QFROMS}
        \begin{QTAGS}\end{QTAGS}
        \begin{QANS}
            $43.2$
        \end{QANS}
        \begin{QSOLLIST}
        \end{QSOLLIST}
        \begin{QEMPTYSPACE}
        \end{QEMPTYSPACE}
    \end{QUESTION}
    \begin{QUESTION}
        \begin{ExamInfo}{84}{學測}{選填}{19}
        \end{ExamInfo}
        \begin{ExamAnsRateInfo}{}{}{}{}
        \end{ExamAnsRateInfo}
        \begin{QBODY}
            	設$ f\left( x \right)={{\left( \sin x+\cos x \right)}^{2}}+4\,\left( \sin x+\cos x \right)$ ,則$f\left( x \right)$的最小值為$\TCNBOX{\TCN - \TCN \sqrt{\TCN}}$。

        \end{QBODY}
        \begin{QFROMS}
        \end{QFROMS}
        \begin{QTAGS}\end{QTAGS}
        \begin{QANS}
                        $2-4\sqrt{2}$
        \end{QANS}
        \begin{QSOLLIST}
        \end{QSOLLIST}
        \begin{QEMPTYSPACE}
        \end{QEMPTYSPACE}
    \end{QUESTION}
    \begin{QUESTION}
        \begin{ExamInfo}{84}{學測}{選填}{20}
        \end{ExamInfo}
        \begin{ExamAnsRateInfo}{}{}{}{}
        \end{ExamAnsRateInfo}
        \begin{QBODY}
            在空間坐標中,設xy平面為一鏡面。有一光線通過點$P\left( 1,\,2,\,1 \right)$,射向鏡面上的點$O\left( 0,\,0,\,0 \right)$,經鏡面反射後通過點R。若$\overline{OR}=2\overline{PO}$,則R點的坐標為$\TCNBOX{\TCN\TCN,\TCN\TCN,\TCN}$ 。
        \end{QBODY}
        \begin{QFROMS}
        \end{QFROMS}
        \begin{QTAGS}\end{QTAGS}
        \begin{QANS}
            $(-2,-4,2)$
        \end{QANS}
        \begin{QSOLLIST}
        \end{QSOLLIST}
        \begin{QEMPTYSPACE}
        \end{QEMPTYSPACE}
    \end{QUESTION}
\end{QUESTIONS}