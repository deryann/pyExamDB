% !TEX encoding = UTF-8 Unicode
% !TEX TS-program = xelatex 
\begin{QUESTIONS}
    \begin{QUESTION}
        \begin{ExamInfo}{89}{學測}{單選}{1}
        \end{ExamInfo}
        \begin{ExamAnsRateInfo}{}{}{}{}
        \end{ExamAnsRateInfo}
        \begin{QBODY}
            有一等腰三角形底邊為$10$,頂角$72^\circ$。下列何者可以表示腰長?
            \begin{QOPS}
                \QOP $5\cdot \sin 36{}^\circ $
                \QOP $5\cdot \tan 36{}^\circ $
                \QOP $5\cdot \cot 36{}^\circ $
                \QOP $5\cdot \sec 36{}^\circ $
                \QOP $5\cdot \csc 36{}^\circ $
            \end{QOPS}
        \end{QBODY}
        \begin{QFROMS}
        \end{QFROMS}
        \begin{QTAGS}\end{QTAGS}
        \begin{QANS}
		(5)
        \end{QANS}
        \begin{QSOLLIST}
        \end{QSOLLIST}
        \begin{QEMPTYSPACE}
        \end{QEMPTYSPACE}
    \end{QUESTION}
    \begin{QUESTION}
        \begin{ExamInfo}{89}{學測}{單選}{2}
        \end{ExamInfo}
        \begin{ExamAnsRateInfo}{}{}{}{}
        \end{ExamAnsRateInfo}
        \begin{QBODY}
            %TODO: 補圖
            在坐標平面上,根據方程式$x+5y-7=0,\ 2x+y+4=0,\ x-y-1=0$畫出三條直線$L_1$,$L_2$,$L_3$,如圖所示。試選出方程式與直線間正確的配置?
            \begin{QOPS}
                \QOP $L_1:x+5y-7=0$;$L_2:2x+y+4=0$;$L_3:x-y-1=0$
                \QOP $L_1:x-y-1=0$; $L_2:x+5y-7=0$;$L_3:2x+y+4=0$
                \QOP $L_1:2x+y+4=0$;$L_2:x+5y-7=0$;$L_3:x-y-1=0$
                \QOP $L_1:x-y-1=0$; $L_2:2x+y+4=0$;$L_3:x+5y-7=0$
                \QOP $L_1:2x+y+4=0$;$L_2:x-y-1=0$ ;$L_3:x+5y-7=0$
            \end{QOPS}
        \end{QBODY}
        \begin{QFROMS}
        \end{QFROMS}
        \begin{QTAGS}\end{QTAGS}
        \begin{QANS}
			(4)
        \end{QANS}
        \begin{QSOLLIST}
        \end{QSOLLIST}
        \begin{QEMPTYSPACE}
        \end{QEMPTYSPACE}
    \end{QUESTION}
    \begin{QUESTION}
        \begin{ExamInfo}{89}{學測}{單選}{3}
        \end{ExamInfo}
        \begin{ExamAnsRateInfo}{}{}{}{}
        \end{ExamAnsRateInfo}
        \begin{QBODY}
            下列5組資料(每組各有10筆)
            A:	1,	1,	1,	1,	1,	10,	10,	10,	10,	10
            B:	1,	1,	1,	1,	1,	5,	5,	5,	5,	5
            C:	4,	4,	4,	5,	5,	5,	5,	6,	6,	6
            D:	1,	1,	2,	2,	3,	3,	4,	4,	5,	5
            E:	1,	2,	3,	4,	5,	6,	7,	8,	9,	10
            試問哪一組資料的標準差最大?
            \begin{QOPS}
                \QOP A	
                \QOP B
                \QOP C	
                \QOP D
                \QOP E               
            \end{QOPS}
        \end{QBODY}
        \begin{QFROMS}
        \end{QFROMS}
        \begin{QTAGS}\end{QTAGS}
        \begin{QANS}
			(1)
        \end{QANS}
        \begin{QSOLLIST}
        \end{QSOLLIST}
        \begin{QEMPTYSPACE}
        \end{QEMPTYSPACE}
    \end{QUESTION}
    \begin{QUESTION}
        \begin{ExamInfo}{89}{學測}{單選}{4}
        \end{ExamInfo}
        \begin{ExamAnsRateInfo}{}{}{}{}
        \end{ExamAnsRateInfo}
        \begin{QBODY}
            如圖所示有5筆(X,Y)資料。試問:去掉哪一筆資料後,剩下來4筆資料的相關係數最大?
            \begin{QOPS}
                   \QOP A
                   \QOP B
                   \QOP C
                   \QOP D
                   \QOP E
            \end{QOPS}
        %TODO:補圖            
        \end{QBODY}
        \begin{QFROMS}
        \end{QFROMS}
        \begin{QTAGS}\end{QTAGS}
        \begin{QANS}
			(4)
        \end{QANS}
        \begin{QSOLLIST}
        \end{QSOLLIST}
        \begin{QEMPTYSPACE}
        \end{QEMPTYSPACE}
    \end{QUESTION}
    \begin{QUESTION}
        \begin{ExamInfo}{89}{學測}{單選}{5}
        \end{ExamInfo}
        \begin{ExamAnsRateInfo}{}{}{}{}
        \end{ExamAnsRateInfo}
        \begin{QBODY}
            假設世界人口自1980年起,50年內每年增長率均固定。已知1987年世界人口達50億,1999年第60億人誕生在賽拉佛耶。根據這些資料推測2023年世界人口數最接近下列哪一個數?
            \begin{QOPS}
                \QOP 75億
                \QOP 80億
                \QOP 86億
                \QOP 92億
                \QOP 100億
            \end{QOPS}            
        \end{QBODY}
        \begin{QFROMS}
        \end{QFROMS}
        \begin{QTAGS}\end{QTAGS}
        \begin{QANS}
			(3)
        \end{QANS}
        \begin{QSOLLIST}
        \end{QSOLLIST}
        \begin{QEMPTYSPACE}
        \end{QEMPTYSPACE}
    \end{QUESTION}
    \begin{QUESTION}
        \begin{ExamInfo}{89}{學測}{單選}{6}
        \end{ExamInfo}
        \begin{ExamAnsRateInfo}{}{}{}{}
        \end{ExamAnsRateInfo}
        \begin{QBODY}
		在1999年6月1日數學家利用超級電腦驗證出${{2}^{6972593}}-1$是一個質數。若想要列印出此質數至少需要多少張A4紙?假定每張A4紙,可列印出3000個數字。在下列選項中,選出最接近的張數。[${{\log }_{10}}2\approx 0.3010$]
		\begin{QOPS}
			\QOP $ 50  $
			\QOP $ 100 $
			\QOP $ 200 $
			\QOP $ 500 $
			\QOP $ 700 $
		\end{QOPS}
        \end{QBODY}
        \begin{QFROMS}
        \end{QFROMS}
        \begin{QTAGS}\end{QTAGS}
        \begin{QANS}
			(5)
        \end{QANS}
        \begin{QSOLLIST}
        \end{QSOLLIST}
        \begin{QEMPTYSPACE}
        \end{QEMPTYSPACE}
    \end{QUESTION}
    \begin{QUESTION}
        \begin{ExamInfo}{89}{學測}{單選}{7}
        \end{ExamInfo}
        \begin{ExamAnsRateInfo}{}{}{}{}
        \end{ExamAnsRateInfo}
        \begin{QBODY}
			設$P_1$表示丟2個公正硬幣時,恰好出現1個正面的機率,$P_2$表示擲2個均勻骰子,恰好出現1個偶數點的機率,$P_3$表示丟4個公正硬幣時,恰好出現2個正面的機率。試問下列選項何者為真?
			\begin{QOPS}
				\QOP $P_1=P_2=P_3$
				\QOP $P_1=P_2>P_3$
				\QOP $P_1=P_3<P_2$
				\QOP $P_1=P_3>P_2$
				\QOP $P_3>P_2>P_1$
			\end{QOPS}
        \end{QBODY}
        \begin{QFROMS}
        \end{QFROMS}
        \begin{QTAGS}\end{QTAGS}
        \begin{QANS}
			(2)
        \end{QANS}
        \begin{QSOLLIST}
        \end{QSOLLIST}
        \begin{QEMPTYSPACE}
        \end{QEMPTYSPACE}
    \end{QUESTION}
\end{QUESTIONS}\begin{QUESTIONS}
    \begin{QUESTION}
        \begin{ExamInfo}{89}{學測}{多選}{8}
        \end{ExamInfo}
        \begin{ExamAnsRateInfo}{}{}{}{}
        \end{ExamAnsRateInfo}
        \begin{QBODY}
			在坐標平面上,以$(-1,\ 1),\ (3,\ 1)$為焦點,且通過點$(3,\ 4)$畫一雙曲線。
試問此雙曲線也會通過下列哪些點?
			\begin{QOPS}
				\QOP	$(1,\ 1)$
				\QOP	$(-1,\ 4)$
				\QOP	$(3,\ -2)$
				\QOP	$(-1,\ -2)$
				\QOP	$(3,\ 1)$
			\end{QOPS}
        \end{QBODY}
        \begin{QFROMS}
        \end{QFROMS}
        \begin{QTAGS}\end{QTAGS}
        \begin{QANS}
			(2)(3)(4)
        \end{QANS}
        \begin{QSOLLIST}
        \end{QSOLLIST}
        \begin{QEMPTYSPACE}
        \end{QEMPTYSPACE}
    \end{QUESTION}
    \begin{QUESTION}
        \begin{ExamInfo}{89}{學測}{多選}{9}
        \end{ExamInfo}
        \begin{ExamAnsRateInfo}{}{}{}{}
        \end{ExamAnsRateInfo}
        \begin{QBODY}
		阿山家在一條東西向馬路的北方D點處,為了不同目的,他走到馬路的路線有下列三條:
		向南走$a$公尺到$A$點之後,繼續向南走$a$公尺到達馬路;
		向東南走$b$公尺到$B$點之後,繼續向南走$b$公尺到達馬路;
		向東走$c$公尺到$C$點之後,繼續向南走$c$公尺到達馬路。
		根據上述資料,下列選項何者為真?
		\begin{QOPS}
			\QOP $c=2a$
			\QOP $a<b<c$
			\QOP $b=\sqrt{2}$a
			\QOP $A,B,C,D$ 四點共圓
			\QOP $A,B,C$ 三點剛好在以$D$ 點為焦點的拋物線上
		\end{QOPS}
        \end{QBODY}
        \begin{QFROMS}
        \end{QFROMS}
        \begin{QTAGS}\end{QTAGS}
        \begin{QANS}
		(1)(2)(5)
        \end{QANS}
        \begin{QSOLLIST}
        \end{QSOLLIST}
        \begin{QEMPTYSPACE}
        \end{QEMPTYSPACE}
    \end{QUESTION}
    \begin{QUESTION}
        \begin{ExamInfo}{89}{學測}{多選}{10}
        \end{ExamInfo}
        \begin{ExamAnsRateInfo}{}{}{}{}
        \end{ExamAnsRateInfo}
        \begin{QBODY}
			將行列式
				$\left| \begin{matrix}
				   \ x & 1 & 2\   \\
				   \ 1 & x & 2\   \\
				   \ 1 & 2 & x  \\
				\end{matrix} \right|$
				展開得到多項式$f(x)$。下列有關$f(x)$的敘述,何者為真?
				\begin{QOPS}
					\QOP $f(x)$是一個三次多項式
					\QOP $f(1)=0$
					\QOP $f(2)=0$
					\QOP $f(-3)=0$
					\QOP $f(5)=0$
				\end{QOPS}
        \end{QBODY}
        \begin{QFROMS}
        \end{QFROMS}
        \begin{QTAGS}\end{QTAGS}
        \begin{QANS}
			(1)(2)(3)(4)
        \end{QANS}
        \begin{QSOLLIST}
        \end{QSOLLIST}
        \begin{QEMPTYSPACE}
        \end{QEMPTYSPACE}
    \end{QUESTION}
\end{QUESTIONS}\begin{QUESTIONS}
    \begin{QUESTION}
        \begin{ExamInfo}{89}{學測}{選填}{11}
        \end{ExamInfo}
        \begin{ExamAnsRateInfo}{}{}{}{}
        \end{ExamAnsRateInfo}
        \begin{QBODY}
			今年(公元2000年是閏年)的1月1日是星期六。試問下一個1月1日
			也是星期六,發生在公元哪一年?答:$\TCNBOX{20\TCN\TCN}$年。
		\end{QBODY}
        \begin{QFROMS}
        \end{QFROMS}
        \begin{QTAGS}\end{QTAGS}
        \begin{QANS}
			$2005$
        \end{QANS}
        \begin{QSOLLIST}
        \end{QSOLLIST}
        \begin{QEMPTYSPACE}
        \end{QEMPTYSPACE}
    \end{QUESTION}
    \begin{QUESTION}
        \begin{ExamInfo}{89}{學測}{選填}{12}
        \end{ExamInfo}
        \begin{ExamAnsRateInfo}{}{}{}{}
        \end{ExamAnsRateInfo}
        \begin{QBODY}
			將自然數按下列規律排列,每一列比前一列多一個數,如下表所示:
			%TODO: 補表格圖
			試問第100列第3個數是多少?答:$\TCNBOX{\TCN\TCN\TCN\TCN}$。
        \end{QBODY}
        \begin{QFROMS}
        \end{QFROMS}
        \begin{QTAGS}\end{QTAGS}
        \begin{QANS}
		$4953$
        \end{QANS}
        \begin{QSOLLIST}
        \end{QSOLLIST}
        \begin{QEMPTYSPACE}
        \end{QEMPTYSPACE}
    \end{QUESTION}
    \begin{QUESTION}
        \begin{ExamInfo}{89}{學測}{選填}{13}
        \end{ExamInfo}
        \begin{ExamAnsRateInfo}{}{}{}{}
        \end{ExamAnsRateInfo}
        \begin{QBODY}
		設三次方程式${{x}^{3}}-17{{x}^{2}}+32x-30=0$有兩複數根$a+i,\ 1+bi$,其中$a,b$是不
		為$0$的實數。試求它的實根。答:$\TCNBOX{\TCN\TCN}$。

        \end{QBODY}
        \begin{QFROMS}
        \end{QFROMS}
        \begin{QTAGS}\end{QTAGS}
        \begin{QANS}
			$15$
        \end{QANS}
        \begin{QSOLLIST}
        \end{QSOLLIST}
        \begin{QEMPTYSPACE}
        \end{QEMPTYSPACE}
    \end{QUESTION}
    \begin{QUESTION}
        \begin{ExamInfo}{89}{學測}{選填}{14}
        \end{ExamInfo}
        \begin{ExamAnsRateInfo}{}{}{}{}
        \end{ExamAnsRateInfo}
        \begin{QBODY}
		空間中有一直線 $L$ 與平面$E:x+2y+3z=9$垂直。試求通過點$(2,-3,4)$且
與直線$L$垂直的平面方程式。答: $\TCNBOX{\TCN x+\TCN y+\TCN z=\TCN}$  。
        \end{QBODY}
        \begin{QFROMS}
        \end{QFROMS}
        \begin{QTAGS}\end{QTAGS}
        \begin{QANS}
		$x+2y+3z=8$
        \end{QANS}
        \begin{QSOLLIST}
        \end{QSOLLIST}
        \begin{QEMPTYSPACE}
        \end{QEMPTYSPACE}
    \end{QUESTION}
    \begin{QUESTION}
        \begin{ExamInfo}{89}{學測}{選填}{15}
        \end{ExamInfo}
        \begin{ExamAnsRateInfo}{}{}{}{}
        \end{ExamAnsRateInfo}
        \begin{QBODY}
		在某海防觀測站的東方12海浬處有A、B兩艘船相會之後,A船以每小時12海浬的速度往南航行,B船以每小時3海浬的速度向北航行。
問幾小時後,觀測站及A、B兩船恰成一直角三角形?答:$\TCNBOX{\TCN}$小時。

        \end{QBODY}
        \begin{QFROMS}
        \end{QFROMS}
        \begin{QTAGS}\end{QTAGS}
        \begin{QANS}
			$2$
        \end{QANS}
        \begin{QSOLLIST}
        \end{QSOLLIST}
        \begin{QEMPTYSPACE}
        \end{QEMPTYSPACE}
    \end{QUESTION}
    \begin{QUESTION}
        \begin{ExamInfo}{89}{學測}{選填}{16}
        \end{ExamInfo}
        \begin{ExamAnsRateInfo}{}{}{}{}
        \end{ExamAnsRateInfo}
        \begin{QBODY}
		氣象局測出在20小時期間,颱風中心的位置由恆春東南方400公里直線移動到恆春南 $15^\circ$西的200公里處,試求颱風移動的平均速度。(整
數以下,四捨五入)答:$\TCNBOX{\TCN\TCN}$公里/時。

        \end{QBODY}
        \begin{QFROMS}
        \end{QFROMS}
        \begin{QTAGS}\end{QTAGS}
        \begin{QANS}
		$17$
        \end{QANS}
        \begin{QSOLLIST}
        \end{QSOLLIST}
        \begin{QEMPTYSPACE}
        \end{QEMPTYSPACE}
    \end{QUESTION}
    \begin{QUESTION}
        \begin{ExamInfo}{89}{學測}{選填}{17}
        \end{ExamInfo}
        \begin{ExamAnsRateInfo}{}{}{}{}
        \end{ExamAnsRateInfo}
        \begin{QBODY}
		桌面上有大小兩顆球,相互靠在一起。已知大球的半徑為20公分,小球半徑5公分。試求這兩顆球分別與桌面相接觸的兩點之間的距離。
答:$\TCNBOX{\TCN\TCN}$公分。

        \end{QBODY}
        \begin{QFROMS}
        \end{QFROMS}
        \begin{QTAGS}\end{QTAGS}
        \begin{QANS}
		$20$
        \end{QANS}
        \begin{QSOLLIST}
        \end{QSOLLIST}
        \begin{QEMPTYSPACE}
        \end{QEMPTYSPACE}
    \end{QUESTION}
    \begin{QUESTION}
        \begin{ExamInfo}{89}{學測}{選填}{18}
        \end{ExamInfo}
        \begin{ExamAnsRateInfo}{}{}{}{}
        \end{ExamAnsRateInfo}
        \begin{QBODY}
			體操委員會由10位女性委員與5位男性委員組成。委員會要由6位委員組團出國考察,如以性別做分層,並在各層依比例隨機抽樣,試問
此考察團共有多少種組成方式?答:$\TCNBOX{\TCN\TCN\TCN\TCN}$種。

        \end{QBODY}
        \begin{QFROMS}
        \end{QFROMS}
        \begin{QTAGS}\end{QTAGS}
        \begin{QANS}
		$2100$
        \end{QANS}
        \begin{QSOLLIST}
        \end{QSOLLIST}
        \begin{QEMPTYSPACE}
        \end{QEMPTYSPACE}
    \end{QUESTION}
    \begin{QUESTION}
        \begin{ExamInfo}{89}{學測}{選填}{19}
        \end{ExamInfo}
        \begin{ExamAnsRateInfo}{}{}{}{}
        \end{ExamAnsRateInfo}
        \begin{QBODY}
		交通規則測驗時,答對有兩種可能,一種是會做而答對,一種是不會做但猜對。已知小華練習交通規則筆試測驗,會做的機率是0.8。現有一題5選1的交通規則選擇題,設小華會做就答對,不會做就亂猜。已知此題小華答對,試問在此條件之下,此題小華是因會做而答對(不是亂猜)的機率是多少?答:$\TCNBOX{\FR{\TCN\TCN}{\TCN\TCN}}$。(以最簡分數表示)
        \end{QBODY}
        \begin{QFROMS}
        \end{QFROMS}
        \begin{QTAGS}\end{QTAGS}
        \begin{QANS}
		$\FR{20}{21}$
        \end{QANS}
        \begin{QSOLLIST}
        \end{QSOLLIST}
        \begin{QEMPTYSPACE}
        \end{QEMPTYSPACE}
    \end{QUESTION}
    \begin{QUESTION}
        \begin{ExamInfo}{89}{學測}{選填}{20}
        \end{ExamInfo}
        \begin{ExamAnsRateInfo}{}{}{}{}
        \end{ExamAnsRateInfo}
        \begin{QBODY}
		如下圖所示,有一船位於甲港口的東方27公里北方8公里A處,直朝位於港口的東方2公里北方3公里B處的航標駛去,到達航標後即修正航向以便直線駛入港口。試問船在航標處的航向修正應該向左轉多
少度?(整數以下,四捨五入)答:$\TCNBOX{\TCN\TCN}$度。
		%TODO 補圖
        \end{QBODY}
        \begin{QFROMS}
        \end{QFROMS}
        \begin{QTAGS}\end{QTAGS}
        \begin{QANS}
		$45$
        \end{QANS}
        \begin{QSOLLIST}
        \end{QSOLLIST}
        \begin{QEMPTYSPACE}
        \end{QEMPTYSPACE}
    \end{QUESTION}
\end{QUESTIONS}