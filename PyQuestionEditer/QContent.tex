% !TEX encoding = UTF-8 Unicode
% !TEX TS-program = xelatex
% 正式題目區:
%    \begin{QUESTION}
%        \begin{QBODY}
%        \end{QBODY}
%        \begin{QTAGS}
%        \end{QTAGS}
%        \begin{QANS}
%        \end{QANS}
%        \begin{QSOL}
%        \end{QSOL}
%    \end{QUESTION}
\begin{QUESTIONS}
\begin{QUESTION}
    \begin{QBODY}
        設 $f\left( x \right)={{x}^{3}}+a{{x}^{2}}+bx+c$ 為整係數多項式,已知$f\left( x \right)=0$ 的三根皆為有理數,
        且 $f\left( \sqrt{3} \right)<0$,$f\left( \sqrt{5} \right)>0$,$f\left( \pi  \right)<0$,$f\left( \sqrt{17} \right)>0$
        則常數項 $c = \originalAnsBox $ 。   
    \end{QBODY}
    \begin{QFROMS}
    \end{QFROMS}
    \begin{QTAGS}\QTAG{B1C2-3} \end{QTAGS}
    \begin{QANS}
    \end{QANS}
    \begin{QSOL}
    \end{QSOL}
    \begin{QEMPTYSPACE}
    \end{QEMPTYSPACE}
\end{QUESTION}
\begin{QUESTION}
    \begin{QBODY}
        設 $f\left( x \right)=a{{x}^{2}}+bx+c$ 為二次函數,且已知 $f\left( x \right)>0$ 的解為 $-3<x<9$ 
        則下列敘述哪些是正確的?
        \begin{QOPS}
           \QOP $a<0$ 
           \QOP $a+b+c=-32$
           \QOP $f\left( 3x \right)\le 0$
           \QOP $y=f\left( x \right)$之圖形與 $x$ 軸交於相異兩點
           \QOP 當 $x=3$ 時,$f\left( x \right)$ 有最大值
        \end{QOPS}       
    \end{QBODY}
    \begin{QFROMS}
    \end{QFROMS}
    \begin{QTAGS}\QTAG{B1C2-4} \end{QTAGS}
    \begin{QANS}
        (1)(3)(4)(5)
    \end{QANS}
    \begin{QSOL}
    \end{QSOL}
    \begin{QEMPTYSPACE}
    \end{QEMPTYSPACE}
\end{QUESTION}
\begin{QUESTION}
    \begin{QBODY}
        已知 $a>0$, $b<0$ ,試選出正確的敘述
        \begin{QOPS}
            \QOP $\sqrt{a}\sqrt{b}=\sqrt{ab}$ 
            \QOP $\frac{\sqrt{a}}{\sqrt{b}}=\sqrt{\frac{a}{b}}$ 
            \QOP $\sqrt{b}=\sqrt{-b}\cdot i$ 
            \QOP $\sqrt{{{b}^{2}}}=b$ 
            \QOP ${{\left( \sqrt{b} \right)}^{2}}=b$ 
        \end{QOPS}
    \end{QBODY}
    \begin{QFROMS}
    \end{QFROMS}
    \begin{QTAGS}\QTAG{B1C2-3} \end{QTAGS}
    \begin{QANS}
        (1)(3)(5)
    \end{QANS}
    \begin{QSOL}
    \end{QSOL}
    \begin{QEMPTYSPACE}
    \end{QEMPTYSPACE}
\end{QUESTION}
\begin{QUESTION}
    \begin{QBODY}
        已知 $0<a<1$ 對於指數函數 $y={{a}^{x}}$ 的圖形特徵,請選出正確的敘述:
        \begin{QOPS}
            \QOP 曲線均在 $y$ 軸右方
            \QOP 通過點 $\left( 0,1 \right)$
            \QOP 曲線由左而右上升
            \QOP 凹口向下
            \QOP 與 $y={{a}^{-x}}$ 的圖形對稱於 $y$ 軸            
        \end{QOPS}
    \end{QBODY}
    \begin{QFROMS}
    \end{QFROMS}
    \begin{QTAGS}\QTAG{B1C3-2} \end{QTAGS}
    \begin{QANS}
        (2)(5)
    \end{QANS}
    \begin{QSOL}
    \end{QSOL}
    \begin{QEMPTYSPACE}
    \end{QEMPTYSPACE}
\end{QUESTION}
\begin{QUESTION}
    \begin{QBODY}
        設有理係數多項式 $f\left( x \right)=5{{x}^{7}}+a{{x}^{6}}+b{{x}^{5}}+c{{x}^{4}}+d{{x}^{3}}+e{{x}^{2}}+fx+3$,試選出正確敘述。
        \begin{QOPS}
            \QOP $2x+7$ 不可能是 $f\left( x \right)$ 之因式
            \QOP $f\left( x \right)=0$ 必有實根
            \QOP 若 $f\left( x \right)=0$ 在 $2$ 與 3 之間有實根,則 $f\left( 2 \right)f\left( 3 \right)<0$
            \QOP 若 $i+2$ 為 $f\left( x \right)=0$ 的根,則 $i-2$ 亦為 $f\left( x \right)=0$ 的根
            \QOP 若 $4-\sqrt{3}$ 為 $f\left( x \right)=0$ 的根,則 $4+\sqrt{3}$ 亦為$f\left( x \right)=0$ 的根
        \end{QOPS}
    \end{QBODY}
    \begin{QFROMS}
    \end{QFROMS}
    \begin{QTAGS}\QTAG{B1C2-3} \end{QTAGS}
    \begin{QANS}
        (2)(5)
    \end{QANS}
    \begin{QSOL}
    \end{QSOL}
    \begin{QEMPTYSPACE}
    \end{QEMPTYSPACE}
\end{QUESTION}
\begin{QUESTION}
    \begin{QBODY}
        設 $x$ 為實數,則下列哪些方程式恰有兩解?\\
        \begin{QOPSINONELINE}
            \QOP $x={{2}^{x}}$ 
            \QOP ${{x}^{2}}={{2}^{x}}$ 
            \QOP ${{2}^{|x|}}=2x+1$ 
            \QOP ${{2}^{x}}=x+3$ 
            \QOP ${{2}^{-|x|}}={{x}^{2}}$
        \end{QOPSINONELINE}
    \end{QBODY}
    \begin{QFROMS}
    \end{QFROMS}
    \begin{QTAGS}\QTAG{B1C3-2} \end{QTAGS}
    \begin{QANS}
        (3)(4)(5)
    \end{QANS}
    \begin{QSOL}
    \end{QSOL}
    \begin{QEMPTYSPACE}
    \end{QEMPTYSPACE}
\end{QUESTION}
\begin{QUESTION}
    \begin{QBODY}
        解下列各不等式:
        \begin{SUBQLIST}
            \SUBQ $-{{x}^{2}}+4x-4<0$
            \SUBQ $\left( 5-x \right){{\left( x+1 \right)}^{7}}{{\left( x-3 \right)}^{12}}\left( {{x}^{2}}-2x-1 \right)\left( {{x}^{2}}+x+1 \right)>0$
            \SUBQ $\frac{x-3}{x+1}\le -1$
        \end{SUBQLIST}
    \end{QBODY}
    \begin{QFROMS}
    \end{QFROMS}
    \begin{QTAGS}\QTAG{B1C2-4} \end{QTAGS}
    \begin{QANS}
        (1) $x \in \mathbb{R} $ 但 $x \ne =2 $
        (2) $-1<x<1-\sqrt{2}$ 或 $1+\sqrt{2} <x<5$ 但 $x\ne3$
        (3) $-1<x\le 1$
    \end{QANS}
    \begin{QSOL}
    \end{QSOL}
    \begin{QEMPTYSPACE}
    \end{QEMPTYSPACE}
\end{QUESTION}
\begin{QUESTION}
    \begin{QBODY}
        解方程式 $12{{x}^{3}}-16{{x}^{2}}-3x+4=0$
    \end{QBODY}
    \begin{QFROMS}
    \end{QFROMS}
    \begin{QTAGS}\QTAG{B1C2-3} \end{QTAGS}
    \begin{QANS}
        $\frac{4}{3}, \frac{1}{2}, -\frac{1}{2}$
    \end{QANS}
    \begin{QSOL}
    \end{QSOL}
    \begin{QEMPTYSPACE}
    \end{QEMPTYSPACE}
\end{QUESTION}
\begin{QUESTION}
    \begin{QBODY}
        設 $\alpha ,\beta $ 為方程式 ${{x}^{2}}+8x+5=0$ 之兩根,則 ${{\left( \sqrt{\alpha }-\sqrt{\beta } \right)}^{2}}= \originalAnsBox $
    \end{QBODY}
    \begin{QFROMS}
    \end{QFROMS}
    \begin{QTAGS}\QTAG{B1C2-3} \end{QTAGS}
    \begin{QANS}
        $-8+2\sqrt{5}$
    \end{QANS}
    \begin{QSOL}
    \end{QSOL}
    \begin{QEMPTYSPACE}
    \end{QEMPTYSPACE}
\end{QUESTION}
\begin{QUESTION}
    \begin{QBODY}
        知二次函數 $y=a{{x}^{2}}+2x+\left( a+1 \right)$ 的圖形恆在直線 $y=3$ 的上方,求實數 $a$ 之範圍。
    \end{QBODY}
    \begin{QFROMS}
    \end{QFROMS}
    \begin{QTAGS}\QTAG{B1C2-4} \end{QTAGS}
    \begin{QANS}
        $a>1+\sqrt{2}$
    \end{QANS}
    \begin{QSOL}
    \end{QSOL}
    \begin{QEMPTYSPACE}
    \end{QEMPTYSPACE}
\end{QUESTION}
\begin{QUESTION}
    \begin{QBODY}
        二次函數 $f\left( x \right)$ 滿足以下三個條件:
        條件一:對於所有實數 $t$ ,恆有 $f\left( 4-t \right)=f\left( 4+t \right)$
        條件二:當 $0\le x\le 6$ 時,$f\left( x \right)$ 的最大值為 $15$ 最小值為 $-1$ 
        條件三:$f\left( 0 \right)>f\left( 6 \right)$
        若$y=f\left( x \right)$ 之圖形與 $x$ 軸交於$A,B$ 兩點,則$\overline{AB}=\originalAnsBox $。
    \end{QBODY}
    \begin{QFROMS}
    \end{QFROMS}
    \begin{QTAGS}\QTAG{B1C2-2} \end{QTAGS}
    \begin{QANS}
        $2$
    \end{QANS}
    \begin{QSOL}
    \end{QSOL}
    \begin{QEMPTYSPACE}
    \end{QEMPTYSPACE}
\end{QUESTION}
\begin{QUESTION}
    \begin{QBODY}
        計算 ${{i}^{2000}}+{{i}^{2001}}+...+{{i}^{2016}}= \originalAnsBox $
    \end{QBODY}
    \begin{QFROMS}
    \end{QFROMS}
    \begin{QTAGS}\QTAG{B1C2-3} \end{QTAGS}
    \begin{QANS}
        $1$
    \end{QANS}
    \begin{QSOL}
    \end{QSOL}
    \begin{QEMPTYSPACE}
    \end{QEMPTYSPACE}
\end{QUESTION}
\begin{QUESTION}
    \begin{QBODY}
        設複數 $z$ 的實部是 2,$\frac{1}{z}$ 的虛部是 $\frac{-1}{5}$ 則$z=\originalAnsBox$。
    \end{QBODY}
    \begin{QFROMS}
    \end{QFROMS}
    \begin{QTAGS}\QTAG{B1C2-3} \end{QTAGS}
    \begin{QANS}
        $2+i \vee 2+4i$
    \end{QANS}
    \begin{QSOL}
    \end{QSOL}
    \begin{QEMPTYSPACE}
    \end{QEMPTYSPACE}
\end{QUESTION}
\begin{QUESTION}
    \begin{QBODY}
        方程式 ${{x}^{3}}-3{{x}^{2}}+ax+b=0$ 有兩相異的複數根為 $c+i$,$2+di$,
        其中 $a,b,c,d$ 為非零的實數,請寫出此方程式的所有根。
    \end{QBODY}
    \begin{QFROMS}
    \end{QFROMS}
    \begin{QTAGS}\QTAG{B1C2-3} \end{QTAGS}
    \begin{QANS}
        $2+i,2-i,-1$
    \end{QANS}
    \begin{QSOL}
    \end{QSOL}
    \begin{QEMPTYSPACE}
    \end{QEMPTYSPACE}
\end{QUESTION}
\begin{QUESTION}
    \begin{QBODY}
        已知 $a$ 為實數,若方程式 $3{{x}^{2}}+\left( a+i \right)x+2i-8=0$ 有實根,則
        \begin{SUBQLIST}
            \SUBQ $a=\_\_\_\_\_\_\_\_$
            \SUBQ 寫出此方程式的所有根
        \end{SUBQLIST}
    \end{QBODY}
    \begin{QFROMS}
    \end{QFROMS}
    \begin{QTAGS}\QTAG{B1C2-3} \end{QTAGS}
    \begin{QANS}
        (1) $2$  (2)  $-2,\frac{4-i}{3}$
    \end{QANS}
    \begin{QSOL}
    \end{QSOL}
    \begin{QEMPTYSPACE}
    \end{QEMPTYSPACE}
\end{QUESTION}
\begin{QUESTION}
    \begin{QBODY}
        設 $f\left( x \right)={{x}^{3}}-5x+1$ 試求 $f\left( x \right)=0$ 的最小正根(四捨五入到小數點後第一位)        
    \end{QBODY}
    \begin{QFROMS}
    \end{QFROMS}
    \begin{QTAGS}\QTAG{B1C2-3} \end{QTAGS}
    \begin{QANS}
        $0.2$
    \end{QANS}
    \begin{QSOL}
    \end{QSOL}
    \begin{QEMPTYSPACE}
    \end{QEMPTYSPACE}
\end{QUESTION}
\begin{QUESTION}
    \begin{QBODY}
        化簡 ${{\left( \frac{1}{4} \right)}^{\frac{1}{3}}}\cdot {{\left( \sqrt[3]{4} \right)}^{4}}\cdot {{\left( 2012+\pi  \right)}^{0}}=\_\_\_\_\_\_\_\_\_\_$。
    \end{QBODY}
    \begin{QFROMS}
    \end{QFROMS}
    \begin{QTAGS}\QTAG{B1C3-1} \end{QTAGS}
    \begin{QANS}
        $4$
    \end{QANS}
    \begin{QSOL}
    \end{QSOL}
    \begin{QEMPTYSPACE}
    \end{QEMPTYSPACE}
\end{QUESTION}
\begin{QUESTION}
    \begin{QBODY}
        設 $a={{3}^{0.7}}$ $b={{3}^{0.03}}$ 若已知 ${{3}^{0.58}}={{a}^{x}}{{b}^{y}}$,其中 $x,y$ 均為整數,且 $|y|\ \le 10$ 
        則數對 $\left( x,y \right)= \originalAnsBox $。
    \end{QBODY}
    \begin{QFROMS}
    \end{QFROMS}
    \begin{QTAGS}\QTAG{B1C3-1} \end{QTAGS}
    \begin{QANS}
        $\left( 1,-4 \right)$
    \end{QANS}
    \begin{QSOL}
    \end{QSOL}
    \begin{QEMPTYSPACE}
    \end{QEMPTYSPACE}
\end{QUESTION}
\begin{QUESTION}
    \begin{QBODY}
        已知 $a>0$,且 ${{a}^{2x}}=\sqrt{2}-1$ 試計算 $\frac{{{a}^{3x}}-{{a}^{-3x}}}{{{a}^{x}}-{{a}^{-x}}}$ 之值 $\originalAnsBox$。
    \end{QBODY}
    \begin{QFROMS}
    \end{QFROMS}
    \begin{QTAGS}\QTAG{B1C3-1} \end{QTAGS}
    \begin{QANS}
        $1+2\sqrt{2}$
    \end{QANS}
    \begin{QSOL}
    \end{QSOL}
    \begin{QEMPTYSPACE}
    \end{QEMPTYSPACE}
\end{QUESTION}
\begin{QUESTION}
    \begin{QBODY}
        解方程式 ${{4}^{x+1}}-31\times {{2}^{x}}-8=0 \originalAnsBox $ 。
    \end{QBODY}
    \begin{QFROMS}
    \end{QFROMS}
    \begin{QTAGS}\QTAG{B1C3-1} \end{QTAGS}
    \begin{QANS}
        $x=3$
    \end{QANS}
    \begin{QSOL}
    \end{QSOL}
    \begin{QEMPTYSPACE}
    \end{QEMPTYSPACE}
\end{QUESTION}
\begin{QUESTION}
    \begin{QBODY}
        解方程式 $4\cdot \left( {{4}^{x}}+{{4}^{-x}} \right)-12\left( {{2}^{x}}+{{2}^{-x}} \right)+13=0$
    \end{QBODY}
    \begin{QFROMS}
    \end{QFROMS}
    \begin{QTAGS}\QTAG{B1C3-1} \end{QTAGS}
    \begin{QANS}
        $1,-1$
    \end{QANS}
    \begin{QSOL}
    \end{QSOL}
    \begin{QEMPTYSPACE}
    \end{QEMPTYSPACE}
\end{QUESTION}
\begin{QUESTION}
    \begin{QBODY}
        試比較下列三數之大小:${{\left( \frac{1}{3} \right)}^{\tfrac{1}{3}}},\ {{\left( \frac{1}{4} \right)}^{\tfrac{1}{4}}},{{\left( \frac{1}{6} \right)}^{\tfrac{1}{6}}}$
    \end{QBODY}
    \begin{QFROMS}
    \end{QFROMS}
    \begin{QTAGS}\QTAG{B1C3-2} \end{QTAGS}
    \begin{QANS}
        ${{\left( \frac{1}{3} \right)}^{\tfrac{1}{3}}}<{{\left( \frac{1}{4} \right)}^{\tfrac{1}{4}}}<{{\left( \frac{1}{6} \right)}^{\tfrac{1}{6}}}$
    \end{QANS}
    \begin{QSOL}
    \end{QSOL}
    \begin{QEMPTYSPACE}
    \end{QEMPTYSPACE}
\end{QUESTION}
\begin{QUESTION}
    \begin{QBODY}
        設 ${{\left( \frac{1}{3} \right)}^{2x-2}}+17\cdot {{\left( \frac{1}{3} \right)}^{x}}-2>0$ 求 $x$ 值的範圍。
    \end{QBODY}
    \begin{QFROMS}
    \end{QFROMS}
    \begin{QTAGS}\QTAG{B1C3-2} \end{QTAGS}
    \begin{QANS}
        $x<2$
    \end{QANS}
    \begin{QSOL}
    \end{QSOL}
    \begin{QEMPTYSPACE}
    \end{QEMPTYSPACE}
\end{QUESTION}
\begin{QUESTION}
    \begin{QBODY}
        ${{67}^{x}}=27$,${{603}^{y}}=81$ 則 $\frac{3}{x}-\frac{4}{y}=\originalAnsBox$。
    \end{QBODY}
    \begin{QFROMS}
    \end{QFROMS}
    \begin{QTAGS}\QTAG{B1C3-1} \end{QTAGS}
    \begin{QANS}
    \end{QANS}
    \begin{QSOL}
    \end{QSOL}
    \begin{QEMPTYSPACE}
    \end{QEMPTYSPACE}
\end{QUESTION}
\begin{QUESTION}
    \begin{QBODY}
        已知 ${{2}^{x}}={{3}^{y}}={{5}^{z}}=a$且$\frac{1}{x}+\frac{1}{y}+\frac{1}{z}=2$,則$a=\originalAnsBox $。
    \end{QBODY}
    \begin{QFROMS}
    \end{QFROMS}
    \begin{QTAGS}\QTAG{B1C3-1} \end{QTAGS}
    \begin{QANS}
        $\sqrt{30}$
    \end{QANS}
    \begin{QSOL}
    \end{QSOL}
    \begin{QEMPTYSPACE}
    \end{QEMPTYSPACE}
\end{QUESTION}
\begin{QUESTION}
    \begin{QBODY}
         $a>0$且${{a}^{4x}}=\sqrt{29-12\sqrt{3+2\sqrt{2}}}$,則
         \begin{SUBQLIST}
             \SUBQ $\frac{{{a}^{6x}}+{{a}^{-6x}}}{{{a}^{2x}}+{{a}^{-2x}}}=\originalAnsBox$,
             \SUBQ $\frac{{{a}^{6x}}+{{a}^{-6x}}}{{{a}^{2x}}-{{a}^{-2x}}}=\originalAnsBox $                 。
         \end{SUBQLIST}
    \end{QBODY}
    \begin{QFROMS}
    \end{QFROMS}
    \begin{QTAGS}\QTAG{B1C3-1} \end{QTAGS}
    \begin{QANS}
        $5$,$-5\sqrt{2}$
    \end{QANS}
    \begin{QSOL}
    \end{QSOL}
    \begin{QEMPTYSPACE}
    \end{QEMPTYSPACE}
\end{QUESTION}
\begin{QUESTION}
    \begin{QBODY}
        ${{4}^{-x}}-3\cdot {{2}^{2-x}}+32=0$ 則 $x=\originalAnsBox $。
    \end{QBODY}
    \begin{QFROMS}
    \end{QFROMS}
    \begin{QTAGS}\QTAG{B1C3-1} \end{QTAGS}
    \begin{QANS}
    \end{QANS}
    \begin{QSOL}
    \end{QSOL}
    \begin{QEMPTYSPACE}
    \end{QEMPTYSPACE}
\end{QUESTION}
\begin{QUESTION}
    \begin{QBODY}
        $\frac{{{4}^{x}}}{32}<8\sqrt{2}<\frac{1}{\ \ {{128}^{-x}}}$
    \end{QBODY}
    \begin{QFROMS}
    \end{QFROMS}
    \begin{QTAGS}\QTAG{B1C3-2} \end{QTAGS}
    \begin{QANS}
    \end{QANS}
    \begin{QSOL}
    \end{QSOL}
    \begin{QEMPTYSPACE}
    \end{QEMPTYSPACE}
\end{QUESTION}
\begin{QUESTION}
    \begin{QBODY}
        $x>0$,解${{x}^{{{x}^{2}}-4}}\ge {{({{x}^{x}})}^{3}}$。                                       
    \end{QBODY}
    \begin{QFROMS}
    \end{QFROMS}
    \begin{QTAGS}\QTAG{B1C3-2} \end{QTAGS}
    \begin{QANS}
        $0<x\le 1$或$x\ge 4$
    \end{QANS}
    \begin{QSOL}
    \end{QSOL}
    \begin{QEMPTYSPACE}
    \end{QEMPTYSPACE}
\end{QUESTION}
\begin{QUESTION}
    \begin{QBODY}
        若${{4}^{x}}-\sqrt{3}\cdot {{2}^{x+2}}+\sqrt{8}=0$之兩根為$\alpha \ \ \beta $,則$\alpha +\beta=\originalAnsBox $
    \end{QBODY}
    \begin{QFROMS}
    \end{QFROMS}
    \begin{QTAGS}\QTAG{B1C3-2} \end{QTAGS}
    \begin{QANS}
        $\frac{3}{2}$
    \end{QANS}
    \begin{QSOL}
    \end{QSOL}
    \begin{QEMPTYSPACE}
    \end{QEMPTYSPACE}
\end{QUESTION}
\begin{QUESTION}
    \begin{QBODY}
        ${{x}^{2}}-1={{2}^{-|\ x\ |}}$ 有幾個實根,各界於哪兩個連續整數之間。
    \end{QBODY}
    \begin{QFROMS}
    \end{QFROMS}
    \begin{QTAGS}\QTAG{B1C3-2} \end{QTAGS}
    \begin{QANS}
    \end{QANS}
    \begin{QSOL}
    \end{QSOL}
    \begin{QEMPTYSPACE}
    \end{QEMPTYSPACE}
\end{QUESTION}
\begin{QUESTION}
    \begin{QBODY}
        $a>0$,$f(x)={{({{a}^{x}}-3)}^{2}}+{{({{a}^{-x}}-3)}^{2}}$之最小值為$\originalAnsBox$。 
    \end{QBODY}
    \begin{QFROMS}
    \end{QFROMS}
    \begin{QTAGS}\QTAG{B1C3-2} \end{QTAGS}
    \begin{QANS}
        $7$
    \end{QANS}
    \begin{QSOL}
    \end{QSOL}
    \begin{QEMPTYSPACE}
    \end{QEMPTYSPACE}
\end{QUESTION}
\begin{QUESTION}
    \begin{QBODY}
         一曲線 $y={{\left( \frac{1}{3} \right)}^{x}}$ 分別與兩直線 ${{L}_{1}}:y=4$,${{L}_{2}}:y=12$ 交於 $P,Q$ 兩點,則直線$\overleftrightarrow{PQ}$ 的斜率為 $\originalAnsBox $。
    \end{QBODY}
    \begin{QFROMS}
    \end{QFROMS}
    \begin{QTAGS}\QTAG{B1C3-2} \end{QTAGS}
    \begin{QANS}
        $-3$
    \end{QANS}
    \begin{QSOL}
    \end{QSOL}
    \begin{QEMPTYSPACE}
    \end{QEMPTYSPACE}
\end{QUESTION}
\begin{QUESTION}
    \begin{QBODY}
        一城市的人口密度是依 $P\left( x \right)\text{=}A\times {{e}^{-kx}}$ (人/$k{{m}^{2}}$) 來決定的,其中 $A$ 和 $e$ 是兩個大於 0 的常數,而 $x$ 是指與市中心的距離(單位為 $km$ )。如果已知市中心的人口密度為10000 (人/$k{{m}^{2}}$),且距離市中心 10 公里處的人口密度為 5000(人/$k{{m}^{2}}$),則求距離市中心 30 公里處的人口密度為 $\originalAnsBox$(人/$k{{m}^{2}}$)
    \end{QBODY}
    \begin{QFROMS}
    \end{QFROMS}
    \begin{QTAGS}\QTAG{B1C3-2} \end{QTAGS}
    \begin{QANS}
        $1250$
    \end{QANS}
    \begin{QSOL}
    \end{QSOL}
    \begin{QEMPTYSPACE}
    \end{QEMPTYSPACE}
\end{QUESTION}
\end{QUESTIONS}
    


